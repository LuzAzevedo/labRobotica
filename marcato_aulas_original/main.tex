

%----------------------------------------------------------------------------------------
% OPÇÕES DE PACOTES (ANTES DO DOCUMENTCLASS)
%----------------------------------------------------------------------------------------
\PassOptionsToPackage{table}{xcolor}  % Passa opção [table] para xcolor (beamer carrega xcolor automaticamente)

%----------------------------------------------------------------------------------------
% CLASSE DO DOCUMENTO E CONFIGURAÇÕES BÁSICAS
%----------------------------------------------------------------------------------------
\documentclass[
    11pt,               % Tamanho padrão da fonte
    % t,                % Alinhar verticalmente ao topo
    %aspectratio=169,   % Definir proporção 16:9
]{beamer}
\graphicspath{{img/}}         % Define o diretório das imagens

%----------------------------------------------------------------------------------------
% PACOTES NECESSÁRIOS (ORDEM IMPORTANTE)
%----------------------------------------------------------------------------------------
\usepackage{
    booktabs,     % Melhora a aparência das linhas em tabelas
    palatino,     % Define Palatino como fonte principal
    subcaption    % Suporte para subfiguras
}
% \usepackage[default]{opensans}  % Define Open Sans como fonte secundária (comentado: pacote não disponível)
\usepackage{listings}           % Para exibição de código (deve ser carregado antes do config)
\input{config/code_langs}       % Importa configurações para highlight de código

%----------------------------------------------------------------------------------------
% CONFIGURAÇÃO DO TEMA
%----------------------------------------------------------------------------------------
% Tema Base
\usetheme{Boadilla}                          % Define o tema principal
\useinnertheme{circles}                      % Tema interno com círculos
\useoutertheme{miniframes}                   % Tema externo com miniframes
\setbeamertemplate{navigation symbols}{}     % Remove símbolos de navegação

% Cores Personalizadas
\definecolor{primaryColor}{RGB}{20,45,105}   % Cor primária - azul escuro
\definecolor{secondaryColor}{RGB}{0,100,160} % Cor secundária - azul médio

% Configurações de Cores
\setbeamercolor{structure}{fg=primaryColor}
\setbeamercolor{palette primary}{bg=primaryColor, fg=white}
\setbeamercolor{palette secondary}{bg=secondaryColor, fg=white}
\setbeamercolor{title}{bg=primaryColor, fg=white}

% Cores do Cabeçalho e Rodapé
%\setbeamercolor{headline}{bg=secondaryColor, fg=white}
%\setbeamercolor{section in head/foot}{bg=primaryColor, fg=white}
\setbeamercolor{subsection in head/foot}{bg=secondaryColor, fg=white}
\setbeamercolor{author in head/foot}{bg=primaryColor, fg=white}
\setbeamercolor{title in head/foot}{bg=secondaryColor, fg=white}
\setbeamercolor{date in head/foot}{bg=primaryColor, fg=white}
\setbeamercolor{page number in head/foot}{bg=primaryColor, fg=white}


% Inseridos André Marcato
\usepackage{hyperref}
\usepackage{graphicx}  % Para inserir imagens
\usepackage{multicol}  % Para múltiplas colunas
\usepackage{tcolorbox} % Para criar caixas de destaque
\usepackage{tikz}
\usepackage[normalem]{ulem}


% Definir comando para adicionar a marca d'água
%\setbeamertemplate{background}{%
%    \begin{tikzpicture}[remember picture,overlay]
%        \node[opacity=0.2] at (current page.center) {
%            \includegraphics[width=1.0\paperwidth]{img/marcadagua2.png} % Ajuste o %tamanho conforme necessário
%        };
%    \end{tikzpicture}
%}

% Configuração do estilo para código Python
\definecolor{darkgreen}{rgb}{0.0, 0.5, 0.0}
\lstdefinestyle{pythonstyle}{
    language=Python,
    basicstyle=\ttfamily\scriptsize,
    keywordstyle=\color{blue},
    stringstyle=\color{red},
    commentstyle=\color{darkgreen},
    numbers=left,
    numberstyle=\tiny,
    stepnumber=1,
    frame=single,
    breaklines=true,
    showstringspaces=false,
    xleftmargin=10pt,
    xrightmargin=5pt
}

\usepackage{mdframed}  

%----------------------------------------------------------------------------------------
% BIBLIOGRAFIA
%----------------------------------------------------------------------------------------
% \usepackage[style=alphabetic,backend=biber]{biblatex}  % Comentado: pacote biblatex não disponível
% \addbibresource{bibliografia.bib}  % Comentado: requer biblatex

%----------------------------------------------------------------------------------------
% INFORMAÇÕES DA APRESENTAÇÃO
%----------------------------------------------------------------------------------------

% Título Aula01
\title[Robótica Móvel]{Plano de Curso \\ Orientações Gerais \\ Motivação}    

% Título Aula02
\title[Robótica Móvel]{Introdução ao ROS \\ Robotic Operating System}

% Título Aula03
\title[Robótica Móvel]{Introdução à Robótica Móvel}

% Título Aula04
\title[Robótica Móvel]{Locomoção}

% Título Aula05
\title[Robótica Móvel]{Sensores}

%\title[Robótica Móvel]{Trabalho Sobre o YOLO}

\author[Juiz de Fora]{Prof. André Luís Marques Marcato}            
\institute[Brasil]{Robótica Móvel}
\date[Prof. André Marcato]{Primeiro Semestre / 2025}

%----------------------------------------------------------------------------------------
% INÍCIO DO DOCUMENTO
%----------------------------------------------------------------------------------------
\begin{document}

% Slide de título com logo
\begin{frame}
%    \begin{center}
%    {\small Preparado para:}
%    \end{center}
%    \begin{figure}
%        \includegraphics[width=0.25\linewidth]{img/Jirau.png}
%    \end{figure}
    \titlepage
\end{frame}

% Sumário
\begin{frame}
    \frametitle{Estrutura da apresentação}
    \tableofcontents
\end{frame}

% (AULA 01) - Plano de Curso
%\section{Bibliografia}


\begin{frame}
	\frametitle{Livros Texto}
	\only<1>
	{
		  \centering
		  \includegraphics[width=4.5cm, height=6cm]{Figuras/RobotProgramming.png}
	}

        \only<2>
	{
		\begin{columns}
			\begin{column}{0.5\textwidth}
				\centering
				\includegraphics[width=4.5cm, height=6cm]{Figuras/ROS_Springer.jpg}
			\end{column} 
			\begin{column}{0.5\textwidth}
				\centering
				\includegraphics[width=4.5cm, height=6cm]{Figuras/Learning.jpg}
			\end{column} 
		\end{columns}
	}	

        \only<3>
	{
		\begin{columns}
			\begin{column}{0.5\textwidth}
				\centering
				\includegraphics[width=4.5cm, height=6cm]{Figuras/LivroPython1.png}
			\end{column} 
			\begin{column}{0.5\textwidth}
				\centering
				\includegraphics[width=4.5cm, height=6cm]{Figuras/LivroPython2.png}
			\end{column} 
		\end{columns}
	}	
\end{frame}

\begin{frame}
	\frametitle{ROS Website (\href{www.ros.org}{www.ros.org})} 
	\only<1>
	{
		\includegraphics[width=9cm, height=6cm]{Figuras/ROS_Website.png} 
	}
	\only<2>
	{
		\includegraphics[width=9cm, height=6cm]{Figuras/ROS_Website_2.png} 
	}
\end{frame}

\section{Motivação}

\begin{frame}
	\frametitle{Ciclo de um Processo em Robótica} 
	\centering

	\begin{tikzpicture}
						[squarednode/.style={rectangle, draw=red!60, fill=red!5, very thick, minimum size = 10mm},]
		\only<1-3>
		{
			\node[squarednode] 									(sense)	{Sensores};
		}
		\only<2-3>
		{	
			\node[squarednode, right of=sense,xshift=2.4cm] 	(think) {Processamento};
		}
		\only<3->
		{	
			\node[squarednode, right of=think,xshift=2.4cm] 	(act)	{Atuação};
		}
		\only<1-3>
		{
			\node[inner sep=0pt, above of=sense, yshift=1cm] 	(fig_sense) 
			{
				\includegraphics[width=.25\textwidth]{Figuras/vision.jpg}
			};
		}
		\only<2-3>
		{		
			\node[inner sep=0pt, above of=think, yshift=1cm] 	(fig_think) 
			{
				\includegraphics[width=.25\textwidth]{Figuras/brain.jpg}
			};
		}
		\only<3->
		{		
			\node[inner sep=0pt, above of=act, yshift=1cm] 		(fig_act) 
			{
				\includegraphics[width=.25\textwidth]{Figuras/act.png}
			};
		}			
		\only<2-3>
		{	
			\draw[->, thick, draw=red!60] (sense.east) -- (think.west) ;
		}
		\only<3->
		{
			\draw[->, thick, draw=red!60] (think.east) -- (act.west) ;
		}
	\end{tikzpicture}
	
\end{frame}

\begin{frame}
	\frametitle{Sensores}
	\centering
	\begin{tikzpicture}
	[vermelho/.style={rectangle, draw=red!60, 	fill=red!5, 	very thick, minimum size = 10mm},
	verde/.style	={rectangle, draw=green!60, fill=green!5, 	very thick, minimum size = 10mm},]
	
	\node[vermelho] 								(sense)		{Sensores};
	\node[verde, right of=sense	 	,xshift=2.4cm] 	(ultrassom) {Ultrassom};
	\node[verde, above of=ultrassom	,yshift=1cm] 	(lasers)	{Laser Scan};
	\node[verde, above of=lasers	,yshift=1cm] 	(camera)	{Câmera};
	\node[verde, below of=ultrassom	,yshift=-1cm] 	(GPS)		{GPS};	
	\node[inner sep=0pt, above of=sense, yshift=1cm] 	(fig_sense) 
	{
		\includegraphics[width=.25\textwidth]{Figuras/vision.jpg}
	};

	\node[inner sep=0pt, right of=camera, xshift=2.4cm] 	(fig_camera) 
	{
		\includegraphics[width=.15\textwidth]{Figuras/camera.jpg}
	};

	\node[inner sep=0pt, right of=lasers, xshift=2.4cm] 	(fig_lasers) 
	{
		\includegraphics[width=.10\textwidth]{Figuras/lidar.jpg}
	};

	\node[inner sep=0pt, right of=ultrassom, xshift=2.4cm] 	(fig_ultrassom) 
	{
		\includegraphics[width=.15\textwidth]{Figuras/ultrassonic.jpg}
	};
	
	\node[inner sep=0pt, right of=GPS, xshift=2.4cm] 	(fig_gps) 
	{
		\includegraphics[width=.10\textwidth]{Figuras/gps.jpg}
	};
	
	\end{tikzpicture}
\end{frame}
%------------------------------------------------


\begin{frame}
	\frametitle{Processamento}
	\centering
	\begin{tikzpicture}
	[vermelho/.style={rectangle, draw=red!60, 	fill=red!5, 	very thick, minimum size = 10mm},
	verde/.style	={rectangle, draw=green!60, fill=green!5, 	very thick, minimum size = 10mm},]
	
	\node[vermelho] 								(think)		{Processamento};
	\node[verde, right of=think	 	,xshift=2.9cm] 	(obi) 		{Otimização Bioinspirada};
	\node[verde, above of=obi		,yshift=1cm] 	(mdl)		{Machine/Deep Learning};
	\node[verde, above of=mdl		,yshift=1cm] 	(ai)		{Inteligência Artificial};
	\node[verde, below of=obi		,yshift=-1cm] 	(signal)	{Processamento de Sinais};	
	\node[inner sep=0pt, above of=think, yshift=1cm] (fig_think) 
	{
		\includegraphics[width=.15\textwidth]{Figuras/brain.jpg}
	};
	
	\node[inner sep=0pt, right of=ai, xshift=2.4cm] 	(fig_ai) 
	{
		\includegraphics[width=.25\textwidth]{Figuras/ai.png}
	};
	
	\node[inner sep=0pt, right of=mdl, xshift=2.4cm] 	(fig_mdl) 
	{
		\includegraphics[width=.15\textwidth]{Figuras/machinelearning.png}
	};
	
	\node[inner sep=0pt, right of=signal, xshift=2.4cm] (fig_signal) 
	{
		\includegraphics[width=.20\textwidth]{Figuras/processamentosinais.png}
	};
	
	\node[inner sep=0pt, right of=obi, xshift=2.4cm] 	(fig_obi) 
	{
		\includegraphics[width=.15\textwidth]{Figuras/gohb.png}
	};
	
	\end{tikzpicture}
\end{frame}
%------------------------------------------------

\begin{frame}
	\frametitle{Processamento}
	{
		\begin{itemize}
			\item Visual Odometry / SLAM / Outros algoritmos de visão
			\item Filtro de Partículas
			\item Algoritmos de Localização
			\item Planejamento de Caminhos e Trajetórias
			\item Controle (PID, Realimentação de estados)
			\item Filtro de Kalman
		\end{itemize}
	}
\end{frame}

\begin{frame}
	\frametitle{Atuadores}
	\centering
	\begin{tikzpicture}
		[vermelho/.style={rectangle, draw=red!60, 	fill=red!5, 	very thick, minimum size = 10mm},
		verde/.style	={rectangle, draw=green!60, fill=green!5, 	very thick, minimum size = 10mm},]
	
		\node[vermelho] 							(act)		{Atuadores};
		\node[verde, right of=act ,xshift=2.9cm] 	(uav) 		{Motores de UAV (Drones)};
		\node[verde, above of=uav ,yshift=1cm] 		(ser)		{Servo Motores};
		\node[verde, above of=ser ,yshift=1cm] 		(auv)		{Motores Aquáticos (AUV)};
		\node[verde, below of=uav ,yshift=-1cm] 	(gas)		{Motores Combustão};	
		\node[inner sep=0pt, above of=think, yshift=1cm] (fig_act) 
		{
			\includegraphics[width=.25\textwidth]{Figuras/act.png}
		};
	
		\node[inner sep=0pt, right of=uav, xshift=2.7cm] 	(fig_uav) 
		{
			\includegraphics[width=.15\textwidth]{Figuras/uav.jpg}
		};
	
		\node[inner sep=0pt, right of=ser, xshift=2.7cm] 	(fig_ser) 
		{
			\includegraphics[width=.15\textwidth]{Figuras/servo.jpg}
		};
	
		\node[inner sep=0pt, right of=auv, xshift=2.7cm] 	(fig_auv) 
		{
			\includegraphics[width=.15\textwidth]{Figuras/auv.jpg}
		};
	
		\node[inner sep=0pt, right of=gas, xshift=2.7cm] 	(fig_gas) 
		{
			\includegraphics[width=.15\textwidth]{Figuras/gas.jpg}
		};
	
	\end{tikzpicture}	
\end{frame}

\begin{frame}
    \frametitle{Motivação para o ROS}
    \begin{columns}
        \only<1-2>
        {
            \begin{column}{0.5\textwidth}
                \centering
                \includegraphics[width=4.3cm, height=6.5cm]{Figuras/Re-Inventing.png}
            \end{column}
        }
        \only<2>
        {
            \begin{column}{0.5\textwidth}
                \href{https://spectrum.ieee.org/automaton/robotics/robotics-software/the-origin-story-of-ros-the-linux-of-robotics}{The Origin Story of ROS, the Linux of Robotics} \\~\\
                \scriptsize
                "The world's most influential robotics software platform"
            \end{column}
        }	
    \end{columns}	
\end{frame}

\begin{frame}
    \frametitle{Motivação para o ROS}
    \centering
    \includegraphics[width=8cm, height=6.5cm]{img02/ciclo_ros.png}
\end{frame}

\section{Objetivos}

\begin{frame}
    \frametitle{Objetivos do Curso}
    \begin{itemize}
        \item Entender o Ecossistema ROS (tópicos, nós, mensagens, serviços)
        \item Desenvolver aplicações para controlar o movimento de um robô
        \item Entender como a posição e orientação são representadas no ROS
        \item Desenvolver programas simples utilizando visão computacional
        \item Utilizar os simuladores (Exemplo: Gazebo)
    \end{itemize}
\end{frame}

\begin{frame}
    \frametitle{\href{https://www.ros.org/}{https://www.ros.org/}}
    \only<1>
    {
        \centering
		\includegraphics[width=10cm, height=6cm]{img02/siteROS1.png}        
    }

    \only<2>
    {
        \centering
        \includegraphics[width=10cm, height=6cm]{img02/siteROS2.png}
    }
\end{frame}

\section{História}
\begin{frame}
	\frametitle{História do ROS}
	\begin{itemize}
		\scriptsize
		\item \href{https://www.linkedin.com/in/eric-berger-806b3b3/}{Eric Berger} e {Keenan Wyrobek} começaram o Doutorado (Ph.D) em Stanford...
		\item Buscaram levantar fundos para o desenvolvimento do projeto o Linux da Robótica.
		\item PR2 (Personal Robotics) - \href{https://www.youtube.com/watch?v=oyHWkQcin7I&feature=youtu.be}{\underline{ver vídeo}}
		\item Atualmente: Ecossistema ROS - Qualquer grupo pode iniciar um repositório de código do ROS ("\textit{federated model}")
	\end{itemize}
			
	\begin{table}
		\begin{tabular}{ | >{\centering\arraybackslash}m{1cm}  >{\centering\arraybackslash}m{7cm} | }  
			\hline
			\includegraphics[width=0.6cm,height=0.6cm]{Figuras/stanford.png} & Stanford Personal Robotics Program (janeiro de 2007)\\
			\hline
			\includegraphics[width=0.6cm,height=0.6cm]{Figuras/willowgarage.jpg} & Laboratório de pesquisa em robótica e incubadora tecnológica (Novembro de 2007) \\
			\hline
			\includegraphics[width=0.6cm,height=0.6cm]{Figuras/openrobotics.png} & Open Source Robotics Foundation (OSRF) ou Open Robotics (Fevereiro de 2013)\\
			\hline
		\end{tabular}
	\end{table}
	
	\centering
	\scriptsize
	"Don’t let anyone crush your crazy" ou "Não deixe ninguém esmagar sua loucura"
\end{frame}


\begin{frame}
	\frametitle{RosCon - ROS Conference (\href{https://roscon.ros.org/2024/} {https://roscon.ros.org/2024/})} 
        \begin{columns}
            \begin{column}{0.5\textwidth}
        	\centering
                \only<1>
                {
    	          \includegraphics[width=4cm,height=6cm]{img02/ROSCon1.png} \\~\\
                }
                \only<2>
                {
    	          \includegraphics[width=6cm,height=3.5cm]{img02/ROSCon2.png} \\~\\
                }
            \end{column}
            \begin{column}{0.5\textwidth}
        	\scriptsize
        	Conjugada com o IROS (ou ICRA):
        	\begin{itemize}
            		\item \href{https://www.iros25.org/}{IEEE/RSJ International Conference on Intelligent Robots and Systems}
            		\item \href{(https://2025.ieee-icra.org/}{International Conference on Robotics and Automation}
            	\end{itemize}
            \end{column}
        \end{columns}
\end{frame}



%\section{Contato}

\begin{frame}
    \frametitle{Sensores de Contato}
    \centering
    \begin{columns}
        \begin{column}{0.5\textwidth}
            \centering
            \begin{tabular}{c c c}
                \includegraphics[width=0.3\textwidth]{img05/contato_1.png} &
                \includegraphics[width=0.3\textwidth]{img05/contato_2.png} &
                \includegraphics[width=0.3\textwidth]{img05/contato_3.png} \\
            \end{tabular}
            \includegraphics[width=0.7\textwidth]{img05/contato_4.png} \\            
        \end{column}
        \begin{column}{0.5\textwidth}
            \begin{itemize}
                \scriptsize
                \item Os sensores táteis, adquirem a informação de contato com o ambiente por meio de interação física.
                \item Podem servir a diversos propósitos, tal qual um para-choque ou então um sensor que identifica a uma garra quando parar, ou em mãos robóticas, informando que o toque ocorreu.
                \item Os mais comuns empregam chaves normalmente abertas e fechadas, efeitos piezelétricos e piezoresistivos, capacitivos e assim por diante.
            \end{itemize}
        \end{column}    
    \end{columns}
\end{frame}

\section{Encoder}
\begin{frame}
    \frametitle{Sensor de movimento: Encoder}
    \centering
    \begin{columns}
        \begin{column}{0.5\textwidth}
            \centering
            \begin{tabular}{c c}
                \includegraphics[width=0.4\textwidth]{img05/encoder_1.png} &
                \includegraphics[width=0.4\textwidth]{img05/encoder_2.png} \\
            \end{tabular}
            \includegraphics[width=0.5\textwidth]{img05/lente_fotografica.png} \\            
        \end{column}
        \begin{column}{0.5\textwidth}
            \begin{itemize}
                \scriptsize
                \item É um dispositivo eletromecânico que conta ou reproduz pulsos elétricos a partir do movimento rotacional de seu eixo.
                \item Pode ser definido como um transdutor de posição angular.       
                \item \textbf{Exemplo de utilização de encoders em robótica}: medir a posição ou velocidade de motores utilizados para mover robôs.
                \item Se os valores dos encoders forem integrados (utilize um integrador digital), proporcionarão uma estimação da posição do robô. A posição do robô é chamada de \textbf{hodometria}.
            \end{itemize}
        \end{column}    
    \end{columns}
\end{frame}

\begin{frame}
    \frametitle{Sensor de movimento: Encoder}
    \begin{columns}
        \begin{column}{0.5\textwidth}
            \centering
            \scriptsize
            \textbf{Funcionamento regular}: o aparato conta o número de incrementos/transições (1 para sinal elevado – houve uma transição – e 0 para sinal baixo – ainda não houve uma transição) mas não pode dizer a direção do movimento.
        \end{column}
        \begin{column}{0.5\textwidth}
            \begin{itemize}
                \scriptsize
                \item Os encoders são sensores \textbf{proprioceptivos}, pois medem estados internos do robô (no caso, posição e velocidade).
                \item Dependendo do fabricante, encontra-se de diversas resoluções, desde 64 a mais de 50.000 incrementos (“ticks”, pulsos, contagens) por cada revolução do eixo do motor.
                \item Para maiores resoluções pode-se utilizar interpolação.
            \end{itemize}
        \end{column}
    \end{columns}
    \centering
    \includegraphics[width=0.7\textwidth]{img05/encoder_3.png} \\            
\end{frame}

\begin{frame}
    \frametitle{Sensor de movimento: Encoder}
    \begin{columns}
        \begin{column}{0.5\textwidth}
            \centering
            \scriptsize
            \textbf{Funcionamento em quadratura}: utiliza-se dois sensores deslocados entre si em fase de 90º (quadratura, um elevado e outro baixo). A ordem de qual sensor produz sinal elevado (1) primeiro diz a direção do movimento. Além da direção, a resolução neste modo de funcionamento é 4 vezes maior.
        \end{column}
        \begin{column}{0.5\textwidth}
            \centering
            \includegraphics[width=0.5\textwidth]{img05/encoder_4.png} \\            
        \end{column}
    \end{columns}
    \begin{columns}
        \begin{column}{0.5\textwidth}
            \centering
            \includegraphics[width=0.7\textwidth]{img05/encoder_6.png} \\            
        \end{column}
        \begin{column}{0.5\textwidth}
            \centering
            \includegraphics[width=0.7\textwidth]{img05/encoder_5.png} \\            
        \end{column}
    \end{columns}
\end{frame}

\section{Giros, Acelerômetros, IMUs}
\begin{frame}
    \frametitle{Sensor de orientação: Giroscópio}
    \centering
    \begin{columns}
        \begin{column}{0.5\textwidth}
            \centering
            \begin{tabular}{c}
                \includegraphics[width=0.8\textwidth]{img05/giroscópio.png} \\
            \end{tabular}
        \end{column}
        \begin{column}{0.5\textwidth}
            \begin{itemize}
                \scriptsize
                \item Os \textbf{ópticos} utilizam lasers emitidos em duas direções, uma viajando no sentido horário e outra no sentido anti-horário.
                \item Quando o aparato gira, as fases relativas dos lasers são deslocadas de acordo com a velocidade angular do movimento, uma vez que para um dos lasers o caminho ficou mais longo, e para o outro mais curto (efeito Sagnac).
                \item Mede-se a diferença de fase entre os dois feixes, e essa diferença entre as fases dos feixes é proporcional à velocidade angular do corpo e, consequentemente, ao ângulo do aparato girante.
                \item Ring Laser Gyroscope, RLG, que utiliza o efeito Sagnac
            \end{itemize}
        \end{column}    
    \end{columns}
\end{frame}

\begin{frame}
    \frametitle{Sensor de Inércia: Acelerômetros}
    \begin{columns}
        \begin{column}{0.4\textwidth}
            \centering
            \includegraphics[width=0.7\textwidth]{img05/massa_mola_amortecedor.png}
            \begin{mdframed}[%
                            backgroundcolor=blue!20,   % cor de fundo
                            linecolor=red,               % cor da borda
                            linewidth=1pt,               % espessura da borda
                            roundcorner=4pt,             % cantos arredondados
                            innertopmargin=6pt,          % espaço interno acima
                            innerbottommargin=6pt,       % espaço interno abaixo
                            innerleftmargin=6pt,         % espaço interno à esquerda
                            innerrightmargin=6pt         % espaço interno à direita
                            ]
                \scriptsize
                \begin{itemize}
                    \scriptsize
                    \item Na superfície da Terra, o acelerômetro sempre indicará, pelo menos, 1g ao longo do eixo vertical, e 0g em queda-livre.
                    \item Por isso, para obter a aceleração inercial (correspondente ao movimento do corpo), é necessário subtrair a gravidade.
                \end{itemize}
            \end{mdframed}
        \end{column}
        \begin{column}{0.6\textwidth}
            \centering
            \begin{itemize}
                \scriptsize
                \item Conforme os princípios físicos, para alterar movimentos de corpos é necessário imprimir forças sobre eles.
                \item Os acelerômetros são sensores que medem todas as forças que estejam atuando sobre eles, incluindo a gravidade.
                \item Os acelerômetros atuam como sistemas massa-mola-amortecedor, equacionado como:
            \end{itemize}
            \begin{equation*}
                F_{\text{aplicada}} = F_{\text{inercial}} + F_{\text{amortecida}} + F_{\text{elástica}} 
            \end{equation*}
            \begin{equation*}
                F_{\text{aplicada}} = m\ddot{x} + c\dot{x} + kx
            \end{equation*}
            \begin{itemize}
                \scriptsize
                \item[] Onde:
                \item \textbf{m} é a massa de prova.
                \item \textbf{c} é o coeficiente de amortecimento.
                \item \textbf{k} é a constante da mola.
            \end{itemize}
        \end{column}
    \end{columns}
\end{frame}

\begin{frame}
    \frametitle{Unidades de Medidas Inerciais (IMU)}
    \centering
    \begin{columns}
        \begin{column}{0.5\textwidth}
            \centering
            \begin{tabular}{c}
                \includegraphics[width=0.9\textwidth]{img05/IMU.png} \\
            \end{tabular}
        \end{column}
        \begin{column}{0.5\textwidth}
            \begin{itemize}
                \scriptsize
                \item As \textbf{Inertial Measurement Units (IMUs)} são dispositivos que medem posições relativas (x, y, z), orientações (roll, pitch, yaw), velocidades e acelerações de um corpo em movimento.
                \item Combinam as funções dos acelerômetros, giroscópios, bússolas além de possuírem integradores para o cálculo de hodometria.
                \item São unidades inerciais completas, sensíveis a erros, corrigindo a hodometria simples (dead reckoning) provida pelos encoders.
            \end{itemize}
        \end{column}    
    \end{columns}
\end{frame}

\section{Câmeras}
\begin{frame}
    \frametitle{Câmeras}
    \centering
    \begin{columns}
        \begin{column}{0.5\textwidth}
            \centering
            \begin{tabular}{c}
                \includegraphics[width=0.7\textwidth]{img05/câmera_2.png} \\
                \includegraphics[width=0.7\textwidth]{img05/câmera_1.png} \\
            \end{tabular}
        \end{column}
        \begin{column}{0.5\textwidth}
            \begin{itemize}
                \scriptsize
                \item As câmeras são sensores bastante populares, que podem ser utilizados das mais variadas formas.
                \item A área de pesquisa em imagens é denominada de visão computacional, em que a visão robótica é uma das áreas.
                \item Por meio de técnicas de manipulação de imagens, é possível obter distâncias, características (features), identificar objetos e pessoas, obter caminhos, dentre muitas outras aplicações.
            \end{itemize}
        \end{column}    
    \end{columns}
\end{frame}

\begin{frame}
    \frametitle{Câmeras - Knectic}
    \centering
    \includegraphics[width=0.95\textwidth]{img05/kinect.png}
\end{frame}

\begin{frame}{O que é a Câmera Kinect?}
    \begin{itemize}
        \item Sensor desenvolvido originalmente pela Microsoft para o Xbox.
        \item Usa visão computacional para captar profundidade, cor e movimento.
        \item Combina:
        \begin{itemize}
            \item Câmera RGB (imagem tradicional)
            \item Sensor de profundidade (depth camera)
            \item Microfones e acelerômetros (em versões completas)
        \end{itemize}
    \end{itemize}
    \end{frame}

\begin{frame}{Câmera Kinect: Como Funciona?}
    \begin{itemize}
        \item A câmera de profundidade projeta um padrão infravermelho (IR).
        \item O padrão refletido é captado por um sensor e processado para estimar a distância de cada ponto.
        \item Resulta em uma \textbf{imagem de profundidade} (depth map), com cada pixel representando uma distância.
    \end{itemize}
    
    \vspace{0.5cm}
    \textbf{Alternativamente (em modelos mais novos):} usa tecnologia de \textit{Time-of-Flight (ToF)} para medir o tempo que a luz leva para voltar ao sensor.
\end{frame}

\begin{frame}{Câmera Kinect: Vantagens}
    \begin{itemize}
        \item Captura simultânea de imagem colorida e profundidade.
        \item Excelente para rastreamento de corpo humano e gestos.
        \item Pronto para uso com SDKs acessíveis.
        \item Operação em tempo real com boa precisão.
    \end{itemize}
\end{frame}

\begin{frame}{Câmera Kinect: Aplicações}
    \begin{itemize}
        \item Jogos e interfaces baseadas em gestos.
        \item Robótica (percepção 3D e navegação).
        \item Mapeamento de ambientes (SLAM).
        \item Realidade aumentada e captura de movimento.
        \item Reconhecimento de pessoas e gestos em segurança.
    \end{itemize}
\end{frame}

\begin{frame}{O que são Câmeras Baseadas em Eventos?}
    \begin{itemize}
        \item Diferente das câmeras tradicionais, que capturam imagens em quadros fixos.
        \item Cada pixel funciona de forma independente e assíncrona.
        \item Gera eventos apenas quando há variação significativa de luminosidade.
    \end{itemize}
    
    \vspace{0.5cm}
    \textbf{Cada evento contém:}
    \begin{itemize}
        \item Posição do pixel $(x, y)$
        \item Tempo do evento (alta precisão temporal)
        \item Polaridade da variação (aumento ou redução de brilho)
    \end{itemize}
    \end{frame}

\begin{frame}{Câmera Baseada em Eventos: Vantagens}
    \begin{itemize}
        \item Altíssima resolução temporal (microsegundos)
        \item Baixa latência e baixo consumo de energia
        \item Alta faixa dinâmica (HDR) – ideal para ambientes com iluminação extrema
        \item Excelente desempenho com objetos rápidos
    \end{itemize}
\end{frame}

\begin{frame}{Câmera Baseada em Eventos: Aplicações}
    \begin{itemize}
        \item Robótica e drones autônomos
        \item Carros autônomos
        \item Sistemas biomiméticos (inspirados na retina)
        \item Visão computacional de alta velocidade
    \end{itemize}
    \href{https://youtu.be/0wGBpgIrd9M?si=j-tcaOsuDFyUieXV}{Link: Davide Scaramuzza }
\end{frame}

\section{GPS}
\begin{frame}
    \frametitle{Global Positioning System (GPS)}
    \centering
    \begin{columns}
        \begin{column}{0.4\textwidth}
            \centering
            \begin{tabular}{c}
                \includegraphics[width=0.8\textwidth]{img05/gps.png}\\
            \end{tabular}
        \end{column}
        \begin{column}{0.6\textwidth}
            \begin{itemize}
                \scriptsize
                \item O Sistema de Posicionamento Global é um sistema de localização global baseado em satélites e trilateração.
                \item Desenvolvido originalmente pelos USA nas décadas de 1950 e 1960 para estratégias exclusivamente militares, foi liberado  uso pela população civil na metade da década de 1990.
                \item Outros países desenvolveram seus próprios sistemas baseados em satélites, como o COMPASS (ou, BeiDou) da China, operacional a partir da década de 2000, o GLONASS russo, contemporâneo ao GPS, e o GALILEO da União Europeia e o NAVIC da Índia.
                \item O projeto original do GPS contava com 24 satélites em órbitas de 12 horas a uma altura de, aproximadamente, 21000 km acima da superfície terrestre, já no meio interplanetário.
            \end{itemize}
        \end{column}    
    \end{columns}
\end{frame}

\begin{frame}{Sistemas de Posicionamento e Segurança Cibernética}
    \textbf{Contexto:}
    \begin{itemize}
        \item O GPS foi desenvolvido pelos EUA com objetivos militares e só posteriormente liberado ao uso civil.
        \item A dependência exclusiva de um sistema estrangeiro representa um risco estratégico e cibernético.
        \item Outros países criaram seus próprios sistemas (GLONASS - Rússia, BeiDou - China, GALILEO - UE, NAVIC - Índia) para garantir autonomia.
    \end{itemize}
    
    \vspace{0.5cm}
    \textbf{Riscos associados à dependência do GPS:}
    \begin{itemize}
        \item Sinal pode ser bloqueado, degradado ou falsificado (\textit{spoofing}) em tempos de crise.
        \item Impacto direto em setores críticos: telecomunicações, transporte, energia e defesa.
        \item Falta de controle sobre atualizações, segurança e integridade dos dados recebidos.
    \end{itemize}
\end{frame}

\begin{frame}{Independência Tecnológica e Geopolítica}
    \begin{itemize}
        \item O domínio de sistemas de navegação por satélite está ligado à soberania nacional e geopolítica.
        \item Ter um sistema próprio ou acesso a múltiplos sistemas reduz vulnerabilidades.
        \item Integração com estratégias de defesa cibernética e proteção de infraestrutura crítica.
        \item Investimentos em sistemas nacionais impulsionam inovação e capacitação tecnológica interna.
    \end{itemize}
    
    \vspace{0.3cm}
    \textbf{Conclusão:} Em um mundo cada vez mais digital e conectado, o controle autônomo sobre sistemas espaciais é um ativo estratégico fundamental.
\end{frame}

\begin{frame}
    \frametitle{Global Positioning System (GPS)}
    \centering
    \begin{columns}
        \begin{column}{0.4\textwidth}
            \centering
            \begin{tabular}{c}
                \includegraphics[width=0.8\textwidth]{img05/satelite_gps.png}\\
            \end{tabular}
        \end{column}
        \begin{column}{0.6\textwidth}
            \begin{itemize}
                \scriptsize
                \item Os satélites de GPS se chamam NAVSTAR (NAVigation Satellite with Time And Ranging).
                \item Cada satélite possui um relógio atômico para sincronia de posicionamento, corrige o efeito Doppler na transmissão de seus dados por meio da Teoria da Relatividade.
                \item As órbitas desses satélites são tais que, a todo momento no horizonte de um usuário, existam, pelo menos, quatro satélites disponíveis.
            \end{itemize}
        \end{column}    
    \end{columns}
\end{frame}

\begin{frame}
    \frametitle{Global Positioning System (GPS)}
    \begin{itemize}
        \scriptsize
        \item A localização do GPS é feita por meio do princípio matemático da trilateração. No caso, trilateração em 3D.
        \item Existem estações GPS bem estabelecidas por todo o planeta.
    	\item A trilateração em 2D é facilmente explicada da seguinte maneira: 
    \end{itemize}
    \only<1>
    {
        \begin{itemize}
            \scriptsize
            \item[] Uma fonte (ex: satélite) lhe revela que está a uma distância d1 de uma estação conhecida (ex: numa cidade) denominada A.
        \end{itemize}
        \centering                
        \includegraphics[width=0.8\textwidth]{img05/gps_1.png}\\
    }
    \only<2>
    {
        \begin{itemize}
            \scriptsize
            \item[] Uma segunda fonte (outro satélite) informa que o usuário está a uma distância d2 de um outro ponto conhecido B.
        \end{itemize}
        \centering                
        \includegraphics[width=0.9\textwidth]{img05/gps_2.png}\\
    }
    \only<3>
    {
        \begin{itemize}
            \scriptsize
            \item[] Para eliminar a ambiguidade, é necessário obter a informação de uma terceira fonte (satélite), de uma distância d3 de C.
        \end{itemize}
        \centering                
        \includegraphics[width=0.9\textwidth]{img05/gps_3.png}\\
    }
    \only<4>
    {
        \centering                
        \includegraphics[width=0.95\textwidth]{img05/gps_4.png}\\
    }
\end{frame}

%\section{Arquiteturas}

\begin{frame}
    \frametitle{P3-DX}
    \only<1>
    {
    \centering
    \includegraphics[width=9cm, height=4cm]{img03/P3-DX-Aria-1.png} 
    }
    \only<2>
    {
    \centering
    \includegraphics[width=9cm, height=4cm]{img03/P3-DX-Aria-2.png} 
    }
\end{frame}

\begin{frame}
    \frametitle{Pixhalk}
    \begin{columns}
        \begin{column}{0.3\textwidth}
            \centering
            \includegraphics[width=4cm, height=3cm]{img03/cube_orange.jpg} 
        \end{column}
        \begin{column}{0.7\textwidth}
            \only<1>
            {
            \centering
            \includegraphics[width=6cm, height=4cm]{img03/rover.png} 
            }
            \only<2>
            {
            \centering
            \includegraphics[width=6cm, height=4cm]{img03/drone.png} 
            }
        \end{column}
    \end{columns}
\end{frame}

\begin{frame}
    \frametitle{Pixhalk}
    \only<1>
    {
    \centering
    \includegraphics[width=5cm, height=7.5cm]{img03/pixhalk_3.png} 
    }
    \only<2>
    {
    \centering
    \includegraphics[width=9cm, height=6cm]{img03/pixhalk_2.png} 
    }
\end{frame}

\begin{frame}
    \frametitle{Framework Completo}
    \centering
    \includegraphics[width=9cm, height=6cm]{img03/mavros.png} 
\end{frame}

\begin{frame}
    \frametitle{Mission Planner}
    \centering
    \includegraphics[width=9cm, height=6cm]{img03/missionplanner.png} 
\end{frame}

\begin{frame}
    \frametitle{QGroundControl}
    \centering
    \includegraphics[width=9cm, height=6cm]{img03/qgroundcontrol.png} 
\end{frame}

\begin{frame}
    \frametitle{Outros}
    \centering
    \includegraphics[width=9cm, height=4.5cm]{img03/outros.png} 
\end{frame}

\begin{frame}
    \frametitle{Exemplo de Configuração - Robô Real}
    \centering
    \includegraphics[width=9cm, height=4cm]{img03/exemplo_config_1.png} 
\end{frame}

\begin{frame}
    \frametitle{Software In The Loop}
    \centering
    \includegraphics[width=9cm, height=4.5cm]{img03/exemplo_config_2_sitl.png} 
\end{frame}

\section{Simuladores}
\begin{frame}
    \frametitle{Simuladores}
    \scriptsize
    \begin{itemize}
        \item Simuladores são convenientes para o desenvolvimento de aplicações robóticas sem a logística necessária para a operação de robôs móveis. Permitem testar as aplicações em ambientes de qualquer dimensão e diferentes topologias (diferentes maneiras de representar o ambiente).
        \item Simuladores implementam todo o software embarcado nos robôs reais e simulam a cinemática e dinâmica dos robôs. Podem oferecer visualização 2D ou 3D.
    \end{itemize}
    \only<1>
    {
    \begin{table}[ht]
      \centering
      \begin{tabular}{cc}
        \includegraphics[width=0.5\linewidth]{img03/v-rep.png} &
        \includegraphics[width=0.34\linewidth]{img03/mobilesim.png} \\
        (V-Rep) & (MobileSim) \\
      \end{tabular}
    \end{table}
    }
    \only<2>
    {
    \begin{table}[ht]
      \centering
      \begin{tabular}{cc}
        \includegraphics[width=0.5\linewidth]{img03/gazebo_1.png} &
        \includegraphics[width=0.34\linewidth]{img03/gazebo_2.png} \\
        (Gazebo) & (Gazebo) \\
      \end{tabular}
    \end{table}
    }
    \only<3>
    {
        \centering
        \includegraphics[width=0.35\linewidth]{img03/turtlesim.png} &
    }
\end{frame}




% (AULA 02) - Introdução ao ROS
\section{Bibliografia}


\begin{frame}
	\frametitle{Livros Texto}
	\only<1>
	{
		  \centering
		  \includegraphics[width=4.5cm, height=6cm]{Figuras/RobotProgramming.png}
	}

        \only<2>
	{
		\begin{columns}
			\begin{column}{0.5\textwidth}
				\centering
				\includegraphics[width=4.5cm, height=6cm]{Figuras/ROS_Springer.jpg}
			\end{column} 
			\begin{column}{0.5\textwidth}
				\centering
				\includegraphics[width=4.5cm, height=6cm]{Figuras/Learning.jpg}
			\end{column} 
		\end{columns}
	}	

        \only<3>
	{
		\begin{columns}
			\begin{column}{0.5\textwidth}
				\centering
				\includegraphics[width=4.5cm, height=6cm]{Figuras/LivroPython1.png}
			\end{column} 
			\begin{column}{0.5\textwidth}
				\centering
				\includegraphics[width=4.5cm, height=6cm]{Figuras/LivroPython2.png}
			\end{column} 
		\end{columns}
	}	
\end{frame}

\begin{frame}
	\frametitle{ROS Website (\href{www.ros.org}{www.ros.org})} 
	\only<1>
	{
		\includegraphics[width=9cm, height=6cm]{Figuras/ROS_Website.png} 
	}
	\only<2>
	{
		\includegraphics[width=9cm, height=6cm]{Figuras/ROS_Website_2.png} 
	}
\end{frame}

\section{Motivação}

\begin{frame}
	\frametitle{Ciclo de um Processo em Robótica} 
	\centering

	\begin{tikzpicture}
						[squarednode/.style={rectangle, draw=red!60, fill=red!5, very thick, minimum size = 10mm},]
		\only<1-3>
		{
			\node[squarednode] 									(sense)	{Sensores};
		}
		\only<2-3>
		{	
			\node[squarednode, right of=sense,xshift=2.4cm] 	(think) {Processamento};
		}
		\only<3->
		{	
			\node[squarednode, right of=think,xshift=2.4cm] 	(act)	{Atuação};
		}
		\only<1-3>
		{
			\node[inner sep=0pt, above of=sense, yshift=1cm] 	(fig_sense) 
			{
				\includegraphics[width=.25\textwidth]{Figuras/vision.jpg}
			};
		}
		\only<2-3>
		{		
			\node[inner sep=0pt, above of=think, yshift=1cm] 	(fig_think) 
			{
				\includegraphics[width=.25\textwidth]{Figuras/brain.jpg}
			};
		}
		\only<3->
		{		
			\node[inner sep=0pt, above of=act, yshift=1cm] 		(fig_act) 
			{
				\includegraphics[width=.25\textwidth]{Figuras/act.png}
			};
		}			
		\only<2-3>
		{	
			\draw[->, thick, draw=red!60] (sense.east) -- (think.west) ;
		}
		\only<3->
		{
			\draw[->, thick, draw=red!60] (think.east) -- (act.west) ;
		}
	\end{tikzpicture}
	
\end{frame}

\begin{frame}
	\frametitle{Sensores}
	\centering
	\begin{tikzpicture}
	[vermelho/.style={rectangle, draw=red!60, 	fill=red!5, 	very thick, minimum size = 10mm},
	verde/.style	={rectangle, draw=green!60, fill=green!5, 	very thick, minimum size = 10mm},]
	
	\node[vermelho] 								(sense)		{Sensores};
	\node[verde, right of=sense	 	,xshift=2.4cm] 	(ultrassom) {Ultrassom};
	\node[verde, above of=ultrassom	,yshift=1cm] 	(lasers)	{Laser Scan};
	\node[verde, above of=lasers	,yshift=1cm] 	(camera)	{Câmera};
	\node[verde, below of=ultrassom	,yshift=-1cm] 	(GPS)		{GPS};	
	\node[inner sep=0pt, above of=sense, yshift=1cm] 	(fig_sense) 
	{
		\includegraphics[width=.25\textwidth]{Figuras/vision.jpg}
	};

	\node[inner sep=0pt, right of=camera, xshift=2.4cm] 	(fig_camera) 
	{
		\includegraphics[width=.15\textwidth]{Figuras/camera.jpg}
	};

	\node[inner sep=0pt, right of=lasers, xshift=2.4cm] 	(fig_lasers) 
	{
		\includegraphics[width=.10\textwidth]{Figuras/lidar.jpg}
	};

	\node[inner sep=0pt, right of=ultrassom, xshift=2.4cm] 	(fig_ultrassom) 
	{
		\includegraphics[width=.15\textwidth]{Figuras/ultrassonic.jpg}
	};
	
	\node[inner sep=0pt, right of=GPS, xshift=2.4cm] 	(fig_gps) 
	{
		\includegraphics[width=.10\textwidth]{Figuras/gps.jpg}
	};
	
	\end{tikzpicture}
\end{frame}
%------------------------------------------------


\begin{frame}
	\frametitle{Processamento}
	\centering
	\begin{tikzpicture}
	[vermelho/.style={rectangle, draw=red!60, 	fill=red!5, 	very thick, minimum size = 10mm},
	verde/.style	={rectangle, draw=green!60, fill=green!5, 	very thick, minimum size = 10mm},]
	
	\node[vermelho] 								(think)		{Processamento};
	\node[verde, right of=think	 	,xshift=2.9cm] 	(obi) 		{Otimização Bioinspirada};
	\node[verde, above of=obi		,yshift=1cm] 	(mdl)		{Machine/Deep Learning};
	\node[verde, above of=mdl		,yshift=1cm] 	(ai)		{Inteligência Artificial};
	\node[verde, below of=obi		,yshift=-1cm] 	(signal)	{Processamento de Sinais};	
	\node[inner sep=0pt, above of=think, yshift=1cm] (fig_think) 
	{
		\includegraphics[width=.15\textwidth]{Figuras/brain.jpg}
	};
	
	\node[inner sep=0pt, right of=ai, xshift=2.4cm] 	(fig_ai) 
	{
		\includegraphics[width=.25\textwidth]{Figuras/ai.png}
	};
	
	\node[inner sep=0pt, right of=mdl, xshift=2.4cm] 	(fig_mdl) 
	{
		\includegraphics[width=.15\textwidth]{Figuras/machinelearning.png}
	};
	
	\node[inner sep=0pt, right of=signal, xshift=2.4cm] (fig_signal) 
	{
		\includegraphics[width=.20\textwidth]{Figuras/processamentosinais.png}
	};
	
	\node[inner sep=0pt, right of=obi, xshift=2.4cm] 	(fig_obi) 
	{
		\includegraphics[width=.15\textwidth]{Figuras/gohb.png}
	};
	
	\end{tikzpicture}
\end{frame}
%------------------------------------------------

\begin{frame}
	\frametitle{Processamento}
	{
		\begin{itemize}
			\item Visual Odometry / SLAM / Outros algoritmos de visão
			\item Filtro de Partículas
			\item Algoritmos de Localização
			\item Planejamento de Caminhos e Trajetórias
			\item Controle (PID, Realimentação de estados)
			\item Filtro de Kalman
		\end{itemize}
	}
\end{frame}

\begin{frame}
	\frametitle{Atuadores}
	\centering
	\begin{tikzpicture}
		[vermelho/.style={rectangle, draw=red!60, 	fill=red!5, 	very thick, minimum size = 10mm},
		verde/.style	={rectangle, draw=green!60, fill=green!5, 	very thick, minimum size = 10mm},]
	
		\node[vermelho] 							(act)		{Atuadores};
		\node[verde, right of=act ,xshift=2.9cm] 	(uav) 		{Motores de UAV (Drones)};
		\node[verde, above of=uav ,yshift=1cm] 		(ser)		{Servo Motores};
		\node[verde, above of=ser ,yshift=1cm] 		(auv)		{Motores Aquáticos (AUV)};
		\node[verde, below of=uav ,yshift=-1cm] 	(gas)		{Motores Combustão};	
		\node[inner sep=0pt, above of=think, yshift=1cm] (fig_act) 
		{
			\includegraphics[width=.25\textwidth]{Figuras/act.png}
		};
	
		\node[inner sep=0pt, right of=uav, xshift=2.7cm] 	(fig_uav) 
		{
			\includegraphics[width=.15\textwidth]{Figuras/uav.jpg}
		};
	
		\node[inner sep=0pt, right of=ser, xshift=2.7cm] 	(fig_ser) 
		{
			\includegraphics[width=.15\textwidth]{Figuras/servo.jpg}
		};
	
		\node[inner sep=0pt, right of=auv, xshift=2.7cm] 	(fig_auv) 
		{
			\includegraphics[width=.15\textwidth]{Figuras/auv.jpg}
		};
	
		\node[inner sep=0pt, right of=gas, xshift=2.7cm] 	(fig_gas) 
		{
			\includegraphics[width=.15\textwidth]{Figuras/gas.jpg}
		};
	
	\end{tikzpicture}	
\end{frame}

\begin{frame}
    \frametitle{Motivação para o ROS}
    \begin{columns}
        \only<1-2>
        {
            \begin{column}{0.5\textwidth}
                \centering
                \includegraphics[width=4.3cm, height=6.5cm]{Figuras/Re-Inventing.png}
            \end{column}
        }
        \only<2>
        {
            \begin{column}{0.5\textwidth}
                \href{https://spectrum.ieee.org/automaton/robotics/robotics-software/the-origin-story-of-ros-the-linux-of-robotics}{The Origin Story of ROS, the Linux of Robotics} \\~\\
                \scriptsize
                "The world's most influential robotics software platform"
            \end{column}
        }	
    \end{columns}	
\end{frame}

\begin{frame}
    \frametitle{Motivação para o ROS}
    \centering
    \includegraphics[width=8cm, height=6.5cm]{img02/ciclo_ros.png}
\end{frame}

\section{Objetivos}

\begin{frame}
    \frametitle{Objetivos do Curso}
    \begin{itemize}
        \item Entender o Ecossistema ROS (tópicos, nós, mensagens, serviços)
        \item Desenvolver aplicações para controlar o movimento de um robô
        \item Entender como a posição e orientação são representadas no ROS
        \item Desenvolver programas simples utilizando visão computacional
        \item Utilizar os simuladores (Exemplo: Gazebo)
    \end{itemize}
\end{frame}

\begin{frame}
    \frametitle{\href{https://www.ros.org/}{https://www.ros.org/}}
    \only<1>
    {
        \centering
		\includegraphics[width=10cm, height=6cm]{img02/siteROS1.png}        
    }

    \only<2>
    {
        \centering
        \includegraphics[width=10cm, height=6cm]{img02/siteROS2.png}
    }
\end{frame}

\section{História}
\begin{frame}
	\frametitle{História do ROS}
	\begin{itemize}
		\scriptsize
		\item \href{https://www.linkedin.com/in/eric-berger-806b3b3/}{Eric Berger} e {Keenan Wyrobek} começaram o Doutorado (Ph.D) em Stanford...
		\item Buscaram levantar fundos para o desenvolvimento do projeto o Linux da Robótica.
		\item PR2 (Personal Robotics) - \href{https://www.youtube.com/watch?v=oyHWkQcin7I&feature=youtu.be}{\underline{ver vídeo}}
		\item Atualmente: Ecossistema ROS - Qualquer grupo pode iniciar um repositório de código do ROS ("\textit{federated model}")
	\end{itemize}
			
	\begin{table}
		\begin{tabular}{ | >{\centering\arraybackslash}m{1cm}  >{\centering\arraybackslash}m{7cm} | }  
			\hline
			\includegraphics[width=0.6cm,height=0.6cm]{Figuras/stanford.png} & Stanford Personal Robotics Program (janeiro de 2007)\\
			\hline
			\includegraphics[width=0.6cm,height=0.6cm]{Figuras/willowgarage.jpg} & Laboratório de pesquisa em robótica e incubadora tecnológica (Novembro de 2007) \\
			\hline
			\includegraphics[width=0.6cm,height=0.6cm]{Figuras/openrobotics.png} & Open Source Robotics Foundation (OSRF) ou Open Robotics (Fevereiro de 2013)\\
			\hline
		\end{tabular}
	\end{table}
	
	\centering
	\scriptsize
	"Don’t let anyone crush your crazy" ou "Não deixe ninguém esmagar sua loucura"
\end{frame}


\begin{frame}
	\frametitle{RosCon - ROS Conference (\href{https://roscon.ros.org/2024/} {https://roscon.ros.org/2024/})} 
        \begin{columns}
            \begin{column}{0.5\textwidth}
        	\centering
                \only<1>
                {
    	          \includegraphics[width=4cm,height=6cm]{img02/ROSCon1.png} \\~\\
                }
                \only<2>
                {
    	          \includegraphics[width=6cm,height=3.5cm]{img02/ROSCon2.png} \\~\\
                }
            \end{column}
            \begin{column}{0.5\textwidth}
        	\scriptsize
        	Conjugada com o IROS (ou ICRA):
        	\begin{itemize}
            		\item \href{https://www.iros25.org/}{IEEE/RSJ International Conference on Intelligent Robots and Systems}
            		\item \href{(https://2025.ieee-icra.org/}{International Conference on Robotics and Automation}
            	\end{itemize}
            \end{column}
        \end{columns}
\end{frame}



\section{Contato}

\begin{frame}
    \frametitle{Sensores de Contato}
    \centering
    \begin{columns}
        \begin{column}{0.5\textwidth}
            \centering
            \begin{tabular}{c c c}
                \includegraphics[width=0.3\textwidth]{img05/contato_1.png} &
                \includegraphics[width=0.3\textwidth]{img05/contato_2.png} &
                \includegraphics[width=0.3\textwidth]{img05/contato_3.png} \\
            \end{tabular}
            \includegraphics[width=0.7\textwidth]{img05/contato_4.png} \\            
        \end{column}
        \begin{column}{0.5\textwidth}
            \begin{itemize}
                \scriptsize
                \item Os sensores táteis, adquirem a informação de contato com o ambiente por meio de interação física.
                \item Podem servir a diversos propósitos, tal qual um para-choque ou então um sensor que identifica a uma garra quando parar, ou em mãos robóticas, informando que o toque ocorreu.
                \item Os mais comuns empregam chaves normalmente abertas e fechadas, efeitos piezelétricos e piezoresistivos, capacitivos e assim por diante.
            \end{itemize}
        \end{column}    
    \end{columns}
\end{frame}

\section{Encoder}
\begin{frame}
    \frametitle{Sensor de movimento: Encoder}
    \centering
    \begin{columns}
        \begin{column}{0.5\textwidth}
            \centering
            \begin{tabular}{c c}
                \includegraphics[width=0.4\textwidth]{img05/encoder_1.png} &
                \includegraphics[width=0.4\textwidth]{img05/encoder_2.png} \\
            \end{tabular}
            \includegraphics[width=0.5\textwidth]{img05/lente_fotografica.png} \\            
        \end{column}
        \begin{column}{0.5\textwidth}
            \begin{itemize}
                \scriptsize
                \item É um dispositivo eletromecânico que conta ou reproduz pulsos elétricos a partir do movimento rotacional de seu eixo.
                \item Pode ser definido como um transdutor de posição angular.       
                \item \textbf{Exemplo de utilização de encoders em robótica}: medir a posição ou velocidade de motores utilizados para mover robôs.
                \item Se os valores dos encoders forem integrados (utilize um integrador digital), proporcionarão uma estimação da posição do robô. A posição do robô é chamada de \textbf{hodometria}.
            \end{itemize}
        \end{column}    
    \end{columns}
\end{frame}

\begin{frame}
    \frametitle{Sensor de movimento: Encoder}
    \begin{columns}
        \begin{column}{0.5\textwidth}
            \centering
            \scriptsize
            \textbf{Funcionamento regular}: o aparato conta o número de incrementos/transições (1 para sinal elevado – houve uma transição – e 0 para sinal baixo – ainda não houve uma transição) mas não pode dizer a direção do movimento.
        \end{column}
        \begin{column}{0.5\textwidth}
            \begin{itemize}
                \scriptsize
                \item Os encoders são sensores \textbf{proprioceptivos}, pois medem estados internos do robô (no caso, posição e velocidade).
                \item Dependendo do fabricante, encontra-se de diversas resoluções, desde 64 a mais de 50.000 incrementos (“ticks”, pulsos, contagens) por cada revolução do eixo do motor.
                \item Para maiores resoluções pode-se utilizar interpolação.
            \end{itemize}
        \end{column}
    \end{columns}
    \centering
    \includegraphics[width=0.7\textwidth]{img05/encoder_3.png} \\            
\end{frame}

\begin{frame}
    \frametitle{Sensor de movimento: Encoder}
    \begin{columns}
        \begin{column}{0.5\textwidth}
            \centering
            \scriptsize
            \textbf{Funcionamento em quadratura}: utiliza-se dois sensores deslocados entre si em fase de 90º (quadratura, um elevado e outro baixo). A ordem de qual sensor produz sinal elevado (1) primeiro diz a direção do movimento. Além da direção, a resolução neste modo de funcionamento é 4 vezes maior.
        \end{column}
        \begin{column}{0.5\textwidth}
            \centering
            \includegraphics[width=0.5\textwidth]{img05/encoder_4.png} \\            
        \end{column}
    \end{columns}
    \begin{columns}
        \begin{column}{0.5\textwidth}
            \centering
            \includegraphics[width=0.7\textwidth]{img05/encoder_6.png} \\            
        \end{column}
        \begin{column}{0.5\textwidth}
            \centering
            \includegraphics[width=0.7\textwidth]{img05/encoder_5.png} \\            
        \end{column}
    \end{columns}
\end{frame}

\section{Giros, Acelerômetros, IMUs}
\begin{frame}
    \frametitle{Sensor de orientação: Giroscópio}
    \centering
    \begin{columns}
        \begin{column}{0.5\textwidth}
            \centering
            \begin{tabular}{c}
                \includegraphics[width=0.8\textwidth]{img05/giroscópio.png} \\
            \end{tabular}
        \end{column}
        \begin{column}{0.5\textwidth}
            \begin{itemize}
                \scriptsize
                \item Os \textbf{ópticos} utilizam lasers emitidos em duas direções, uma viajando no sentido horário e outra no sentido anti-horário.
                \item Quando o aparato gira, as fases relativas dos lasers são deslocadas de acordo com a velocidade angular do movimento, uma vez que para um dos lasers o caminho ficou mais longo, e para o outro mais curto (efeito Sagnac).
                \item Mede-se a diferença de fase entre os dois feixes, e essa diferença entre as fases dos feixes é proporcional à velocidade angular do corpo e, consequentemente, ao ângulo do aparato girante.
                \item Ring Laser Gyroscope, RLG, que utiliza o efeito Sagnac
            \end{itemize}
        \end{column}    
    \end{columns}
\end{frame}

\begin{frame}
    \frametitle{Sensor de Inércia: Acelerômetros}
    \begin{columns}
        \begin{column}{0.4\textwidth}
            \centering
            \includegraphics[width=0.7\textwidth]{img05/massa_mola_amortecedor.png}
            \begin{mdframed}[%
                            backgroundcolor=blue!20,   % cor de fundo
                            linecolor=red,               % cor da borda
                            linewidth=1pt,               % espessura da borda
                            roundcorner=4pt,             % cantos arredondados
                            innertopmargin=6pt,          % espaço interno acima
                            innerbottommargin=6pt,       % espaço interno abaixo
                            innerleftmargin=6pt,         % espaço interno à esquerda
                            innerrightmargin=6pt         % espaço interno à direita
                            ]
                \scriptsize
                \begin{itemize}
                    \scriptsize
                    \item Na superfície da Terra, o acelerômetro sempre indicará, pelo menos, 1g ao longo do eixo vertical, e 0g em queda-livre.
                    \item Por isso, para obter a aceleração inercial (correspondente ao movimento do corpo), é necessário subtrair a gravidade.
                \end{itemize}
            \end{mdframed}
        \end{column}
        \begin{column}{0.6\textwidth}
            \centering
            \begin{itemize}
                \scriptsize
                \item Conforme os princípios físicos, para alterar movimentos de corpos é necessário imprimir forças sobre eles.
                \item Os acelerômetros são sensores que medem todas as forças que estejam atuando sobre eles, incluindo a gravidade.
                \item Os acelerômetros atuam como sistemas massa-mola-amortecedor, equacionado como:
            \end{itemize}
            \begin{equation*}
                F_{\text{aplicada}} = F_{\text{inercial}} + F_{\text{amortecida}} + F_{\text{elástica}} 
            \end{equation*}
            \begin{equation*}
                F_{\text{aplicada}} = m\ddot{x} + c\dot{x} + kx
            \end{equation*}
            \begin{itemize}
                \scriptsize
                \item[] Onde:
                \item \textbf{m} é a massa de prova.
                \item \textbf{c} é o coeficiente de amortecimento.
                \item \textbf{k} é a constante da mola.
            \end{itemize}
        \end{column}
    \end{columns}
\end{frame}

\begin{frame}
    \frametitle{Unidades de Medidas Inerciais (IMU)}
    \centering
    \begin{columns}
        \begin{column}{0.5\textwidth}
            \centering
            \begin{tabular}{c}
                \includegraphics[width=0.9\textwidth]{img05/IMU.png} \\
            \end{tabular}
        \end{column}
        \begin{column}{0.5\textwidth}
            \begin{itemize}
                \scriptsize
                \item As \textbf{Inertial Measurement Units (IMUs)} são dispositivos que medem posições relativas (x, y, z), orientações (roll, pitch, yaw), velocidades e acelerações de um corpo em movimento.
                \item Combinam as funções dos acelerômetros, giroscópios, bússolas além de possuírem integradores para o cálculo de hodometria.
                \item São unidades inerciais completas, sensíveis a erros, corrigindo a hodometria simples (dead reckoning) provida pelos encoders.
            \end{itemize}
        \end{column}    
    \end{columns}
\end{frame}

\section{Câmeras}
\begin{frame}
    \frametitle{Câmeras}
    \centering
    \begin{columns}
        \begin{column}{0.5\textwidth}
            \centering
            \begin{tabular}{c}
                \includegraphics[width=0.7\textwidth]{img05/câmera_2.png} \\
                \includegraphics[width=0.7\textwidth]{img05/câmera_1.png} \\
            \end{tabular}
        \end{column}
        \begin{column}{0.5\textwidth}
            \begin{itemize}
                \scriptsize
                \item As câmeras são sensores bastante populares, que podem ser utilizados das mais variadas formas.
                \item A área de pesquisa em imagens é denominada de visão computacional, em que a visão robótica é uma das áreas.
                \item Por meio de técnicas de manipulação de imagens, é possível obter distâncias, características (features), identificar objetos e pessoas, obter caminhos, dentre muitas outras aplicações.
            \end{itemize}
        \end{column}    
    \end{columns}
\end{frame}

\begin{frame}
    \frametitle{Câmeras - Knectic}
    \centering
    \includegraphics[width=0.95\textwidth]{img05/kinect.png}
\end{frame}

\begin{frame}{O que é a Câmera Kinect?}
    \begin{itemize}
        \item Sensor desenvolvido originalmente pela Microsoft para o Xbox.
        \item Usa visão computacional para captar profundidade, cor e movimento.
        \item Combina:
        \begin{itemize}
            \item Câmera RGB (imagem tradicional)
            \item Sensor de profundidade (depth camera)
            \item Microfones e acelerômetros (em versões completas)
        \end{itemize}
    \end{itemize}
    \end{frame}

\begin{frame}{Câmera Kinect: Como Funciona?}
    \begin{itemize}
        \item A câmera de profundidade projeta um padrão infravermelho (IR).
        \item O padrão refletido é captado por um sensor e processado para estimar a distância de cada ponto.
        \item Resulta em uma \textbf{imagem de profundidade} (depth map), com cada pixel representando uma distância.
    \end{itemize}
    
    \vspace{0.5cm}
    \textbf{Alternativamente (em modelos mais novos):} usa tecnologia de \textit{Time-of-Flight (ToF)} para medir o tempo que a luz leva para voltar ao sensor.
\end{frame}

\begin{frame}{Câmera Kinect: Vantagens}
    \begin{itemize}
        \item Captura simultânea de imagem colorida e profundidade.
        \item Excelente para rastreamento de corpo humano e gestos.
        \item Pronto para uso com SDKs acessíveis.
        \item Operação em tempo real com boa precisão.
    \end{itemize}
\end{frame}

\begin{frame}{Câmera Kinect: Aplicações}
    \begin{itemize}
        \item Jogos e interfaces baseadas em gestos.
        \item Robótica (percepção 3D e navegação).
        \item Mapeamento de ambientes (SLAM).
        \item Realidade aumentada e captura de movimento.
        \item Reconhecimento de pessoas e gestos em segurança.
    \end{itemize}
\end{frame}

\begin{frame}{O que são Câmeras Baseadas em Eventos?}
    \begin{itemize}
        \item Diferente das câmeras tradicionais, que capturam imagens em quadros fixos.
        \item Cada pixel funciona de forma independente e assíncrona.
        \item Gera eventos apenas quando há variação significativa de luminosidade.
    \end{itemize}
    
    \vspace{0.5cm}
    \textbf{Cada evento contém:}
    \begin{itemize}
        \item Posição do pixel $(x, y)$
        \item Tempo do evento (alta precisão temporal)
        \item Polaridade da variação (aumento ou redução de brilho)
    \end{itemize}
    \end{frame}

\begin{frame}{Câmera Baseada em Eventos: Vantagens}
    \begin{itemize}
        \item Altíssima resolução temporal (microsegundos)
        \item Baixa latência e baixo consumo de energia
        \item Alta faixa dinâmica (HDR) – ideal para ambientes com iluminação extrema
        \item Excelente desempenho com objetos rápidos
    \end{itemize}
\end{frame}

\begin{frame}{Câmera Baseada em Eventos: Aplicações}
    \begin{itemize}
        \item Robótica e drones autônomos
        \item Carros autônomos
        \item Sistemas biomiméticos (inspirados na retina)
        \item Visão computacional de alta velocidade
    \end{itemize}
    \href{https://youtu.be/0wGBpgIrd9M?si=j-tcaOsuDFyUieXV}{Link: Davide Scaramuzza }
\end{frame}

\section{GPS}
\begin{frame}
    \frametitle{Global Positioning System (GPS)}
    \centering
    \begin{columns}
        \begin{column}{0.4\textwidth}
            \centering
            \begin{tabular}{c}
                \includegraphics[width=0.8\textwidth]{img05/gps.png}\\
            \end{tabular}
        \end{column}
        \begin{column}{0.6\textwidth}
            \begin{itemize}
                \scriptsize
                \item O Sistema de Posicionamento Global é um sistema de localização global baseado em satélites e trilateração.
                \item Desenvolvido originalmente pelos USA nas décadas de 1950 e 1960 para estratégias exclusivamente militares, foi liberado  uso pela população civil na metade da década de 1990.
                \item Outros países desenvolveram seus próprios sistemas baseados em satélites, como o COMPASS (ou, BeiDou) da China, operacional a partir da década de 2000, o GLONASS russo, contemporâneo ao GPS, e o GALILEO da União Europeia e o NAVIC da Índia.
                \item O projeto original do GPS contava com 24 satélites em órbitas de 12 horas a uma altura de, aproximadamente, 21000 km acima da superfície terrestre, já no meio interplanetário.
            \end{itemize}
        \end{column}    
    \end{columns}
\end{frame}

\begin{frame}{Sistemas de Posicionamento e Segurança Cibernética}
    \textbf{Contexto:}
    \begin{itemize}
        \item O GPS foi desenvolvido pelos EUA com objetivos militares e só posteriormente liberado ao uso civil.
        \item A dependência exclusiva de um sistema estrangeiro representa um risco estratégico e cibernético.
        \item Outros países criaram seus próprios sistemas (GLONASS - Rússia, BeiDou - China, GALILEO - UE, NAVIC - Índia) para garantir autonomia.
    \end{itemize}
    
    \vspace{0.5cm}
    \textbf{Riscos associados à dependência do GPS:}
    \begin{itemize}
        \item Sinal pode ser bloqueado, degradado ou falsificado (\textit{spoofing}) em tempos de crise.
        \item Impacto direto em setores críticos: telecomunicações, transporte, energia e defesa.
        \item Falta de controle sobre atualizações, segurança e integridade dos dados recebidos.
    \end{itemize}
\end{frame}

\begin{frame}{Independência Tecnológica e Geopolítica}
    \begin{itemize}
        \item O domínio de sistemas de navegação por satélite está ligado à soberania nacional e geopolítica.
        \item Ter um sistema próprio ou acesso a múltiplos sistemas reduz vulnerabilidades.
        \item Integração com estratégias de defesa cibernética e proteção de infraestrutura crítica.
        \item Investimentos em sistemas nacionais impulsionam inovação e capacitação tecnológica interna.
    \end{itemize}
    
    \vspace{0.3cm}
    \textbf{Conclusão:} Em um mundo cada vez mais digital e conectado, o controle autônomo sobre sistemas espaciais é um ativo estratégico fundamental.
\end{frame}

\begin{frame}
    \frametitle{Global Positioning System (GPS)}
    \centering
    \begin{columns}
        \begin{column}{0.4\textwidth}
            \centering
            \begin{tabular}{c}
                \includegraphics[width=0.8\textwidth]{img05/satelite_gps.png}\\
            \end{tabular}
        \end{column}
        \begin{column}{0.6\textwidth}
            \begin{itemize}
                \scriptsize
                \item Os satélites de GPS se chamam NAVSTAR (NAVigation Satellite with Time And Ranging).
                \item Cada satélite possui um relógio atômico para sincronia de posicionamento, corrige o efeito Doppler na transmissão de seus dados por meio da Teoria da Relatividade.
                \item As órbitas desses satélites são tais que, a todo momento no horizonte de um usuário, existam, pelo menos, quatro satélites disponíveis.
            \end{itemize}
        \end{column}    
    \end{columns}
\end{frame}

\begin{frame}
    \frametitle{Global Positioning System (GPS)}
    \begin{itemize}
        \scriptsize
        \item A localização do GPS é feita por meio do princípio matemático da trilateração. No caso, trilateração em 3D.
        \item Existem estações GPS bem estabelecidas por todo o planeta.
    	\item A trilateração em 2D é facilmente explicada da seguinte maneira: 
    \end{itemize}
    \only<1>
    {
        \begin{itemize}
            \scriptsize
            \item[] Uma fonte (ex: satélite) lhe revela que está a uma distância d1 de uma estação conhecida (ex: numa cidade) denominada A.
        \end{itemize}
        \centering                
        \includegraphics[width=0.8\textwidth]{img05/gps_1.png}\\
    }
    \only<2>
    {
        \begin{itemize}
            \scriptsize
            \item[] Uma segunda fonte (outro satélite) informa que o usuário está a uma distância d2 de um outro ponto conhecido B.
        \end{itemize}
        \centering                
        \includegraphics[width=0.9\textwidth]{img05/gps_2.png}\\
    }
    \only<3>
    {
        \begin{itemize}
            \scriptsize
            \item[] Para eliminar a ambiguidade, é necessário obter a informação de uma terceira fonte (satélite), de uma distância d3 de C.
        \end{itemize}
        \centering                
        \includegraphics[width=0.9\textwidth]{img05/gps_3.png}\\
    }
    \only<4>
    {
        \centering                
        \includegraphics[width=0.95\textwidth]{img05/gps_4.png}\\
    }
\end{frame}

\section{Arquiteturas}

\begin{frame}
    \frametitle{P3-DX}
    \only<1>
    {
    \centering
    \includegraphics[width=9cm, height=4cm]{img03/P3-DX-Aria-1.png} 
    }
    \only<2>
    {
    \centering
    \includegraphics[width=9cm, height=4cm]{img03/P3-DX-Aria-2.png} 
    }
\end{frame}

\begin{frame}
    \frametitle{Pixhalk}
    \begin{columns}
        \begin{column}{0.3\textwidth}
            \centering
            \includegraphics[width=4cm, height=3cm]{img03/cube_orange.jpg} 
        \end{column}
        \begin{column}{0.7\textwidth}
            \only<1>
            {
            \centering
            \includegraphics[width=6cm, height=4cm]{img03/rover.png} 
            }
            \only<2>
            {
            \centering
            \includegraphics[width=6cm, height=4cm]{img03/drone.png} 
            }
        \end{column}
    \end{columns}
\end{frame}

\begin{frame}
    \frametitle{Pixhalk}
    \only<1>
    {
    \centering
    \includegraphics[width=5cm, height=7.5cm]{img03/pixhalk_3.png} 
    }
    \only<2>
    {
    \centering
    \includegraphics[width=9cm, height=6cm]{img03/pixhalk_2.png} 
    }
\end{frame}

\begin{frame}
    \frametitle{Framework Completo}
    \centering
    \includegraphics[width=9cm, height=6cm]{img03/mavros.png} 
\end{frame}

\begin{frame}
    \frametitle{Mission Planner}
    \centering
    \includegraphics[width=9cm, height=6cm]{img03/missionplanner.png} 
\end{frame}

\begin{frame}
    \frametitle{QGroundControl}
    \centering
    \includegraphics[width=9cm, height=6cm]{img03/qgroundcontrol.png} 
\end{frame}

\begin{frame}
    \frametitle{Outros}
    \centering
    \includegraphics[width=9cm, height=4.5cm]{img03/outros.png} 
\end{frame}

\begin{frame}
    \frametitle{Exemplo de Configuração - Robô Real}
    \centering
    \includegraphics[width=9cm, height=4cm]{img03/exemplo_config_1.png} 
\end{frame}

\begin{frame}
    \frametitle{Software In The Loop}
    \centering
    \includegraphics[width=9cm, height=4.5cm]{img03/exemplo_config_2_sitl.png} 
\end{frame}

\section{Simuladores}
\begin{frame}
    \frametitle{Simuladores}
    \scriptsize
    \begin{itemize}
        \item Simuladores são convenientes para o desenvolvimento de aplicações robóticas sem a logística necessária para a operação de robôs móveis. Permitem testar as aplicações em ambientes de qualquer dimensão e diferentes topologias (diferentes maneiras de representar o ambiente).
        \item Simuladores implementam todo o software embarcado nos robôs reais e simulam a cinemática e dinâmica dos robôs. Podem oferecer visualização 2D ou 3D.
    \end{itemize}
    \only<1>
    {
    \begin{table}[ht]
      \centering
      \begin{tabular}{cc}
        \includegraphics[width=0.5\linewidth]{img03/v-rep.png} &
        \includegraphics[width=0.34\linewidth]{img03/mobilesim.png} \\
        (V-Rep) & (MobileSim) \\
      \end{tabular}
    \end{table}
    }
    \only<2>
    {
    \begin{table}[ht]
      \centering
      \begin{tabular}{cc}
        \includegraphics[width=0.5\linewidth]{img03/gazebo_1.png} &
        \includegraphics[width=0.34\linewidth]{img03/gazebo_2.png} \\
        (Gazebo) & (Gazebo) \\
      \end{tabular}
    \end{table}
    }
    \only<3>
    {
        \centering
        \includegraphics[width=0.35\linewidth]{img03/turtlesim.png} &
    }
\end{frame}



\section{Outras Formas Locomoção}

\begin{frame}
    \frametitle{Outras Formas de Locomoção}
    \centering
    \begin{tabular}{cc}
        \includegraphics[width=0.25\textwidth]{img04/balao.png} &
        \includegraphics[width=0.35\textwidth]{img04/shark.png} \\
        \includegraphics[width=0.30\textwidth]{img04/fish.png} &
        \includegraphics[width=0.55\textwidth]{img04/planar.png} \\    
    \end{tabular}
\end{frame}

\begin{frame}
    \frametitle{Manobrabilidade}
    \begin{columns}
        \begin{column}{0.5\textwidth}
            \centering
            \includegraphics[width=0.9\textwidth]{img04/mecanum_wheels.jpeg}
        \end{column}
        \begin{column}{0.5\textwidth}
            \begin{itemize}
                \scriptsize
                \item \textbf{Manobrabilidade} refere-se a facilidade do robô em executar movimentos (manobras) ao longo do ambiente.
                \item Alguns robôs são \textbf{holonômicos / omnidirecionais}, o que significa que podem se mover a qualquer momento e para qualquer direção, independentemente da sua orientação.
                \item Isso requer uma configuração de rodas \textbf{omnidirecionais}.
            \end{itemize}
        \end{column}
    \end{columns}
\end{frame}

\begin{frame}
    \frametitle{Manobrabilidade}
    \begin{columns}
        \begin{column}{0.5\textwidth}
            \centering
            \includegraphics[width=0.9\textwidth]{img04/ackermann.png}
        \end{column}
        \begin{column}{0.5\textwidth}
            \begin{itemize}
                \scriptsize
                \item Em contraste, considere a configuração conhecida como modelo de \textbf{tração Ackermann} (exemplo: os carros de passeio).
                \item Essa configuração não permite que o robô faça movimentos omnidirecionais. 
                \item Ao contrário, obriga o veículo a se deslocar sobre círculos com raios produzidos pelo esterçamento das rodas, inclusive, círculos estes de raios maiores que o próprio veículo. 
                \item O carro sempre se movimenta sobre círculos, e \textbf{nunca caminha “de lado”, ou seja, sobre o eixo-y}.
            \end{itemize}
        \end{column}
    \end{columns}
\end{frame}


\begin{frame}
    \frametitle{Manobrabilidade}
    \begin{columns}
        \begin{column}{0.5\textwidth}
            \centering
            \begin{mdframed}[%
                            backgroundcolor=blue!20,   % cor de fundo
                            linecolor=red,               % cor da borda
                            linewidth=1pt,               % espessura da borda
                            roundcorner=4pt,             % cantos arredondados
                            innertopmargin=6pt,          % espaço interno acima
                            innerbottommargin=6pt,       % espaço interno abaixo
                            innerleftmargin=6pt,         % espaço interno à esquerda
                            innerrightmargin=6pt         % espaço interno à direita
                            ]
                \scriptsize
                Quanto maior a manobrabilidade de um robô, geralmente, maior é a dificuldade de processamento para gerar sinais de controle para levar o robô às referências desejadas.
            \end{mdframed}
        \end{column}
        \begin{column}{0.5\textwidth}
            \begin{itemize}
                \scriptsize
                \item Controlabilidade refere-se a facilidade em se gerar sinais de controle para um robô de modo a leva-lo a algum resultado.
                \item Há, em geral, uma correlação \textbf{inversa entre a manobrabilidade e a controlabilidade}.
                \item Por exemplo, robôs omnidirecionais com quatro rodas suécas requerem uma grande quantidade de cálculos para transformar o movimento desejado em comandos individuais para cada uma das rodas.
                \item Robôs simples, como o modelo diferencial (P3DX) requerem menor quantidade de cálculos para a obtenção dos comandos individuais de cada roda.
            \end{itemize}
        \end{column}
    \end{columns}
\end{frame}



\section{ROS - First Steps}

\begin{frame}
	\frametitle{O que é o ROS?}

	\only<1>
	{
		\begin{itemize}
			\item ROS é um \textit{framework open source} para o desenvolvimento de software para robôs
			\item Provê uma funcionalidade análoga a de um sistema operacional
		\end{itemize}
	}

	\only<2>
	{
		\begin{itemize}
			\item Provês serviços de sistema operacional
			\item Abstração de hardware
			\item Controle de dispositivos em baixo nível
			\item Implementação de funcionalidades comumente utilizadas 
			\item Transferência de mensagens entre processos
			\item Gerenciamento de pacotes
		\end{itemize}
	}
	\only<3>
	{
		\centering
		\begin{tikzpicture}
			[vermelho/.style={rectangle, draw=red!90, 	fill=red!25, 	very thick, minimum size = 10mm},
			verde/.style	={rectangle, draw=green!90, fill=green!25, 	very thick, minimum size = 10mm},
			azul/.style		={rectangle, draw=blue!90, 	fill=blue!25, 	very thick, minimum size = 10mm},]			
			\node[vermelho] 						(aps)	{USER APPS};
			\node[verde,below of=aps ,yshift=-1cm]	(ros) 	{ROS};
			\node[azul, below of=ros ,yshift=-1cm] 	(ser)	{Hardware (Sensores, Atuadores, Robôs)};
		\end{tikzpicture}			
	}
\end{frame}

\begin{frame}
	\frametitle{Ecossistema ROS} 
	\only<1>
	{
		\centering
		\includegraphics[width=9cm,height=6cm]{Figuras/RosEcosystem.png} \\~\\
		\footnote{\scriptsize ROS Robot Programming (in English).Authors:Yoonseok Pyo, Hancheol Cho, Leon Jung, Darby Lim}
	}
	\only<2>
	{
		\centering
		\includegraphics[width=11cm,height=6cm]{Figuras/universoROS.png} 
	}	
\end{frame}

\begin{frame}
	\frametitle{Por que usar ROS?}
	\begin{itemize}
		\item Re-utilização de códigos em Robótica (P\&D)
		\item Ambiente de desenvolvimento pronto para utilizar
		\item Contínuo suporte 
		\item Ferramentas amigáveis
		\item Concebido para ser escalável
		\item Comunidade ativa ao redor do mundo. \scriptsize Clique em: \href{http://answers.ros.org}{\underline{http://answers.ros.org}}
	\end{itemize}
	\centering
	\includegraphics[width=2cm,height=1cm]{Figuras/ROSAnswers.png} 
\end{frame}

\begin{frame}
	\frametitle{Configurando o Ambiente de Trabalho - Sistemas Operacionais}
	\centering
	\includegraphics[width=7.5cm,height=6cm]{Figuras/SistemasOperacionais.png}
	\begin{itemize}
		\item Pode ser utilizado em conjunto com a maioria dos sistemas operacionais
		\item É recomendado o Ubuntu como sistema operacional mais estável
	\end{itemize}

\end{frame}

\begin{frame}
	\frametitle{Ambiente de Desenvolvimento}
	\begin{itemize}
		\item \underline{\textbf{Hardware}}: \textit{Desktop} ou \textit{laptop} usando processador Intel ou AMD
		\item \underline{\textbf{Sistema Operacional}}: Ubuntu (versão compatível com a versão do ROS - verificar site oficial
		\item \underline{\textbf{ROS}}: Noetic Ninjemys
	\end{itemize}
\end{frame}

\begin{frame}
	\frametitle{Máquinas Virtuais}
	\begin{table}
		\begin{tabular}{  >{\centering\arraybackslash}m{1cm}  >{\centering\arraybackslash}m{7cm}  }  
			\includegraphics[width=1cm,height=1cm]{Figuras/VirtualBox.png} & Virtual Box (\href{www.virtualbox.org}{\underline{www.virtualbox.org}})\\
			\includegraphics[width=1.4cm,height=1cm]{Figuras/Parallels.png} & Parallels (\href{www.parallels.com}{\underline{www.parallels.com}}) \\
			\includegraphics[width=1cm,height=1cm]{Figuras/VMWare.png} & VMWare (\href{www.vmware.com}{\underline{www.vmware.com}})\\
		\end{tabular}
	\end{table}
	
	\centering
	\scriptsize
	\color{red}
	Diversas imagens de instalação do ROS disponíveis on-line
\end{frame}


\begin{frame}
    \frametitle{Preparando o Ubunto antes de Instalar o ROS}
	\only<1>
		{
		\frametitle{Instalação Geral - NTP (Network Time Protocol}
		\textit{\underline{Objetivo}}: Reduzir diferença de tempo entre os pacotes ROS na comunicação entre múltiplos computadores, ou seja, melhorar a sincronia entre os processos.
		\vspace{0.5cm}
		\begin{figure}
			\includegraphics[width=11cm,height=1.7cm]{Figuras/Ntp.png}
		\end{figure}
	}

	\only<2>
	{
		\frametitle{Instalação Geral - Adding Source List}
		\textit{\underline{Objetivo}}: Adicionar o endereço do repositório do ROS no \textit{ros.latest.list}. Abra um novo terminal e digite o seguinte comando.
		\vspace{0.5cm}
		\begin{figure}
			\includegraphics[width=11cm,height=1.7cm]{Figuras/SourceList.png}
		\end{figure}
	}

	\only<3>
	{
		\frametitle{Instalação Geral - Definindo Chave Pública}
		\textit{\underline{Objetivo}}: Define uma chave pública para acessar o repositório do ROS para o download de pacotes. Acessar a página oficial (Wiki) do ROS para atualizar a chave que pode ser alterada.
		\vspace{0.5cm}
		\begin{figure}
			\includegraphics[width=11cm,height=1.7cm]{Figuras/Chave.png}
		\end{figure}
	}

	\only<4>
	{
		\frametitle{Instalação Geral - Adding Atualizando o Sistema}
		\textit{\underline{Objetivo}}: Após a adição do endereço do repositório do ROS na lista de fontes, deve ser realizada uma nova indexação da lista de pacotes. Embora não seja mandatório, é recomendável a realização de um upgrade em todos os pacotes previamente instalados no UBUNTU antes da instalação do ROS.
		\vspace{0.5cm}
		\begin{figure}
			\includegraphics[width=11cm,height=1cm]{Figuras/Update.png}
		\end{figure}
	}

	\only<5>
	{
		\frametitle{Instalação Geral - Instalando o ROS}
		\textit{\underline{Objetivo}}: Finalmente, devemos instalar os pacotes do ROS para desktop usando o seguinte comando. Esta instalação incluirá as principais funcionalidades do ROS (ROS, rqt, RViz, robot-related-libraries, simulation, navigation, etc)
		\vspace{0.5cm}
		\begin{figure}
			\includegraphics[width=11cm,height=1cm]{Figuras/Install.png}
		\end{figure}
	}
\end{frame}

\begin{frame}
	\frametitle{Conceitos Básicos: Nó (\textit{Node})}
	\begin{itemize}
		\item O princípio básico de operação do ROS é a facilidade de modularização de processos que são executados em conjunto pelo sistema operacional
		\item Nó (ou Node): um processo que usa o framework do ROS
		\item Os nós podem ser executados em máquinas diferentes de forma transparente
		\item O \textbf{roscore} é um processo único (ou nó principal e obrigatório) que gerencia a comunicação entre todos os nós
	\end{itemize}
\end{frame}

\begin{frame}
	\frametitle{Conceitos Básicos: \textit{roscore}}
	\begin{itemize}
		\item Funciona basicamente como um name server
		\item Os nós se comunicam com o ROSCORE o qual é definido por uma variável de ambiente denominada ROS\_MASTER\_URI
	\end{itemize}
	\centering
	\includegraphics[width=8cm,height=3cm]{Figuras/roscore.png}
\end{frame}

\begin{frame}
	\frametitle{Conceitos Básicos: Tópicos}
	\begin{itemize}
		\item Tópico é o meio através do qual uma mensagem pode ser enviada de um nó para outro (ou vários outros) 
		\item A forma de comunicação segue um modelo de \textit{publisher-subscriber}
		\item \underline{\textit{Publish}}: Envia uma mensagem para um tópico
		\item \underline{\textit{Subscribe}}: São chamados caso uma mensagem é publicada
		\item As mensagens publicadas são transmitidas para todos os \textit{subscribers} 
	\end{itemize}
	\centering
	\centering
	\includegraphics[width=8.5cm,height=2.3cm]{Figuras/topicos.png}	
\end{frame}

\begin{frame}
	\frametitle{Conceitos Básicos: Serviços}
	\begin{itemize}
		\item \textit{SERVICE} é um mecanismo pelo qual um nó envia uma requisição para outro e recebe uma resposta à chamada
		\item A forma de comunicação segue um modelo \textit{request-response}
		\item Um serviço é chamado com uma estrutura de \textit{request} e, em resposta, uma estrutura de resposta é retornada
	\end{itemize}
	\centering
	\centering
	\includegraphics[width=8.5cm,height=2.3cm]{Figuras/servico.png}	
\end{frame}

\begin{frame}
	\frametitle{Conceitos Básicos: \textit{Action}}
	\begin{itemize}
		\item \textit{ACTION} é usada quando uma requisição leva longo tempo para ser executada e é necessário que a sua evolução deve ser acompanhada.
		\item É muito similar ao serviço, onde \textit{'goals'} e \textit{'results'} são análogos aos \textit{'requests'} e \textit{'responses'} do \textit{SERVICE}.
		\item Diferentemente do \textit{SERVICE}, uma \textit{ACTION} é frequentemente usado quando é necessário enviar tarefas mais complexas para o robô como, por exemplo, cancelar um objetivo previamente enviado enquanto as tarefas estão em andamento.
	\end{itemize}
	\centering
	\centering
	\includegraphics[width=6.5cm,height=2.5cm]{Figuras/Action.png}	
\end{frame}

\begin{frame}
	\frametitle{Tópicos, Serviços e Actions}
	\begin{table}
		\scriptsize
		\begin{tabular}{c c c p{4cm}}
			\hline
			\hline
			\textbf{Tipo} & \textbf{Característica} & & \textbf{Descrição} \\		
			\hline		
			Tópico			& Assíncrona	& Unidirecional & Usado quando os dados são trocados continuamente. \\
			Serviço			& Síncrona 		& Bidirecional  & Usado quando o cliente requisita e recebe um estado corrente. \\
			\textit{Action}	& Assíncrona	& Bidirecional  & Usado quando é difícil usar o serviço devido a longos tempos de resposta após a solicitação ou quando é necessário um valor de \textit{feedback} intermediário. \\
			\hline
			\hline
		\end{tabular}
	\end{table}
\end{frame}

% (AULA 03) - Introdução à Robótica Móvel
%\section{Bibliografia}


\begin{frame}
	\frametitle{Livros Texto}
	\only<1>
	{
		  \centering
		  \includegraphics[width=4.5cm, height=6cm]{Figuras/RobotProgramming.png}
	}

        \only<2>
	{
		\begin{columns}
			\begin{column}{0.5\textwidth}
				\centering
				\includegraphics[width=4.5cm, height=6cm]{Figuras/ROS_Springer.jpg}
			\end{column} 
			\begin{column}{0.5\textwidth}
				\centering
				\includegraphics[width=4.5cm, height=6cm]{Figuras/Learning.jpg}
			\end{column} 
		\end{columns}
	}	

        \only<3>
	{
		\begin{columns}
			\begin{column}{0.5\textwidth}
				\centering
				\includegraphics[width=4.5cm, height=6cm]{Figuras/LivroPython1.png}
			\end{column} 
			\begin{column}{0.5\textwidth}
				\centering
				\includegraphics[width=4.5cm, height=6cm]{Figuras/LivroPython2.png}
			\end{column} 
		\end{columns}
	}	
\end{frame}

\begin{frame}
	\frametitle{ROS Website (\href{www.ros.org}{www.ros.org})} 
	\only<1>
	{
		\includegraphics[width=9cm, height=6cm]{Figuras/ROS_Website.png} 
	}
	\only<2>
	{
		\includegraphics[width=9cm, height=6cm]{Figuras/ROS_Website_2.png} 
	}
\end{frame}

\section{Motivação}

\begin{frame}
	\frametitle{Ciclo de um Processo em Robótica} 
	\centering

	\begin{tikzpicture}
						[squarednode/.style={rectangle, draw=red!60, fill=red!5, very thick, minimum size = 10mm},]
		\only<1-3>
		{
			\node[squarednode] 									(sense)	{Sensores};
		}
		\only<2-3>
		{	
			\node[squarednode, right of=sense,xshift=2.4cm] 	(think) {Processamento};
		}
		\only<3->
		{	
			\node[squarednode, right of=think,xshift=2.4cm] 	(act)	{Atuação};
		}
		\only<1-3>
		{
			\node[inner sep=0pt, above of=sense, yshift=1cm] 	(fig_sense) 
			{
				\includegraphics[width=.25\textwidth]{Figuras/vision.jpg}
			};
		}
		\only<2-3>
		{		
			\node[inner sep=0pt, above of=think, yshift=1cm] 	(fig_think) 
			{
				\includegraphics[width=.25\textwidth]{Figuras/brain.jpg}
			};
		}
		\only<3->
		{		
			\node[inner sep=0pt, above of=act, yshift=1cm] 		(fig_act) 
			{
				\includegraphics[width=.25\textwidth]{Figuras/act.png}
			};
		}			
		\only<2-3>
		{	
			\draw[->, thick, draw=red!60] (sense.east) -- (think.west) ;
		}
		\only<3->
		{
			\draw[->, thick, draw=red!60] (think.east) -- (act.west) ;
		}
	\end{tikzpicture}
	
\end{frame}

\begin{frame}
	\frametitle{Sensores}
	\centering
	\begin{tikzpicture}
	[vermelho/.style={rectangle, draw=red!60, 	fill=red!5, 	very thick, minimum size = 10mm},
	verde/.style	={rectangle, draw=green!60, fill=green!5, 	very thick, minimum size = 10mm},]
	
	\node[vermelho] 								(sense)		{Sensores};
	\node[verde, right of=sense	 	,xshift=2.4cm] 	(ultrassom) {Ultrassom};
	\node[verde, above of=ultrassom	,yshift=1cm] 	(lasers)	{Laser Scan};
	\node[verde, above of=lasers	,yshift=1cm] 	(camera)	{Câmera};
	\node[verde, below of=ultrassom	,yshift=-1cm] 	(GPS)		{GPS};	
	\node[inner sep=0pt, above of=sense, yshift=1cm] 	(fig_sense) 
	{
		\includegraphics[width=.25\textwidth]{Figuras/vision.jpg}
	};

	\node[inner sep=0pt, right of=camera, xshift=2.4cm] 	(fig_camera) 
	{
		\includegraphics[width=.15\textwidth]{Figuras/camera.jpg}
	};

	\node[inner sep=0pt, right of=lasers, xshift=2.4cm] 	(fig_lasers) 
	{
		\includegraphics[width=.10\textwidth]{Figuras/lidar.jpg}
	};

	\node[inner sep=0pt, right of=ultrassom, xshift=2.4cm] 	(fig_ultrassom) 
	{
		\includegraphics[width=.15\textwidth]{Figuras/ultrassonic.jpg}
	};
	
	\node[inner sep=0pt, right of=GPS, xshift=2.4cm] 	(fig_gps) 
	{
		\includegraphics[width=.10\textwidth]{Figuras/gps.jpg}
	};
	
	\end{tikzpicture}
\end{frame}
%------------------------------------------------


\begin{frame}
	\frametitle{Processamento}
	\centering
	\begin{tikzpicture}
	[vermelho/.style={rectangle, draw=red!60, 	fill=red!5, 	very thick, minimum size = 10mm},
	verde/.style	={rectangle, draw=green!60, fill=green!5, 	very thick, minimum size = 10mm},]
	
	\node[vermelho] 								(think)		{Processamento};
	\node[verde, right of=think	 	,xshift=2.9cm] 	(obi) 		{Otimização Bioinspirada};
	\node[verde, above of=obi		,yshift=1cm] 	(mdl)		{Machine/Deep Learning};
	\node[verde, above of=mdl		,yshift=1cm] 	(ai)		{Inteligência Artificial};
	\node[verde, below of=obi		,yshift=-1cm] 	(signal)	{Processamento de Sinais};	
	\node[inner sep=0pt, above of=think, yshift=1cm] (fig_think) 
	{
		\includegraphics[width=.15\textwidth]{Figuras/brain.jpg}
	};
	
	\node[inner sep=0pt, right of=ai, xshift=2.4cm] 	(fig_ai) 
	{
		\includegraphics[width=.25\textwidth]{Figuras/ai.png}
	};
	
	\node[inner sep=0pt, right of=mdl, xshift=2.4cm] 	(fig_mdl) 
	{
		\includegraphics[width=.15\textwidth]{Figuras/machinelearning.png}
	};
	
	\node[inner sep=0pt, right of=signal, xshift=2.4cm] (fig_signal) 
	{
		\includegraphics[width=.20\textwidth]{Figuras/processamentosinais.png}
	};
	
	\node[inner sep=0pt, right of=obi, xshift=2.4cm] 	(fig_obi) 
	{
		\includegraphics[width=.15\textwidth]{Figuras/gohb.png}
	};
	
	\end{tikzpicture}
\end{frame}
%------------------------------------------------

\begin{frame}
	\frametitle{Processamento}
	{
		\begin{itemize}
			\item Visual Odometry / SLAM / Outros algoritmos de visão
			\item Filtro de Partículas
			\item Algoritmos de Localização
			\item Planejamento de Caminhos e Trajetórias
			\item Controle (PID, Realimentação de estados)
			\item Filtro de Kalman
		\end{itemize}
	}
\end{frame}

\begin{frame}
	\frametitle{Atuadores}
	\centering
	\begin{tikzpicture}
		[vermelho/.style={rectangle, draw=red!60, 	fill=red!5, 	very thick, minimum size = 10mm},
		verde/.style	={rectangle, draw=green!60, fill=green!5, 	very thick, minimum size = 10mm},]
	
		\node[vermelho] 							(act)		{Atuadores};
		\node[verde, right of=act ,xshift=2.9cm] 	(uav) 		{Motores de UAV (Drones)};
		\node[verde, above of=uav ,yshift=1cm] 		(ser)		{Servo Motores};
		\node[verde, above of=ser ,yshift=1cm] 		(auv)		{Motores Aquáticos (AUV)};
		\node[verde, below of=uav ,yshift=-1cm] 	(gas)		{Motores Combustão};	
		\node[inner sep=0pt, above of=think, yshift=1cm] (fig_act) 
		{
			\includegraphics[width=.25\textwidth]{Figuras/act.png}
		};
	
		\node[inner sep=0pt, right of=uav, xshift=2.7cm] 	(fig_uav) 
		{
			\includegraphics[width=.15\textwidth]{Figuras/uav.jpg}
		};
	
		\node[inner sep=0pt, right of=ser, xshift=2.7cm] 	(fig_ser) 
		{
			\includegraphics[width=.15\textwidth]{Figuras/servo.jpg}
		};
	
		\node[inner sep=0pt, right of=auv, xshift=2.7cm] 	(fig_auv) 
		{
			\includegraphics[width=.15\textwidth]{Figuras/auv.jpg}
		};
	
		\node[inner sep=0pt, right of=gas, xshift=2.7cm] 	(fig_gas) 
		{
			\includegraphics[width=.15\textwidth]{Figuras/gas.jpg}
		};
	
	\end{tikzpicture}	
\end{frame}

\begin{frame}
    \frametitle{Motivação para o ROS}
    \begin{columns}
        \only<1-2>
        {
            \begin{column}{0.5\textwidth}
                \centering
                \includegraphics[width=4.3cm, height=6.5cm]{Figuras/Re-Inventing.png}
            \end{column}
        }
        \only<2>
        {
            \begin{column}{0.5\textwidth}
                \href{https://spectrum.ieee.org/automaton/robotics/robotics-software/the-origin-story-of-ros-the-linux-of-robotics}{The Origin Story of ROS, the Linux of Robotics} \\~\\
                \scriptsize
                "The world's most influential robotics software platform"
            \end{column}
        }	
    \end{columns}	
\end{frame}

\begin{frame}
    \frametitle{Motivação para o ROS}
    \centering
    \includegraphics[width=8cm, height=6.5cm]{img02/ciclo_ros.png}
\end{frame}

\section{Objetivos}

\begin{frame}
    \frametitle{Objetivos do Curso}
    \begin{itemize}
        \item Entender o Ecossistema ROS (tópicos, nós, mensagens, serviços)
        \item Desenvolver aplicações para controlar o movimento de um robô
        \item Entender como a posição e orientação são representadas no ROS
        \item Desenvolver programas simples utilizando visão computacional
        \item Utilizar os simuladores (Exemplo: Gazebo)
    \end{itemize}
\end{frame}

\begin{frame}
    \frametitle{\href{https://www.ros.org/}{https://www.ros.org/}}
    \only<1>
    {
        \centering
		\includegraphics[width=10cm, height=6cm]{img02/siteROS1.png}        
    }

    \only<2>
    {
        \centering
        \includegraphics[width=10cm, height=6cm]{img02/siteROS2.png}
    }
\end{frame}

\section{História}
\begin{frame}
	\frametitle{História do ROS}
	\begin{itemize}
		\scriptsize
		\item \href{https://www.linkedin.com/in/eric-berger-806b3b3/}{Eric Berger} e {Keenan Wyrobek} começaram o Doutorado (Ph.D) em Stanford...
		\item Buscaram levantar fundos para o desenvolvimento do projeto o Linux da Robótica.
		\item PR2 (Personal Robotics) - \href{https://www.youtube.com/watch?v=oyHWkQcin7I&feature=youtu.be}{\underline{ver vídeo}}
		\item Atualmente: Ecossistema ROS - Qualquer grupo pode iniciar um repositório de código do ROS ("\textit{federated model}")
	\end{itemize}
			
	\begin{table}
		\begin{tabular}{ | >{\centering\arraybackslash}m{1cm}  >{\centering\arraybackslash}m{7cm} | }  
			\hline
			\includegraphics[width=0.6cm,height=0.6cm]{Figuras/stanford.png} & Stanford Personal Robotics Program (janeiro de 2007)\\
			\hline
			\includegraphics[width=0.6cm,height=0.6cm]{Figuras/willowgarage.jpg} & Laboratório de pesquisa em robótica e incubadora tecnológica (Novembro de 2007) \\
			\hline
			\includegraphics[width=0.6cm,height=0.6cm]{Figuras/openrobotics.png} & Open Source Robotics Foundation (OSRF) ou Open Robotics (Fevereiro de 2013)\\
			\hline
		\end{tabular}
	\end{table}
	
	\centering
	\scriptsize
	"Don’t let anyone crush your crazy" ou "Não deixe ninguém esmagar sua loucura"
\end{frame}


\begin{frame}
	\frametitle{RosCon - ROS Conference (\href{https://roscon.ros.org/2024/} {https://roscon.ros.org/2024/})} 
        \begin{columns}
            \begin{column}{0.5\textwidth}
        	\centering
                \only<1>
                {
    	          \includegraphics[width=4cm,height=6cm]{img02/ROSCon1.png} \\~\\
                }
                \only<2>
                {
    	          \includegraphics[width=6cm,height=3.5cm]{img02/ROSCon2.png} \\~\\
                }
            \end{column}
            \begin{column}{0.5\textwidth}
        	\scriptsize
        	Conjugada com o IROS (ou ICRA):
        	\begin{itemize}
            		\item \href{https://www.iros25.org/}{IEEE/RSJ International Conference on Intelligent Robots and Systems}
            		\item \href{(https://2025.ieee-icra.org/}{International Conference on Robotics and Automation}
            	\end{itemize}
            \end{column}
        \end{columns}
\end{frame}



%\section{Contato}

\begin{frame}
    \frametitle{Sensores de Contato}
    \centering
    \begin{columns}
        \begin{column}{0.5\textwidth}
            \centering
            \begin{tabular}{c c c}
                \includegraphics[width=0.3\textwidth]{img05/contato_1.png} &
                \includegraphics[width=0.3\textwidth]{img05/contato_2.png} &
                \includegraphics[width=0.3\textwidth]{img05/contato_3.png} \\
            \end{tabular}
            \includegraphics[width=0.7\textwidth]{img05/contato_4.png} \\            
        \end{column}
        \begin{column}{0.5\textwidth}
            \begin{itemize}
                \scriptsize
                \item Os sensores táteis, adquirem a informação de contato com o ambiente por meio de interação física.
                \item Podem servir a diversos propósitos, tal qual um para-choque ou então um sensor que identifica a uma garra quando parar, ou em mãos robóticas, informando que o toque ocorreu.
                \item Os mais comuns empregam chaves normalmente abertas e fechadas, efeitos piezelétricos e piezoresistivos, capacitivos e assim por diante.
            \end{itemize}
        \end{column}    
    \end{columns}
\end{frame}

\section{Encoder}
\begin{frame}
    \frametitle{Sensor de movimento: Encoder}
    \centering
    \begin{columns}
        \begin{column}{0.5\textwidth}
            \centering
            \begin{tabular}{c c}
                \includegraphics[width=0.4\textwidth]{img05/encoder_1.png} &
                \includegraphics[width=0.4\textwidth]{img05/encoder_2.png} \\
            \end{tabular}
            \includegraphics[width=0.5\textwidth]{img05/lente_fotografica.png} \\            
        \end{column}
        \begin{column}{0.5\textwidth}
            \begin{itemize}
                \scriptsize
                \item É um dispositivo eletromecânico que conta ou reproduz pulsos elétricos a partir do movimento rotacional de seu eixo.
                \item Pode ser definido como um transdutor de posição angular.       
                \item \textbf{Exemplo de utilização de encoders em robótica}: medir a posição ou velocidade de motores utilizados para mover robôs.
                \item Se os valores dos encoders forem integrados (utilize um integrador digital), proporcionarão uma estimação da posição do robô. A posição do robô é chamada de \textbf{hodometria}.
            \end{itemize}
        \end{column}    
    \end{columns}
\end{frame}

\begin{frame}
    \frametitle{Sensor de movimento: Encoder}
    \begin{columns}
        \begin{column}{0.5\textwidth}
            \centering
            \scriptsize
            \textbf{Funcionamento regular}: o aparato conta o número de incrementos/transições (1 para sinal elevado – houve uma transição – e 0 para sinal baixo – ainda não houve uma transição) mas não pode dizer a direção do movimento.
        \end{column}
        \begin{column}{0.5\textwidth}
            \begin{itemize}
                \scriptsize
                \item Os encoders são sensores \textbf{proprioceptivos}, pois medem estados internos do robô (no caso, posição e velocidade).
                \item Dependendo do fabricante, encontra-se de diversas resoluções, desde 64 a mais de 50.000 incrementos (“ticks”, pulsos, contagens) por cada revolução do eixo do motor.
                \item Para maiores resoluções pode-se utilizar interpolação.
            \end{itemize}
        \end{column}
    \end{columns}
    \centering
    \includegraphics[width=0.7\textwidth]{img05/encoder_3.png} \\            
\end{frame}

\begin{frame}
    \frametitle{Sensor de movimento: Encoder}
    \begin{columns}
        \begin{column}{0.5\textwidth}
            \centering
            \scriptsize
            \textbf{Funcionamento em quadratura}: utiliza-se dois sensores deslocados entre si em fase de 90º (quadratura, um elevado e outro baixo). A ordem de qual sensor produz sinal elevado (1) primeiro diz a direção do movimento. Além da direção, a resolução neste modo de funcionamento é 4 vezes maior.
        \end{column}
        \begin{column}{0.5\textwidth}
            \centering
            \includegraphics[width=0.5\textwidth]{img05/encoder_4.png} \\            
        \end{column}
    \end{columns}
    \begin{columns}
        \begin{column}{0.5\textwidth}
            \centering
            \includegraphics[width=0.7\textwidth]{img05/encoder_6.png} \\            
        \end{column}
        \begin{column}{0.5\textwidth}
            \centering
            \includegraphics[width=0.7\textwidth]{img05/encoder_5.png} \\            
        \end{column}
    \end{columns}
\end{frame}

\section{Giros, Acelerômetros, IMUs}
\begin{frame}
    \frametitle{Sensor de orientação: Giroscópio}
    \centering
    \begin{columns}
        \begin{column}{0.5\textwidth}
            \centering
            \begin{tabular}{c}
                \includegraphics[width=0.8\textwidth]{img05/giroscópio.png} \\
            \end{tabular}
        \end{column}
        \begin{column}{0.5\textwidth}
            \begin{itemize}
                \scriptsize
                \item Os \textbf{ópticos} utilizam lasers emitidos em duas direções, uma viajando no sentido horário e outra no sentido anti-horário.
                \item Quando o aparato gira, as fases relativas dos lasers são deslocadas de acordo com a velocidade angular do movimento, uma vez que para um dos lasers o caminho ficou mais longo, e para o outro mais curto (efeito Sagnac).
                \item Mede-se a diferença de fase entre os dois feixes, e essa diferença entre as fases dos feixes é proporcional à velocidade angular do corpo e, consequentemente, ao ângulo do aparato girante.
                \item Ring Laser Gyroscope, RLG, que utiliza o efeito Sagnac
            \end{itemize}
        \end{column}    
    \end{columns}
\end{frame}

\begin{frame}
    \frametitle{Sensor de Inércia: Acelerômetros}
    \begin{columns}
        \begin{column}{0.4\textwidth}
            \centering
            \includegraphics[width=0.7\textwidth]{img05/massa_mola_amortecedor.png}
            \begin{mdframed}[%
                            backgroundcolor=blue!20,   % cor de fundo
                            linecolor=red,               % cor da borda
                            linewidth=1pt,               % espessura da borda
                            roundcorner=4pt,             % cantos arredondados
                            innertopmargin=6pt,          % espaço interno acima
                            innerbottommargin=6pt,       % espaço interno abaixo
                            innerleftmargin=6pt,         % espaço interno à esquerda
                            innerrightmargin=6pt         % espaço interno à direita
                            ]
                \scriptsize
                \begin{itemize}
                    \scriptsize
                    \item Na superfície da Terra, o acelerômetro sempre indicará, pelo menos, 1g ao longo do eixo vertical, e 0g em queda-livre.
                    \item Por isso, para obter a aceleração inercial (correspondente ao movimento do corpo), é necessário subtrair a gravidade.
                \end{itemize}
            \end{mdframed}
        \end{column}
        \begin{column}{0.6\textwidth}
            \centering
            \begin{itemize}
                \scriptsize
                \item Conforme os princípios físicos, para alterar movimentos de corpos é necessário imprimir forças sobre eles.
                \item Os acelerômetros são sensores que medem todas as forças que estejam atuando sobre eles, incluindo a gravidade.
                \item Os acelerômetros atuam como sistemas massa-mola-amortecedor, equacionado como:
            \end{itemize}
            \begin{equation*}
                F_{\text{aplicada}} = F_{\text{inercial}} + F_{\text{amortecida}} + F_{\text{elástica}} 
            \end{equation*}
            \begin{equation*}
                F_{\text{aplicada}} = m\ddot{x} + c\dot{x} + kx
            \end{equation*}
            \begin{itemize}
                \scriptsize
                \item[] Onde:
                \item \textbf{m} é a massa de prova.
                \item \textbf{c} é o coeficiente de amortecimento.
                \item \textbf{k} é a constante da mola.
            \end{itemize}
        \end{column}
    \end{columns}
\end{frame}

\begin{frame}
    \frametitle{Unidades de Medidas Inerciais (IMU)}
    \centering
    \begin{columns}
        \begin{column}{0.5\textwidth}
            \centering
            \begin{tabular}{c}
                \includegraphics[width=0.9\textwidth]{img05/IMU.png} \\
            \end{tabular}
        \end{column}
        \begin{column}{0.5\textwidth}
            \begin{itemize}
                \scriptsize
                \item As \textbf{Inertial Measurement Units (IMUs)} são dispositivos que medem posições relativas (x, y, z), orientações (roll, pitch, yaw), velocidades e acelerações de um corpo em movimento.
                \item Combinam as funções dos acelerômetros, giroscópios, bússolas além de possuírem integradores para o cálculo de hodometria.
                \item São unidades inerciais completas, sensíveis a erros, corrigindo a hodometria simples (dead reckoning) provida pelos encoders.
            \end{itemize}
        \end{column}    
    \end{columns}
\end{frame}

\section{Câmeras}
\begin{frame}
    \frametitle{Câmeras}
    \centering
    \begin{columns}
        \begin{column}{0.5\textwidth}
            \centering
            \begin{tabular}{c}
                \includegraphics[width=0.7\textwidth]{img05/câmera_2.png} \\
                \includegraphics[width=0.7\textwidth]{img05/câmera_1.png} \\
            \end{tabular}
        \end{column}
        \begin{column}{0.5\textwidth}
            \begin{itemize}
                \scriptsize
                \item As câmeras são sensores bastante populares, que podem ser utilizados das mais variadas formas.
                \item A área de pesquisa em imagens é denominada de visão computacional, em que a visão robótica é uma das áreas.
                \item Por meio de técnicas de manipulação de imagens, é possível obter distâncias, características (features), identificar objetos e pessoas, obter caminhos, dentre muitas outras aplicações.
            \end{itemize}
        \end{column}    
    \end{columns}
\end{frame}

\begin{frame}
    \frametitle{Câmeras - Knectic}
    \centering
    \includegraphics[width=0.95\textwidth]{img05/kinect.png}
\end{frame}

\begin{frame}{O que é a Câmera Kinect?}
    \begin{itemize}
        \item Sensor desenvolvido originalmente pela Microsoft para o Xbox.
        \item Usa visão computacional para captar profundidade, cor e movimento.
        \item Combina:
        \begin{itemize}
            \item Câmera RGB (imagem tradicional)
            \item Sensor de profundidade (depth camera)
            \item Microfones e acelerômetros (em versões completas)
        \end{itemize}
    \end{itemize}
    \end{frame}

\begin{frame}{Câmera Kinect: Como Funciona?}
    \begin{itemize}
        \item A câmera de profundidade projeta um padrão infravermelho (IR).
        \item O padrão refletido é captado por um sensor e processado para estimar a distância de cada ponto.
        \item Resulta em uma \textbf{imagem de profundidade} (depth map), com cada pixel representando uma distância.
    \end{itemize}
    
    \vspace{0.5cm}
    \textbf{Alternativamente (em modelos mais novos):} usa tecnologia de \textit{Time-of-Flight (ToF)} para medir o tempo que a luz leva para voltar ao sensor.
\end{frame}

\begin{frame}{Câmera Kinect: Vantagens}
    \begin{itemize}
        \item Captura simultânea de imagem colorida e profundidade.
        \item Excelente para rastreamento de corpo humano e gestos.
        \item Pronto para uso com SDKs acessíveis.
        \item Operação em tempo real com boa precisão.
    \end{itemize}
\end{frame}

\begin{frame}{Câmera Kinect: Aplicações}
    \begin{itemize}
        \item Jogos e interfaces baseadas em gestos.
        \item Robótica (percepção 3D e navegação).
        \item Mapeamento de ambientes (SLAM).
        \item Realidade aumentada e captura de movimento.
        \item Reconhecimento de pessoas e gestos em segurança.
    \end{itemize}
\end{frame}

\begin{frame}{O que são Câmeras Baseadas em Eventos?}
    \begin{itemize}
        \item Diferente das câmeras tradicionais, que capturam imagens em quadros fixos.
        \item Cada pixel funciona de forma independente e assíncrona.
        \item Gera eventos apenas quando há variação significativa de luminosidade.
    \end{itemize}
    
    \vspace{0.5cm}
    \textbf{Cada evento contém:}
    \begin{itemize}
        \item Posição do pixel $(x, y)$
        \item Tempo do evento (alta precisão temporal)
        \item Polaridade da variação (aumento ou redução de brilho)
    \end{itemize}
    \end{frame}

\begin{frame}{Câmera Baseada em Eventos: Vantagens}
    \begin{itemize}
        \item Altíssima resolução temporal (microsegundos)
        \item Baixa latência e baixo consumo de energia
        \item Alta faixa dinâmica (HDR) – ideal para ambientes com iluminação extrema
        \item Excelente desempenho com objetos rápidos
    \end{itemize}
\end{frame}

\begin{frame}{Câmera Baseada em Eventos: Aplicações}
    \begin{itemize}
        \item Robótica e drones autônomos
        \item Carros autônomos
        \item Sistemas biomiméticos (inspirados na retina)
        \item Visão computacional de alta velocidade
    \end{itemize}
    \href{https://youtu.be/0wGBpgIrd9M?si=j-tcaOsuDFyUieXV}{Link: Davide Scaramuzza }
\end{frame}

\section{GPS}
\begin{frame}
    \frametitle{Global Positioning System (GPS)}
    \centering
    \begin{columns}
        \begin{column}{0.4\textwidth}
            \centering
            \begin{tabular}{c}
                \includegraphics[width=0.8\textwidth]{img05/gps.png}\\
            \end{tabular}
        \end{column}
        \begin{column}{0.6\textwidth}
            \begin{itemize}
                \scriptsize
                \item O Sistema de Posicionamento Global é um sistema de localização global baseado em satélites e trilateração.
                \item Desenvolvido originalmente pelos USA nas décadas de 1950 e 1960 para estratégias exclusivamente militares, foi liberado  uso pela população civil na metade da década de 1990.
                \item Outros países desenvolveram seus próprios sistemas baseados em satélites, como o COMPASS (ou, BeiDou) da China, operacional a partir da década de 2000, o GLONASS russo, contemporâneo ao GPS, e o GALILEO da União Europeia e o NAVIC da Índia.
                \item O projeto original do GPS contava com 24 satélites em órbitas de 12 horas a uma altura de, aproximadamente, 21000 km acima da superfície terrestre, já no meio interplanetário.
            \end{itemize}
        \end{column}    
    \end{columns}
\end{frame}

\begin{frame}{Sistemas de Posicionamento e Segurança Cibernética}
    \textbf{Contexto:}
    \begin{itemize}
        \item O GPS foi desenvolvido pelos EUA com objetivos militares e só posteriormente liberado ao uso civil.
        \item A dependência exclusiva de um sistema estrangeiro representa um risco estratégico e cibernético.
        \item Outros países criaram seus próprios sistemas (GLONASS - Rússia, BeiDou - China, GALILEO - UE, NAVIC - Índia) para garantir autonomia.
    \end{itemize}
    
    \vspace{0.5cm}
    \textbf{Riscos associados à dependência do GPS:}
    \begin{itemize}
        \item Sinal pode ser bloqueado, degradado ou falsificado (\textit{spoofing}) em tempos de crise.
        \item Impacto direto em setores críticos: telecomunicações, transporte, energia e defesa.
        \item Falta de controle sobre atualizações, segurança e integridade dos dados recebidos.
    \end{itemize}
\end{frame}

\begin{frame}{Independência Tecnológica e Geopolítica}
    \begin{itemize}
        \item O domínio de sistemas de navegação por satélite está ligado à soberania nacional e geopolítica.
        \item Ter um sistema próprio ou acesso a múltiplos sistemas reduz vulnerabilidades.
        \item Integração com estratégias de defesa cibernética e proteção de infraestrutura crítica.
        \item Investimentos em sistemas nacionais impulsionam inovação e capacitação tecnológica interna.
    \end{itemize}
    
    \vspace{0.3cm}
    \textbf{Conclusão:} Em um mundo cada vez mais digital e conectado, o controle autônomo sobre sistemas espaciais é um ativo estratégico fundamental.
\end{frame}

\begin{frame}
    \frametitle{Global Positioning System (GPS)}
    \centering
    \begin{columns}
        \begin{column}{0.4\textwidth}
            \centering
            \begin{tabular}{c}
                \includegraphics[width=0.8\textwidth]{img05/satelite_gps.png}\\
            \end{tabular}
        \end{column}
        \begin{column}{0.6\textwidth}
            \begin{itemize}
                \scriptsize
                \item Os satélites de GPS se chamam NAVSTAR (NAVigation Satellite with Time And Ranging).
                \item Cada satélite possui um relógio atômico para sincronia de posicionamento, corrige o efeito Doppler na transmissão de seus dados por meio da Teoria da Relatividade.
                \item As órbitas desses satélites são tais que, a todo momento no horizonte de um usuário, existam, pelo menos, quatro satélites disponíveis.
            \end{itemize}
        \end{column}    
    \end{columns}
\end{frame}

\begin{frame}
    \frametitle{Global Positioning System (GPS)}
    \begin{itemize}
        \scriptsize
        \item A localização do GPS é feita por meio do princípio matemático da trilateração. No caso, trilateração em 3D.
        \item Existem estações GPS bem estabelecidas por todo o planeta.
    	\item A trilateração em 2D é facilmente explicada da seguinte maneira: 
    \end{itemize}
    \only<1>
    {
        \begin{itemize}
            \scriptsize
            \item[] Uma fonte (ex: satélite) lhe revela que está a uma distância d1 de uma estação conhecida (ex: numa cidade) denominada A.
        \end{itemize}
        \centering                
        \includegraphics[width=0.8\textwidth]{img05/gps_1.png}\\
    }
    \only<2>
    {
        \begin{itemize}
            \scriptsize
            \item[] Uma segunda fonte (outro satélite) informa que o usuário está a uma distância d2 de um outro ponto conhecido B.
        \end{itemize}
        \centering                
        \includegraphics[width=0.9\textwidth]{img05/gps_2.png}\\
    }
    \only<3>
    {
        \begin{itemize}
            \scriptsize
            \item[] Para eliminar a ambiguidade, é necessário obter a informação de uma terceira fonte (satélite), de uma distância d3 de C.
        \end{itemize}
        \centering                
        \includegraphics[width=0.9\textwidth]{img05/gps_3.png}\\
    }
    \only<4>
    {
        \centering                
        \includegraphics[width=0.95\textwidth]{img05/gps_4.png}\\
    }
\end{frame}

%\section{Arquiteturas}

\begin{frame}
    \frametitle{P3-DX}
    \only<1>
    {
    \centering
    \includegraphics[width=9cm, height=4cm]{img03/P3-DX-Aria-1.png} 
    }
    \only<2>
    {
    \centering
    \includegraphics[width=9cm, height=4cm]{img03/P3-DX-Aria-2.png} 
    }
\end{frame}

\begin{frame}
    \frametitle{Pixhalk}
    \begin{columns}
        \begin{column}{0.3\textwidth}
            \centering
            \includegraphics[width=4cm, height=3cm]{img03/cube_orange.jpg} 
        \end{column}
        \begin{column}{0.7\textwidth}
            \only<1>
            {
            \centering
            \includegraphics[width=6cm, height=4cm]{img03/rover.png} 
            }
            \only<2>
            {
            \centering
            \includegraphics[width=6cm, height=4cm]{img03/drone.png} 
            }
        \end{column}
    \end{columns}
\end{frame}

\begin{frame}
    \frametitle{Pixhalk}
    \only<1>
    {
    \centering
    \includegraphics[width=5cm, height=7.5cm]{img03/pixhalk_3.png} 
    }
    \only<2>
    {
    \centering
    \includegraphics[width=9cm, height=6cm]{img03/pixhalk_2.png} 
    }
\end{frame}

\begin{frame}
    \frametitle{Framework Completo}
    \centering
    \includegraphics[width=9cm, height=6cm]{img03/mavros.png} 
\end{frame}

\begin{frame}
    \frametitle{Mission Planner}
    \centering
    \includegraphics[width=9cm, height=6cm]{img03/missionplanner.png} 
\end{frame}

\begin{frame}
    \frametitle{QGroundControl}
    \centering
    \includegraphics[width=9cm, height=6cm]{img03/qgroundcontrol.png} 
\end{frame}

\begin{frame}
    \frametitle{Outros}
    \centering
    \includegraphics[width=9cm, height=4.5cm]{img03/outros.png} 
\end{frame}

\begin{frame}
    \frametitle{Exemplo de Configuração - Robô Real}
    \centering
    \includegraphics[width=9cm, height=4cm]{img03/exemplo_config_1.png} 
\end{frame}

\begin{frame}
    \frametitle{Software In The Loop}
    \centering
    \includegraphics[width=9cm, height=4.5cm]{img03/exemplo_config_2_sitl.png} 
\end{frame}

\section{Simuladores}
\begin{frame}
    \frametitle{Simuladores}
    \scriptsize
    \begin{itemize}
        \item Simuladores são convenientes para o desenvolvimento de aplicações robóticas sem a logística necessária para a operação de robôs móveis. Permitem testar as aplicações em ambientes de qualquer dimensão e diferentes topologias (diferentes maneiras de representar o ambiente).
        \item Simuladores implementam todo o software embarcado nos robôs reais e simulam a cinemática e dinâmica dos robôs. Podem oferecer visualização 2D ou 3D.
    \end{itemize}
    \only<1>
    {
    \begin{table}[ht]
      \centering
      \begin{tabular}{cc}
        \includegraphics[width=0.5\linewidth]{img03/v-rep.png} &
        \includegraphics[width=0.34\linewidth]{img03/mobilesim.png} \\
        (V-Rep) & (MobileSim) \\
      \end{tabular}
    \end{table}
    }
    \only<2>
    {
    \begin{table}[ht]
      \centering
      \begin{tabular}{cc}
        \includegraphics[width=0.5\linewidth]{img03/gazebo_1.png} &
        \includegraphics[width=0.34\linewidth]{img03/gazebo_2.png} \\
        (Gazebo) & (Gazebo) \\
      \end{tabular}
    \end{table}
    }
    \only<3>
    {
        \centering
        \includegraphics[width=0.35\linewidth]{img03/turtlesim.png} &
    }
\end{frame}




% (AULA 04) - Locomoção
%\section{Bibliografia}


\begin{frame}
	\frametitle{Livros Texto}
	\only<1>
	{
		  \centering
		  \includegraphics[width=4.5cm, height=6cm]{Figuras/RobotProgramming.png}
	}

        \only<2>
	{
		\begin{columns}
			\begin{column}{0.5\textwidth}
				\centering
				\includegraphics[width=4.5cm, height=6cm]{Figuras/ROS_Springer.jpg}
			\end{column} 
			\begin{column}{0.5\textwidth}
				\centering
				\includegraphics[width=4.5cm, height=6cm]{Figuras/Learning.jpg}
			\end{column} 
		\end{columns}
	}	

        \only<3>
	{
		\begin{columns}
			\begin{column}{0.5\textwidth}
				\centering
				\includegraphics[width=4.5cm, height=6cm]{Figuras/LivroPython1.png}
			\end{column} 
			\begin{column}{0.5\textwidth}
				\centering
				\includegraphics[width=4.5cm, height=6cm]{Figuras/LivroPython2.png}
			\end{column} 
		\end{columns}
	}	
\end{frame}

\begin{frame}
	\frametitle{ROS Website (\href{www.ros.org}{www.ros.org})} 
	\only<1>
	{
		\includegraphics[width=9cm, height=6cm]{Figuras/ROS_Website.png} 
	}
	\only<2>
	{
		\includegraphics[width=9cm, height=6cm]{Figuras/ROS_Website_2.png} 
	}
\end{frame}

\section{Motivação}

\begin{frame}
	\frametitle{Ciclo de um Processo em Robótica} 
	\centering

	\begin{tikzpicture}
						[squarednode/.style={rectangle, draw=red!60, fill=red!5, very thick, minimum size = 10mm},]
		\only<1-3>
		{
			\node[squarednode] 									(sense)	{Sensores};
		}
		\only<2-3>
		{	
			\node[squarednode, right of=sense,xshift=2.4cm] 	(think) {Processamento};
		}
		\only<3->
		{	
			\node[squarednode, right of=think,xshift=2.4cm] 	(act)	{Atuação};
		}
		\only<1-3>
		{
			\node[inner sep=0pt, above of=sense, yshift=1cm] 	(fig_sense) 
			{
				\includegraphics[width=.25\textwidth]{Figuras/vision.jpg}
			};
		}
		\only<2-3>
		{		
			\node[inner sep=0pt, above of=think, yshift=1cm] 	(fig_think) 
			{
				\includegraphics[width=.25\textwidth]{Figuras/brain.jpg}
			};
		}
		\only<3->
		{		
			\node[inner sep=0pt, above of=act, yshift=1cm] 		(fig_act) 
			{
				\includegraphics[width=.25\textwidth]{Figuras/act.png}
			};
		}			
		\only<2-3>
		{	
			\draw[->, thick, draw=red!60] (sense.east) -- (think.west) ;
		}
		\only<3->
		{
			\draw[->, thick, draw=red!60] (think.east) -- (act.west) ;
		}
	\end{tikzpicture}
	
\end{frame}

\begin{frame}
	\frametitle{Sensores}
	\centering
	\begin{tikzpicture}
	[vermelho/.style={rectangle, draw=red!60, 	fill=red!5, 	very thick, minimum size = 10mm},
	verde/.style	={rectangle, draw=green!60, fill=green!5, 	very thick, minimum size = 10mm},]
	
	\node[vermelho] 								(sense)		{Sensores};
	\node[verde, right of=sense	 	,xshift=2.4cm] 	(ultrassom) {Ultrassom};
	\node[verde, above of=ultrassom	,yshift=1cm] 	(lasers)	{Laser Scan};
	\node[verde, above of=lasers	,yshift=1cm] 	(camera)	{Câmera};
	\node[verde, below of=ultrassom	,yshift=-1cm] 	(GPS)		{GPS};	
	\node[inner sep=0pt, above of=sense, yshift=1cm] 	(fig_sense) 
	{
		\includegraphics[width=.25\textwidth]{Figuras/vision.jpg}
	};

	\node[inner sep=0pt, right of=camera, xshift=2.4cm] 	(fig_camera) 
	{
		\includegraphics[width=.15\textwidth]{Figuras/camera.jpg}
	};

	\node[inner sep=0pt, right of=lasers, xshift=2.4cm] 	(fig_lasers) 
	{
		\includegraphics[width=.10\textwidth]{Figuras/lidar.jpg}
	};

	\node[inner sep=0pt, right of=ultrassom, xshift=2.4cm] 	(fig_ultrassom) 
	{
		\includegraphics[width=.15\textwidth]{Figuras/ultrassonic.jpg}
	};
	
	\node[inner sep=0pt, right of=GPS, xshift=2.4cm] 	(fig_gps) 
	{
		\includegraphics[width=.10\textwidth]{Figuras/gps.jpg}
	};
	
	\end{tikzpicture}
\end{frame}
%------------------------------------------------


\begin{frame}
	\frametitle{Processamento}
	\centering
	\begin{tikzpicture}
	[vermelho/.style={rectangle, draw=red!60, 	fill=red!5, 	very thick, minimum size = 10mm},
	verde/.style	={rectangle, draw=green!60, fill=green!5, 	very thick, minimum size = 10mm},]
	
	\node[vermelho] 								(think)		{Processamento};
	\node[verde, right of=think	 	,xshift=2.9cm] 	(obi) 		{Otimização Bioinspirada};
	\node[verde, above of=obi		,yshift=1cm] 	(mdl)		{Machine/Deep Learning};
	\node[verde, above of=mdl		,yshift=1cm] 	(ai)		{Inteligência Artificial};
	\node[verde, below of=obi		,yshift=-1cm] 	(signal)	{Processamento de Sinais};	
	\node[inner sep=0pt, above of=think, yshift=1cm] (fig_think) 
	{
		\includegraphics[width=.15\textwidth]{Figuras/brain.jpg}
	};
	
	\node[inner sep=0pt, right of=ai, xshift=2.4cm] 	(fig_ai) 
	{
		\includegraphics[width=.25\textwidth]{Figuras/ai.png}
	};
	
	\node[inner sep=0pt, right of=mdl, xshift=2.4cm] 	(fig_mdl) 
	{
		\includegraphics[width=.15\textwidth]{Figuras/machinelearning.png}
	};
	
	\node[inner sep=0pt, right of=signal, xshift=2.4cm] (fig_signal) 
	{
		\includegraphics[width=.20\textwidth]{Figuras/processamentosinais.png}
	};
	
	\node[inner sep=0pt, right of=obi, xshift=2.4cm] 	(fig_obi) 
	{
		\includegraphics[width=.15\textwidth]{Figuras/gohb.png}
	};
	
	\end{tikzpicture}
\end{frame}
%------------------------------------------------

\begin{frame}
	\frametitle{Processamento}
	{
		\begin{itemize}
			\item Visual Odometry / SLAM / Outros algoritmos de visão
			\item Filtro de Partículas
			\item Algoritmos de Localização
			\item Planejamento de Caminhos e Trajetórias
			\item Controle (PID, Realimentação de estados)
			\item Filtro de Kalman
		\end{itemize}
	}
\end{frame}

\begin{frame}
	\frametitle{Atuadores}
	\centering
	\begin{tikzpicture}
		[vermelho/.style={rectangle, draw=red!60, 	fill=red!5, 	very thick, minimum size = 10mm},
		verde/.style	={rectangle, draw=green!60, fill=green!5, 	very thick, minimum size = 10mm},]
	
		\node[vermelho] 							(act)		{Atuadores};
		\node[verde, right of=act ,xshift=2.9cm] 	(uav) 		{Motores de UAV (Drones)};
		\node[verde, above of=uav ,yshift=1cm] 		(ser)		{Servo Motores};
		\node[verde, above of=ser ,yshift=1cm] 		(auv)		{Motores Aquáticos (AUV)};
		\node[verde, below of=uav ,yshift=-1cm] 	(gas)		{Motores Combustão};	
		\node[inner sep=0pt, above of=think, yshift=1cm] (fig_act) 
		{
			\includegraphics[width=.25\textwidth]{Figuras/act.png}
		};
	
		\node[inner sep=0pt, right of=uav, xshift=2.7cm] 	(fig_uav) 
		{
			\includegraphics[width=.15\textwidth]{Figuras/uav.jpg}
		};
	
		\node[inner sep=0pt, right of=ser, xshift=2.7cm] 	(fig_ser) 
		{
			\includegraphics[width=.15\textwidth]{Figuras/servo.jpg}
		};
	
		\node[inner sep=0pt, right of=auv, xshift=2.7cm] 	(fig_auv) 
		{
			\includegraphics[width=.15\textwidth]{Figuras/auv.jpg}
		};
	
		\node[inner sep=0pt, right of=gas, xshift=2.7cm] 	(fig_gas) 
		{
			\includegraphics[width=.15\textwidth]{Figuras/gas.jpg}
		};
	
	\end{tikzpicture}	
\end{frame}

\begin{frame}
    \frametitle{Motivação para o ROS}
    \begin{columns}
        \only<1-2>
        {
            \begin{column}{0.5\textwidth}
                \centering
                \includegraphics[width=4.3cm, height=6.5cm]{Figuras/Re-Inventing.png}
            \end{column}
        }
        \only<2>
        {
            \begin{column}{0.5\textwidth}
                \href{https://spectrum.ieee.org/automaton/robotics/robotics-software/the-origin-story-of-ros-the-linux-of-robotics}{The Origin Story of ROS, the Linux of Robotics} \\~\\
                \scriptsize
                "The world's most influential robotics software platform"
            \end{column}
        }	
    \end{columns}	
\end{frame}

\begin{frame}
    \frametitle{Motivação para o ROS}
    \centering
    \includegraphics[width=8cm, height=6.5cm]{img02/ciclo_ros.png}
\end{frame}

\section{Objetivos}

\begin{frame}
    \frametitle{Objetivos do Curso}
    \begin{itemize}
        \item Entender o Ecossistema ROS (tópicos, nós, mensagens, serviços)
        \item Desenvolver aplicações para controlar o movimento de um robô
        \item Entender como a posição e orientação são representadas no ROS
        \item Desenvolver programas simples utilizando visão computacional
        \item Utilizar os simuladores (Exemplo: Gazebo)
    \end{itemize}
\end{frame}

\begin{frame}
    \frametitle{\href{https://www.ros.org/}{https://www.ros.org/}}
    \only<1>
    {
        \centering
		\includegraphics[width=10cm, height=6cm]{img02/siteROS1.png}        
    }

    \only<2>
    {
        \centering
        \includegraphics[width=10cm, height=6cm]{img02/siteROS2.png}
    }
\end{frame}

\section{História}
\begin{frame}
	\frametitle{História do ROS}
	\begin{itemize}
		\scriptsize
		\item \href{https://www.linkedin.com/in/eric-berger-806b3b3/}{Eric Berger} e {Keenan Wyrobek} começaram o Doutorado (Ph.D) em Stanford...
		\item Buscaram levantar fundos para o desenvolvimento do projeto o Linux da Robótica.
		\item PR2 (Personal Robotics) - \href{https://www.youtube.com/watch?v=oyHWkQcin7I&feature=youtu.be}{\underline{ver vídeo}}
		\item Atualmente: Ecossistema ROS - Qualquer grupo pode iniciar um repositório de código do ROS ("\textit{federated model}")
	\end{itemize}
			
	\begin{table}
		\begin{tabular}{ | >{\centering\arraybackslash}m{1cm}  >{\centering\arraybackslash}m{7cm} | }  
			\hline
			\includegraphics[width=0.6cm,height=0.6cm]{Figuras/stanford.png} & Stanford Personal Robotics Program (janeiro de 2007)\\
			\hline
			\includegraphics[width=0.6cm,height=0.6cm]{Figuras/willowgarage.jpg} & Laboratório de pesquisa em robótica e incubadora tecnológica (Novembro de 2007) \\
			\hline
			\includegraphics[width=0.6cm,height=0.6cm]{Figuras/openrobotics.png} & Open Source Robotics Foundation (OSRF) ou Open Robotics (Fevereiro de 2013)\\
			\hline
		\end{tabular}
	\end{table}
	
	\centering
	\scriptsize
	"Don’t let anyone crush your crazy" ou "Não deixe ninguém esmagar sua loucura"
\end{frame}


\begin{frame}
	\frametitle{RosCon - ROS Conference (\href{https://roscon.ros.org/2024/} {https://roscon.ros.org/2024/})} 
        \begin{columns}
            \begin{column}{0.5\textwidth}
        	\centering
                \only<1>
                {
    	          \includegraphics[width=4cm,height=6cm]{img02/ROSCon1.png} \\~\\
                }
                \only<2>
                {
    	          \includegraphics[width=6cm,height=3.5cm]{img02/ROSCon2.png} \\~\\
                }
            \end{column}
            \begin{column}{0.5\textwidth}
        	\scriptsize
        	Conjugada com o IROS (ou ICRA):
        	\begin{itemize}
            		\item \href{https://www.iros25.org/}{IEEE/RSJ International Conference on Intelligent Robots and Systems}
            		\item \href{(https://2025.ieee-icra.org/}{International Conference on Robotics and Automation}
            	\end{itemize}
            \end{column}
        \end{columns}
\end{frame}



%\section{Contato}

\begin{frame}
    \frametitle{Sensores de Contato}
    \centering
    \begin{columns}
        \begin{column}{0.5\textwidth}
            \centering
            \begin{tabular}{c c c}
                \includegraphics[width=0.3\textwidth]{img05/contato_1.png} &
                \includegraphics[width=0.3\textwidth]{img05/contato_2.png} &
                \includegraphics[width=0.3\textwidth]{img05/contato_3.png} \\
            \end{tabular}
            \includegraphics[width=0.7\textwidth]{img05/contato_4.png} \\            
        \end{column}
        \begin{column}{0.5\textwidth}
            \begin{itemize}
                \scriptsize
                \item Os sensores táteis, adquirem a informação de contato com o ambiente por meio de interação física.
                \item Podem servir a diversos propósitos, tal qual um para-choque ou então um sensor que identifica a uma garra quando parar, ou em mãos robóticas, informando que o toque ocorreu.
                \item Os mais comuns empregam chaves normalmente abertas e fechadas, efeitos piezelétricos e piezoresistivos, capacitivos e assim por diante.
            \end{itemize}
        \end{column}    
    \end{columns}
\end{frame}

\section{Encoder}
\begin{frame}
    \frametitle{Sensor de movimento: Encoder}
    \centering
    \begin{columns}
        \begin{column}{0.5\textwidth}
            \centering
            \begin{tabular}{c c}
                \includegraphics[width=0.4\textwidth]{img05/encoder_1.png} &
                \includegraphics[width=0.4\textwidth]{img05/encoder_2.png} \\
            \end{tabular}
            \includegraphics[width=0.5\textwidth]{img05/lente_fotografica.png} \\            
        \end{column}
        \begin{column}{0.5\textwidth}
            \begin{itemize}
                \scriptsize
                \item É um dispositivo eletromecânico que conta ou reproduz pulsos elétricos a partir do movimento rotacional de seu eixo.
                \item Pode ser definido como um transdutor de posição angular.       
                \item \textbf{Exemplo de utilização de encoders em robótica}: medir a posição ou velocidade de motores utilizados para mover robôs.
                \item Se os valores dos encoders forem integrados (utilize um integrador digital), proporcionarão uma estimação da posição do robô. A posição do robô é chamada de \textbf{hodometria}.
            \end{itemize}
        \end{column}    
    \end{columns}
\end{frame}

\begin{frame}
    \frametitle{Sensor de movimento: Encoder}
    \begin{columns}
        \begin{column}{0.5\textwidth}
            \centering
            \scriptsize
            \textbf{Funcionamento regular}: o aparato conta o número de incrementos/transições (1 para sinal elevado – houve uma transição – e 0 para sinal baixo – ainda não houve uma transição) mas não pode dizer a direção do movimento.
        \end{column}
        \begin{column}{0.5\textwidth}
            \begin{itemize}
                \scriptsize
                \item Os encoders são sensores \textbf{proprioceptivos}, pois medem estados internos do robô (no caso, posição e velocidade).
                \item Dependendo do fabricante, encontra-se de diversas resoluções, desde 64 a mais de 50.000 incrementos (“ticks”, pulsos, contagens) por cada revolução do eixo do motor.
                \item Para maiores resoluções pode-se utilizar interpolação.
            \end{itemize}
        \end{column}
    \end{columns}
    \centering
    \includegraphics[width=0.7\textwidth]{img05/encoder_3.png} \\            
\end{frame}

\begin{frame}
    \frametitle{Sensor de movimento: Encoder}
    \begin{columns}
        \begin{column}{0.5\textwidth}
            \centering
            \scriptsize
            \textbf{Funcionamento em quadratura}: utiliza-se dois sensores deslocados entre si em fase de 90º (quadratura, um elevado e outro baixo). A ordem de qual sensor produz sinal elevado (1) primeiro diz a direção do movimento. Além da direção, a resolução neste modo de funcionamento é 4 vezes maior.
        \end{column}
        \begin{column}{0.5\textwidth}
            \centering
            \includegraphics[width=0.5\textwidth]{img05/encoder_4.png} \\            
        \end{column}
    \end{columns}
    \begin{columns}
        \begin{column}{0.5\textwidth}
            \centering
            \includegraphics[width=0.7\textwidth]{img05/encoder_6.png} \\            
        \end{column}
        \begin{column}{0.5\textwidth}
            \centering
            \includegraphics[width=0.7\textwidth]{img05/encoder_5.png} \\            
        \end{column}
    \end{columns}
\end{frame}

\section{Giros, Acelerômetros, IMUs}
\begin{frame}
    \frametitle{Sensor de orientação: Giroscópio}
    \centering
    \begin{columns}
        \begin{column}{0.5\textwidth}
            \centering
            \begin{tabular}{c}
                \includegraphics[width=0.8\textwidth]{img05/giroscópio.png} \\
            \end{tabular}
        \end{column}
        \begin{column}{0.5\textwidth}
            \begin{itemize}
                \scriptsize
                \item Os \textbf{ópticos} utilizam lasers emitidos em duas direções, uma viajando no sentido horário e outra no sentido anti-horário.
                \item Quando o aparato gira, as fases relativas dos lasers são deslocadas de acordo com a velocidade angular do movimento, uma vez que para um dos lasers o caminho ficou mais longo, e para o outro mais curto (efeito Sagnac).
                \item Mede-se a diferença de fase entre os dois feixes, e essa diferença entre as fases dos feixes é proporcional à velocidade angular do corpo e, consequentemente, ao ângulo do aparato girante.
                \item Ring Laser Gyroscope, RLG, que utiliza o efeito Sagnac
            \end{itemize}
        \end{column}    
    \end{columns}
\end{frame}

\begin{frame}
    \frametitle{Sensor de Inércia: Acelerômetros}
    \begin{columns}
        \begin{column}{0.4\textwidth}
            \centering
            \includegraphics[width=0.7\textwidth]{img05/massa_mola_amortecedor.png}
            \begin{mdframed}[%
                            backgroundcolor=blue!20,   % cor de fundo
                            linecolor=red,               % cor da borda
                            linewidth=1pt,               % espessura da borda
                            roundcorner=4pt,             % cantos arredondados
                            innertopmargin=6pt,          % espaço interno acima
                            innerbottommargin=6pt,       % espaço interno abaixo
                            innerleftmargin=6pt,         % espaço interno à esquerda
                            innerrightmargin=6pt         % espaço interno à direita
                            ]
                \scriptsize
                \begin{itemize}
                    \scriptsize
                    \item Na superfície da Terra, o acelerômetro sempre indicará, pelo menos, 1g ao longo do eixo vertical, e 0g em queda-livre.
                    \item Por isso, para obter a aceleração inercial (correspondente ao movimento do corpo), é necessário subtrair a gravidade.
                \end{itemize}
            \end{mdframed}
        \end{column}
        \begin{column}{0.6\textwidth}
            \centering
            \begin{itemize}
                \scriptsize
                \item Conforme os princípios físicos, para alterar movimentos de corpos é necessário imprimir forças sobre eles.
                \item Os acelerômetros são sensores que medem todas as forças que estejam atuando sobre eles, incluindo a gravidade.
                \item Os acelerômetros atuam como sistemas massa-mola-amortecedor, equacionado como:
            \end{itemize}
            \begin{equation*}
                F_{\text{aplicada}} = F_{\text{inercial}} + F_{\text{amortecida}} + F_{\text{elástica}} 
            \end{equation*}
            \begin{equation*}
                F_{\text{aplicada}} = m\ddot{x} + c\dot{x} + kx
            \end{equation*}
            \begin{itemize}
                \scriptsize
                \item[] Onde:
                \item \textbf{m} é a massa de prova.
                \item \textbf{c} é o coeficiente de amortecimento.
                \item \textbf{k} é a constante da mola.
            \end{itemize}
        \end{column}
    \end{columns}
\end{frame}

\begin{frame}
    \frametitle{Unidades de Medidas Inerciais (IMU)}
    \centering
    \begin{columns}
        \begin{column}{0.5\textwidth}
            \centering
            \begin{tabular}{c}
                \includegraphics[width=0.9\textwidth]{img05/IMU.png} \\
            \end{tabular}
        \end{column}
        \begin{column}{0.5\textwidth}
            \begin{itemize}
                \scriptsize
                \item As \textbf{Inertial Measurement Units (IMUs)} são dispositivos que medem posições relativas (x, y, z), orientações (roll, pitch, yaw), velocidades e acelerações de um corpo em movimento.
                \item Combinam as funções dos acelerômetros, giroscópios, bússolas além de possuírem integradores para o cálculo de hodometria.
                \item São unidades inerciais completas, sensíveis a erros, corrigindo a hodometria simples (dead reckoning) provida pelos encoders.
            \end{itemize}
        \end{column}    
    \end{columns}
\end{frame}

\section{Câmeras}
\begin{frame}
    \frametitle{Câmeras}
    \centering
    \begin{columns}
        \begin{column}{0.5\textwidth}
            \centering
            \begin{tabular}{c}
                \includegraphics[width=0.7\textwidth]{img05/câmera_2.png} \\
                \includegraphics[width=0.7\textwidth]{img05/câmera_1.png} \\
            \end{tabular}
        \end{column}
        \begin{column}{0.5\textwidth}
            \begin{itemize}
                \scriptsize
                \item As câmeras são sensores bastante populares, que podem ser utilizados das mais variadas formas.
                \item A área de pesquisa em imagens é denominada de visão computacional, em que a visão robótica é uma das áreas.
                \item Por meio de técnicas de manipulação de imagens, é possível obter distâncias, características (features), identificar objetos e pessoas, obter caminhos, dentre muitas outras aplicações.
            \end{itemize}
        \end{column}    
    \end{columns}
\end{frame}

\begin{frame}
    \frametitle{Câmeras - Knectic}
    \centering
    \includegraphics[width=0.95\textwidth]{img05/kinect.png}
\end{frame}

\begin{frame}{O que é a Câmera Kinect?}
    \begin{itemize}
        \item Sensor desenvolvido originalmente pela Microsoft para o Xbox.
        \item Usa visão computacional para captar profundidade, cor e movimento.
        \item Combina:
        \begin{itemize}
            \item Câmera RGB (imagem tradicional)
            \item Sensor de profundidade (depth camera)
            \item Microfones e acelerômetros (em versões completas)
        \end{itemize}
    \end{itemize}
    \end{frame}

\begin{frame}{Câmera Kinect: Como Funciona?}
    \begin{itemize}
        \item A câmera de profundidade projeta um padrão infravermelho (IR).
        \item O padrão refletido é captado por um sensor e processado para estimar a distância de cada ponto.
        \item Resulta em uma \textbf{imagem de profundidade} (depth map), com cada pixel representando uma distância.
    \end{itemize}
    
    \vspace{0.5cm}
    \textbf{Alternativamente (em modelos mais novos):} usa tecnologia de \textit{Time-of-Flight (ToF)} para medir o tempo que a luz leva para voltar ao sensor.
\end{frame}

\begin{frame}{Câmera Kinect: Vantagens}
    \begin{itemize}
        \item Captura simultânea de imagem colorida e profundidade.
        \item Excelente para rastreamento de corpo humano e gestos.
        \item Pronto para uso com SDKs acessíveis.
        \item Operação em tempo real com boa precisão.
    \end{itemize}
\end{frame}

\begin{frame}{Câmera Kinect: Aplicações}
    \begin{itemize}
        \item Jogos e interfaces baseadas em gestos.
        \item Robótica (percepção 3D e navegação).
        \item Mapeamento de ambientes (SLAM).
        \item Realidade aumentada e captura de movimento.
        \item Reconhecimento de pessoas e gestos em segurança.
    \end{itemize}
\end{frame}

\begin{frame}{O que são Câmeras Baseadas em Eventos?}
    \begin{itemize}
        \item Diferente das câmeras tradicionais, que capturam imagens em quadros fixos.
        \item Cada pixel funciona de forma independente e assíncrona.
        \item Gera eventos apenas quando há variação significativa de luminosidade.
    \end{itemize}
    
    \vspace{0.5cm}
    \textbf{Cada evento contém:}
    \begin{itemize}
        \item Posição do pixel $(x, y)$
        \item Tempo do evento (alta precisão temporal)
        \item Polaridade da variação (aumento ou redução de brilho)
    \end{itemize}
    \end{frame}

\begin{frame}{Câmera Baseada em Eventos: Vantagens}
    \begin{itemize}
        \item Altíssima resolução temporal (microsegundos)
        \item Baixa latência e baixo consumo de energia
        \item Alta faixa dinâmica (HDR) – ideal para ambientes com iluminação extrema
        \item Excelente desempenho com objetos rápidos
    \end{itemize}
\end{frame}

\begin{frame}{Câmera Baseada em Eventos: Aplicações}
    \begin{itemize}
        \item Robótica e drones autônomos
        \item Carros autônomos
        \item Sistemas biomiméticos (inspirados na retina)
        \item Visão computacional de alta velocidade
    \end{itemize}
    \href{https://youtu.be/0wGBpgIrd9M?si=j-tcaOsuDFyUieXV}{Link: Davide Scaramuzza }
\end{frame}

\section{GPS}
\begin{frame}
    \frametitle{Global Positioning System (GPS)}
    \centering
    \begin{columns}
        \begin{column}{0.4\textwidth}
            \centering
            \begin{tabular}{c}
                \includegraphics[width=0.8\textwidth]{img05/gps.png}\\
            \end{tabular}
        \end{column}
        \begin{column}{0.6\textwidth}
            \begin{itemize}
                \scriptsize
                \item O Sistema de Posicionamento Global é um sistema de localização global baseado em satélites e trilateração.
                \item Desenvolvido originalmente pelos USA nas décadas de 1950 e 1960 para estratégias exclusivamente militares, foi liberado  uso pela população civil na metade da década de 1990.
                \item Outros países desenvolveram seus próprios sistemas baseados em satélites, como o COMPASS (ou, BeiDou) da China, operacional a partir da década de 2000, o GLONASS russo, contemporâneo ao GPS, e o GALILEO da União Europeia e o NAVIC da Índia.
                \item O projeto original do GPS contava com 24 satélites em órbitas de 12 horas a uma altura de, aproximadamente, 21000 km acima da superfície terrestre, já no meio interplanetário.
            \end{itemize}
        \end{column}    
    \end{columns}
\end{frame}

\begin{frame}{Sistemas de Posicionamento e Segurança Cibernética}
    \textbf{Contexto:}
    \begin{itemize}
        \item O GPS foi desenvolvido pelos EUA com objetivos militares e só posteriormente liberado ao uso civil.
        \item A dependência exclusiva de um sistema estrangeiro representa um risco estratégico e cibernético.
        \item Outros países criaram seus próprios sistemas (GLONASS - Rússia, BeiDou - China, GALILEO - UE, NAVIC - Índia) para garantir autonomia.
    \end{itemize}
    
    \vspace{0.5cm}
    \textbf{Riscos associados à dependência do GPS:}
    \begin{itemize}
        \item Sinal pode ser bloqueado, degradado ou falsificado (\textit{spoofing}) em tempos de crise.
        \item Impacto direto em setores críticos: telecomunicações, transporte, energia e defesa.
        \item Falta de controle sobre atualizações, segurança e integridade dos dados recebidos.
    \end{itemize}
\end{frame}

\begin{frame}{Independência Tecnológica e Geopolítica}
    \begin{itemize}
        \item O domínio de sistemas de navegação por satélite está ligado à soberania nacional e geopolítica.
        \item Ter um sistema próprio ou acesso a múltiplos sistemas reduz vulnerabilidades.
        \item Integração com estratégias de defesa cibernética e proteção de infraestrutura crítica.
        \item Investimentos em sistemas nacionais impulsionam inovação e capacitação tecnológica interna.
    \end{itemize}
    
    \vspace{0.3cm}
    \textbf{Conclusão:} Em um mundo cada vez mais digital e conectado, o controle autônomo sobre sistemas espaciais é um ativo estratégico fundamental.
\end{frame}

\begin{frame}
    \frametitle{Global Positioning System (GPS)}
    \centering
    \begin{columns}
        \begin{column}{0.4\textwidth}
            \centering
            \begin{tabular}{c}
                \includegraphics[width=0.8\textwidth]{img05/satelite_gps.png}\\
            \end{tabular}
        \end{column}
        \begin{column}{0.6\textwidth}
            \begin{itemize}
                \scriptsize
                \item Os satélites de GPS se chamam NAVSTAR (NAVigation Satellite with Time And Ranging).
                \item Cada satélite possui um relógio atômico para sincronia de posicionamento, corrige o efeito Doppler na transmissão de seus dados por meio da Teoria da Relatividade.
                \item As órbitas desses satélites são tais que, a todo momento no horizonte de um usuário, existam, pelo menos, quatro satélites disponíveis.
            \end{itemize}
        \end{column}    
    \end{columns}
\end{frame}

\begin{frame}
    \frametitle{Global Positioning System (GPS)}
    \begin{itemize}
        \scriptsize
        \item A localização do GPS é feita por meio do princípio matemático da trilateração. No caso, trilateração em 3D.
        \item Existem estações GPS bem estabelecidas por todo o planeta.
    	\item A trilateração em 2D é facilmente explicada da seguinte maneira: 
    \end{itemize}
    \only<1>
    {
        \begin{itemize}
            \scriptsize
            \item[] Uma fonte (ex: satélite) lhe revela que está a uma distância d1 de uma estação conhecida (ex: numa cidade) denominada A.
        \end{itemize}
        \centering                
        \includegraphics[width=0.8\textwidth]{img05/gps_1.png}\\
    }
    \only<2>
    {
        \begin{itemize}
            \scriptsize
            \item[] Uma segunda fonte (outro satélite) informa que o usuário está a uma distância d2 de um outro ponto conhecido B.
        \end{itemize}
        \centering                
        \includegraphics[width=0.9\textwidth]{img05/gps_2.png}\\
    }
    \only<3>
    {
        \begin{itemize}
            \scriptsize
            \item[] Para eliminar a ambiguidade, é necessário obter a informação de uma terceira fonte (satélite), de uma distância d3 de C.
        \end{itemize}
        \centering                
        \includegraphics[width=0.9\textwidth]{img05/gps_3.png}\\
    }
    \only<4>
    {
        \centering                
        \includegraphics[width=0.95\textwidth]{img05/gps_4.png}\\
    }
\end{frame}

%\section{Arquiteturas}

\begin{frame}
    \frametitle{P3-DX}
    \only<1>
    {
    \centering
    \includegraphics[width=9cm, height=4cm]{img03/P3-DX-Aria-1.png} 
    }
    \only<2>
    {
    \centering
    \includegraphics[width=9cm, height=4cm]{img03/P3-DX-Aria-2.png} 
    }
\end{frame}

\begin{frame}
    \frametitle{Pixhalk}
    \begin{columns}
        \begin{column}{0.3\textwidth}
            \centering
            \includegraphics[width=4cm, height=3cm]{img03/cube_orange.jpg} 
        \end{column}
        \begin{column}{0.7\textwidth}
            \only<1>
            {
            \centering
            \includegraphics[width=6cm, height=4cm]{img03/rover.png} 
            }
            \only<2>
            {
            \centering
            \includegraphics[width=6cm, height=4cm]{img03/drone.png} 
            }
        \end{column}
    \end{columns}
\end{frame}

\begin{frame}
    \frametitle{Pixhalk}
    \only<1>
    {
    \centering
    \includegraphics[width=5cm, height=7.5cm]{img03/pixhalk_3.png} 
    }
    \only<2>
    {
    \centering
    \includegraphics[width=9cm, height=6cm]{img03/pixhalk_2.png} 
    }
\end{frame}

\begin{frame}
    \frametitle{Framework Completo}
    \centering
    \includegraphics[width=9cm, height=6cm]{img03/mavros.png} 
\end{frame}

\begin{frame}
    \frametitle{Mission Planner}
    \centering
    \includegraphics[width=9cm, height=6cm]{img03/missionplanner.png} 
\end{frame}

\begin{frame}
    \frametitle{QGroundControl}
    \centering
    \includegraphics[width=9cm, height=6cm]{img03/qgroundcontrol.png} 
\end{frame}

\begin{frame}
    \frametitle{Outros}
    \centering
    \includegraphics[width=9cm, height=4.5cm]{img03/outros.png} 
\end{frame}

\begin{frame}
    \frametitle{Exemplo de Configuração - Robô Real}
    \centering
    \includegraphics[width=9cm, height=4cm]{img03/exemplo_config_1.png} 
\end{frame}

\begin{frame}
    \frametitle{Software In The Loop}
    \centering
    \includegraphics[width=9cm, height=4.5cm]{img03/exemplo_config_2_sitl.png} 
\end{frame}

\section{Simuladores}
\begin{frame}
    \frametitle{Simuladores}
    \scriptsize
    \begin{itemize}
        \item Simuladores são convenientes para o desenvolvimento de aplicações robóticas sem a logística necessária para a operação de robôs móveis. Permitem testar as aplicações em ambientes de qualquer dimensão e diferentes topologias (diferentes maneiras de representar o ambiente).
        \item Simuladores implementam todo o software embarcado nos robôs reais e simulam a cinemática e dinâmica dos robôs. Podem oferecer visualização 2D ou 3D.
    \end{itemize}
    \only<1>
    {
    \begin{table}[ht]
      \centering
      \begin{tabular}{cc}
        \includegraphics[width=0.5\linewidth]{img03/v-rep.png} &
        \includegraphics[width=0.34\linewidth]{img03/mobilesim.png} \\
        (V-Rep) & (MobileSim) \\
      \end{tabular}
    \end{table}
    }
    \only<2>
    {
    \begin{table}[ht]
      \centering
      \begin{tabular}{cc}
        \includegraphics[width=0.5\linewidth]{img03/gazebo_1.png} &
        \includegraphics[width=0.34\linewidth]{img03/gazebo_2.png} \\
        (Gazebo) & (Gazebo) \\
      \end{tabular}
    \end{table}
    }
    \only<3>
    {
        \centering
        \includegraphics[width=0.35\linewidth]{img03/turtlesim.png} &
    }
\end{frame}



%\section{Outras Formas Locomoção}

\begin{frame}
    \frametitle{Outras Formas de Locomoção}
    \centering
    \begin{tabular}{cc}
        \includegraphics[width=0.25\textwidth]{img04/balao.png} &
        \includegraphics[width=0.35\textwidth]{img04/shark.png} \\
        \includegraphics[width=0.30\textwidth]{img04/fish.png} &
        \includegraphics[width=0.55\textwidth]{img04/planar.png} \\    
    \end{tabular}
\end{frame}

\begin{frame}
    \frametitle{Manobrabilidade}
    \begin{columns}
        \begin{column}{0.5\textwidth}
            \centering
            \includegraphics[width=0.9\textwidth]{img04/mecanum_wheels.jpeg}
        \end{column}
        \begin{column}{0.5\textwidth}
            \begin{itemize}
                \scriptsize
                \item \textbf{Manobrabilidade} refere-se a facilidade do robô em executar movimentos (manobras) ao longo do ambiente.
                \item Alguns robôs são \textbf{holonômicos / omnidirecionais}, o que significa que podem se mover a qualquer momento e para qualquer direção, independentemente da sua orientação.
                \item Isso requer uma configuração de rodas \textbf{omnidirecionais}.
            \end{itemize}
        \end{column}
    \end{columns}
\end{frame}

\begin{frame}
    \frametitle{Manobrabilidade}
    \begin{columns}
        \begin{column}{0.5\textwidth}
            \centering
            \includegraphics[width=0.9\textwidth]{img04/ackermann.png}
        \end{column}
        \begin{column}{0.5\textwidth}
            \begin{itemize}
                \scriptsize
                \item Em contraste, considere a configuração conhecida como modelo de \textbf{tração Ackermann} (exemplo: os carros de passeio).
                \item Essa configuração não permite que o robô faça movimentos omnidirecionais. 
                \item Ao contrário, obriga o veículo a se deslocar sobre círculos com raios produzidos pelo esterçamento das rodas, inclusive, círculos estes de raios maiores que o próprio veículo. 
                \item O carro sempre se movimenta sobre círculos, e \textbf{nunca caminha “de lado”, ou seja, sobre o eixo-y}.
            \end{itemize}
        \end{column}
    \end{columns}
\end{frame}


\begin{frame}
    \frametitle{Manobrabilidade}
    \begin{columns}
        \begin{column}{0.5\textwidth}
            \centering
            \begin{mdframed}[%
                            backgroundcolor=blue!20,   % cor de fundo
                            linecolor=red,               % cor da borda
                            linewidth=1pt,               % espessura da borda
                            roundcorner=4pt,             % cantos arredondados
                            innertopmargin=6pt,          % espaço interno acima
                            innerbottommargin=6pt,       % espaço interno abaixo
                            innerleftmargin=6pt,         % espaço interno à esquerda
                            innerrightmargin=6pt         % espaço interno à direita
                            ]
                \scriptsize
                Quanto maior a manobrabilidade de um robô, geralmente, maior é a dificuldade de processamento para gerar sinais de controle para levar o robô às referências desejadas.
            \end{mdframed}
        \end{column}
        \begin{column}{0.5\textwidth}
            \begin{itemize}
                \scriptsize
                \item Controlabilidade refere-se a facilidade em se gerar sinais de controle para um robô de modo a leva-lo a algum resultado.
                \item Há, em geral, uma correlação \textbf{inversa entre a manobrabilidade e a controlabilidade}.
                \item Por exemplo, robôs omnidirecionais com quatro rodas suécas requerem uma grande quantidade de cálculos para transformar o movimento desejado em comandos individuais para cada uma das rodas.
                \item Robôs simples, como o modelo diferencial (P3DX) requerem menor quantidade de cálculos para a obtenção dos comandos individuais de cada roda.
            \end{itemize}
        \end{column}
    \end{columns}
\end{frame}




% (AULA 05) - Sensores
%\section{Bibliografia}


\begin{frame}
	\frametitle{Livros Texto}
	\only<1>
	{
		  \centering
		  \includegraphics[width=4.5cm, height=6cm]{Figuras/RobotProgramming.png}
	}

        \only<2>
	{
		\begin{columns}
			\begin{column}{0.5\textwidth}
				\centering
				\includegraphics[width=4.5cm, height=6cm]{Figuras/ROS_Springer.jpg}
			\end{column} 
			\begin{column}{0.5\textwidth}
				\centering
				\includegraphics[width=4.5cm, height=6cm]{Figuras/Learning.jpg}
			\end{column} 
		\end{columns}
	}	

        \only<3>
	{
		\begin{columns}
			\begin{column}{0.5\textwidth}
				\centering
				\includegraphics[width=4.5cm, height=6cm]{Figuras/LivroPython1.png}
			\end{column} 
			\begin{column}{0.5\textwidth}
				\centering
				\includegraphics[width=4.5cm, height=6cm]{Figuras/LivroPython2.png}
			\end{column} 
		\end{columns}
	}	
\end{frame}

\begin{frame}
	\frametitle{ROS Website (\href{www.ros.org}{www.ros.org})} 
	\only<1>
	{
		\includegraphics[width=9cm, height=6cm]{Figuras/ROS_Website.png} 
	}
	\only<2>
	{
		\includegraphics[width=9cm, height=6cm]{Figuras/ROS_Website_2.png} 
	}
\end{frame}

\section{Motivação}

\begin{frame}
	\frametitle{Ciclo de um Processo em Robótica} 
	\centering

	\begin{tikzpicture}
						[squarednode/.style={rectangle, draw=red!60, fill=red!5, very thick, minimum size = 10mm},]
		\only<1-3>
		{
			\node[squarednode] 									(sense)	{Sensores};
		}
		\only<2-3>
		{	
			\node[squarednode, right of=sense,xshift=2.4cm] 	(think) {Processamento};
		}
		\only<3->
		{	
			\node[squarednode, right of=think,xshift=2.4cm] 	(act)	{Atuação};
		}
		\only<1-3>
		{
			\node[inner sep=0pt, above of=sense, yshift=1cm] 	(fig_sense) 
			{
				\includegraphics[width=.25\textwidth]{Figuras/vision.jpg}
			};
		}
		\only<2-3>
		{		
			\node[inner sep=0pt, above of=think, yshift=1cm] 	(fig_think) 
			{
				\includegraphics[width=.25\textwidth]{Figuras/brain.jpg}
			};
		}
		\only<3->
		{		
			\node[inner sep=0pt, above of=act, yshift=1cm] 		(fig_act) 
			{
				\includegraphics[width=.25\textwidth]{Figuras/act.png}
			};
		}			
		\only<2-3>
		{	
			\draw[->, thick, draw=red!60] (sense.east) -- (think.west) ;
		}
		\only<3->
		{
			\draw[->, thick, draw=red!60] (think.east) -- (act.west) ;
		}
	\end{tikzpicture}
	
\end{frame}

\begin{frame}
	\frametitle{Sensores}
	\centering
	\begin{tikzpicture}
	[vermelho/.style={rectangle, draw=red!60, 	fill=red!5, 	very thick, minimum size = 10mm},
	verde/.style	={rectangle, draw=green!60, fill=green!5, 	very thick, minimum size = 10mm},]
	
	\node[vermelho] 								(sense)		{Sensores};
	\node[verde, right of=sense	 	,xshift=2.4cm] 	(ultrassom) {Ultrassom};
	\node[verde, above of=ultrassom	,yshift=1cm] 	(lasers)	{Laser Scan};
	\node[verde, above of=lasers	,yshift=1cm] 	(camera)	{Câmera};
	\node[verde, below of=ultrassom	,yshift=-1cm] 	(GPS)		{GPS};	
	\node[inner sep=0pt, above of=sense, yshift=1cm] 	(fig_sense) 
	{
		\includegraphics[width=.25\textwidth]{Figuras/vision.jpg}
	};

	\node[inner sep=0pt, right of=camera, xshift=2.4cm] 	(fig_camera) 
	{
		\includegraphics[width=.15\textwidth]{Figuras/camera.jpg}
	};

	\node[inner sep=0pt, right of=lasers, xshift=2.4cm] 	(fig_lasers) 
	{
		\includegraphics[width=.10\textwidth]{Figuras/lidar.jpg}
	};

	\node[inner sep=0pt, right of=ultrassom, xshift=2.4cm] 	(fig_ultrassom) 
	{
		\includegraphics[width=.15\textwidth]{Figuras/ultrassonic.jpg}
	};
	
	\node[inner sep=0pt, right of=GPS, xshift=2.4cm] 	(fig_gps) 
	{
		\includegraphics[width=.10\textwidth]{Figuras/gps.jpg}
	};
	
	\end{tikzpicture}
\end{frame}
%------------------------------------------------


\begin{frame}
	\frametitle{Processamento}
	\centering
	\begin{tikzpicture}
	[vermelho/.style={rectangle, draw=red!60, 	fill=red!5, 	very thick, minimum size = 10mm},
	verde/.style	={rectangle, draw=green!60, fill=green!5, 	very thick, minimum size = 10mm},]
	
	\node[vermelho] 								(think)		{Processamento};
	\node[verde, right of=think	 	,xshift=2.9cm] 	(obi) 		{Otimização Bioinspirada};
	\node[verde, above of=obi		,yshift=1cm] 	(mdl)		{Machine/Deep Learning};
	\node[verde, above of=mdl		,yshift=1cm] 	(ai)		{Inteligência Artificial};
	\node[verde, below of=obi		,yshift=-1cm] 	(signal)	{Processamento de Sinais};	
	\node[inner sep=0pt, above of=think, yshift=1cm] (fig_think) 
	{
		\includegraphics[width=.15\textwidth]{Figuras/brain.jpg}
	};
	
	\node[inner sep=0pt, right of=ai, xshift=2.4cm] 	(fig_ai) 
	{
		\includegraphics[width=.25\textwidth]{Figuras/ai.png}
	};
	
	\node[inner sep=0pt, right of=mdl, xshift=2.4cm] 	(fig_mdl) 
	{
		\includegraphics[width=.15\textwidth]{Figuras/machinelearning.png}
	};
	
	\node[inner sep=0pt, right of=signal, xshift=2.4cm] (fig_signal) 
	{
		\includegraphics[width=.20\textwidth]{Figuras/processamentosinais.png}
	};
	
	\node[inner sep=0pt, right of=obi, xshift=2.4cm] 	(fig_obi) 
	{
		\includegraphics[width=.15\textwidth]{Figuras/gohb.png}
	};
	
	\end{tikzpicture}
\end{frame}
%------------------------------------------------

\begin{frame}
	\frametitle{Processamento}
	{
		\begin{itemize}
			\item Visual Odometry / SLAM / Outros algoritmos de visão
			\item Filtro de Partículas
			\item Algoritmos de Localização
			\item Planejamento de Caminhos e Trajetórias
			\item Controle (PID, Realimentação de estados)
			\item Filtro de Kalman
		\end{itemize}
	}
\end{frame}

\begin{frame}
	\frametitle{Atuadores}
	\centering
	\begin{tikzpicture}
		[vermelho/.style={rectangle, draw=red!60, 	fill=red!5, 	very thick, minimum size = 10mm},
		verde/.style	={rectangle, draw=green!60, fill=green!5, 	very thick, minimum size = 10mm},]
	
		\node[vermelho] 							(act)		{Atuadores};
		\node[verde, right of=act ,xshift=2.9cm] 	(uav) 		{Motores de UAV (Drones)};
		\node[verde, above of=uav ,yshift=1cm] 		(ser)		{Servo Motores};
		\node[verde, above of=ser ,yshift=1cm] 		(auv)		{Motores Aquáticos (AUV)};
		\node[verde, below of=uav ,yshift=-1cm] 	(gas)		{Motores Combustão};	
		\node[inner sep=0pt, above of=think, yshift=1cm] (fig_act) 
		{
			\includegraphics[width=.25\textwidth]{Figuras/act.png}
		};
	
		\node[inner sep=0pt, right of=uav, xshift=2.7cm] 	(fig_uav) 
		{
			\includegraphics[width=.15\textwidth]{Figuras/uav.jpg}
		};
	
		\node[inner sep=0pt, right of=ser, xshift=2.7cm] 	(fig_ser) 
		{
			\includegraphics[width=.15\textwidth]{Figuras/servo.jpg}
		};
	
		\node[inner sep=0pt, right of=auv, xshift=2.7cm] 	(fig_auv) 
		{
			\includegraphics[width=.15\textwidth]{Figuras/auv.jpg}
		};
	
		\node[inner sep=0pt, right of=gas, xshift=2.7cm] 	(fig_gas) 
		{
			\includegraphics[width=.15\textwidth]{Figuras/gas.jpg}
		};
	
	\end{tikzpicture}	
\end{frame}

\begin{frame}
    \frametitle{Motivação para o ROS}
    \begin{columns}
        \only<1-2>
        {
            \begin{column}{0.5\textwidth}
                \centering
                \includegraphics[width=4.3cm, height=6.5cm]{Figuras/Re-Inventing.png}
            \end{column}
        }
        \only<2>
        {
            \begin{column}{0.5\textwidth}
                \href{https://spectrum.ieee.org/automaton/robotics/robotics-software/the-origin-story-of-ros-the-linux-of-robotics}{The Origin Story of ROS, the Linux of Robotics} \\~\\
                \scriptsize
                "The world's most influential robotics software platform"
            \end{column}
        }	
    \end{columns}	
\end{frame}

\begin{frame}
    \frametitle{Motivação para o ROS}
    \centering
    \includegraphics[width=8cm, height=6.5cm]{img02/ciclo_ros.png}
\end{frame}

\section{Objetivos}

\begin{frame}
    \frametitle{Objetivos do Curso}
    \begin{itemize}
        \item Entender o Ecossistema ROS (tópicos, nós, mensagens, serviços)
        \item Desenvolver aplicações para controlar o movimento de um robô
        \item Entender como a posição e orientação são representadas no ROS
        \item Desenvolver programas simples utilizando visão computacional
        \item Utilizar os simuladores (Exemplo: Gazebo)
    \end{itemize}
\end{frame}

\begin{frame}
    \frametitle{\href{https://www.ros.org/}{https://www.ros.org/}}
    \only<1>
    {
        \centering
		\includegraphics[width=10cm, height=6cm]{img02/siteROS1.png}        
    }

    \only<2>
    {
        \centering
        \includegraphics[width=10cm, height=6cm]{img02/siteROS2.png}
    }
\end{frame}

\section{História}
\begin{frame}
	\frametitle{História do ROS}
	\begin{itemize}
		\scriptsize
		\item \href{https://www.linkedin.com/in/eric-berger-806b3b3/}{Eric Berger} e {Keenan Wyrobek} começaram o Doutorado (Ph.D) em Stanford...
		\item Buscaram levantar fundos para o desenvolvimento do projeto o Linux da Robótica.
		\item PR2 (Personal Robotics) - \href{https://www.youtube.com/watch?v=oyHWkQcin7I&feature=youtu.be}{\underline{ver vídeo}}
		\item Atualmente: Ecossistema ROS - Qualquer grupo pode iniciar um repositório de código do ROS ("\textit{federated model}")
	\end{itemize}
			
	\begin{table}
		\begin{tabular}{ | >{\centering\arraybackslash}m{1cm}  >{\centering\arraybackslash}m{7cm} | }  
			\hline
			\includegraphics[width=0.6cm,height=0.6cm]{Figuras/stanford.png} & Stanford Personal Robotics Program (janeiro de 2007)\\
			\hline
			\includegraphics[width=0.6cm,height=0.6cm]{Figuras/willowgarage.jpg} & Laboratório de pesquisa em robótica e incubadora tecnológica (Novembro de 2007) \\
			\hline
			\includegraphics[width=0.6cm,height=0.6cm]{Figuras/openrobotics.png} & Open Source Robotics Foundation (OSRF) ou Open Robotics (Fevereiro de 2013)\\
			\hline
		\end{tabular}
	\end{table}
	
	\centering
	\scriptsize
	"Don’t let anyone crush your crazy" ou "Não deixe ninguém esmagar sua loucura"
\end{frame}


\begin{frame}
	\frametitle{RosCon - ROS Conference (\href{https://roscon.ros.org/2024/} {https://roscon.ros.org/2024/})} 
        \begin{columns}
            \begin{column}{0.5\textwidth}
        	\centering
                \only<1>
                {
    	          \includegraphics[width=4cm,height=6cm]{img02/ROSCon1.png} \\~\\
                }
                \only<2>
                {
    	          \includegraphics[width=6cm,height=3.5cm]{img02/ROSCon2.png} \\~\\
                }
            \end{column}
            \begin{column}{0.5\textwidth}
        	\scriptsize
        	Conjugada com o IROS (ou ICRA):
        	\begin{itemize}
            		\item \href{https://www.iros25.org/}{IEEE/RSJ International Conference on Intelligent Robots and Systems}
            		\item \href{(https://2025.ieee-icra.org/}{International Conference on Robotics and Automation}
            	\end{itemize}
            \end{column}
        \end{columns}
\end{frame}



%\section{Contato}

\begin{frame}
    \frametitle{Sensores de Contato}
    \centering
    \begin{columns}
        \begin{column}{0.5\textwidth}
            \centering
            \begin{tabular}{c c c}
                \includegraphics[width=0.3\textwidth]{img05/contato_1.png} &
                \includegraphics[width=0.3\textwidth]{img05/contato_2.png} &
                \includegraphics[width=0.3\textwidth]{img05/contato_3.png} \\
            \end{tabular}
            \includegraphics[width=0.7\textwidth]{img05/contato_4.png} \\            
        \end{column}
        \begin{column}{0.5\textwidth}
            \begin{itemize}
                \scriptsize
                \item Os sensores táteis, adquirem a informação de contato com o ambiente por meio de interação física.
                \item Podem servir a diversos propósitos, tal qual um para-choque ou então um sensor que identifica a uma garra quando parar, ou em mãos robóticas, informando que o toque ocorreu.
                \item Os mais comuns empregam chaves normalmente abertas e fechadas, efeitos piezelétricos e piezoresistivos, capacitivos e assim por diante.
            \end{itemize}
        \end{column}    
    \end{columns}
\end{frame}

\section{Encoder}
\begin{frame}
    \frametitle{Sensor de movimento: Encoder}
    \centering
    \begin{columns}
        \begin{column}{0.5\textwidth}
            \centering
            \begin{tabular}{c c}
                \includegraphics[width=0.4\textwidth]{img05/encoder_1.png} &
                \includegraphics[width=0.4\textwidth]{img05/encoder_2.png} \\
            \end{tabular}
            \includegraphics[width=0.5\textwidth]{img05/lente_fotografica.png} \\            
        \end{column}
        \begin{column}{0.5\textwidth}
            \begin{itemize}
                \scriptsize
                \item É um dispositivo eletromecânico que conta ou reproduz pulsos elétricos a partir do movimento rotacional de seu eixo.
                \item Pode ser definido como um transdutor de posição angular.       
                \item \textbf{Exemplo de utilização de encoders em robótica}: medir a posição ou velocidade de motores utilizados para mover robôs.
                \item Se os valores dos encoders forem integrados (utilize um integrador digital), proporcionarão uma estimação da posição do robô. A posição do robô é chamada de \textbf{hodometria}.
            \end{itemize}
        \end{column}    
    \end{columns}
\end{frame}

\begin{frame}
    \frametitle{Sensor de movimento: Encoder}
    \begin{columns}
        \begin{column}{0.5\textwidth}
            \centering
            \scriptsize
            \textbf{Funcionamento regular}: o aparato conta o número de incrementos/transições (1 para sinal elevado – houve uma transição – e 0 para sinal baixo – ainda não houve uma transição) mas não pode dizer a direção do movimento.
        \end{column}
        \begin{column}{0.5\textwidth}
            \begin{itemize}
                \scriptsize
                \item Os encoders são sensores \textbf{proprioceptivos}, pois medem estados internos do robô (no caso, posição e velocidade).
                \item Dependendo do fabricante, encontra-se de diversas resoluções, desde 64 a mais de 50.000 incrementos (“ticks”, pulsos, contagens) por cada revolução do eixo do motor.
                \item Para maiores resoluções pode-se utilizar interpolação.
            \end{itemize}
        \end{column}
    \end{columns}
    \centering
    \includegraphics[width=0.7\textwidth]{img05/encoder_3.png} \\            
\end{frame}

\begin{frame}
    \frametitle{Sensor de movimento: Encoder}
    \begin{columns}
        \begin{column}{0.5\textwidth}
            \centering
            \scriptsize
            \textbf{Funcionamento em quadratura}: utiliza-se dois sensores deslocados entre si em fase de 90º (quadratura, um elevado e outro baixo). A ordem de qual sensor produz sinal elevado (1) primeiro diz a direção do movimento. Além da direção, a resolução neste modo de funcionamento é 4 vezes maior.
        \end{column}
        \begin{column}{0.5\textwidth}
            \centering
            \includegraphics[width=0.5\textwidth]{img05/encoder_4.png} \\            
        \end{column}
    \end{columns}
    \begin{columns}
        \begin{column}{0.5\textwidth}
            \centering
            \includegraphics[width=0.7\textwidth]{img05/encoder_6.png} \\            
        \end{column}
        \begin{column}{0.5\textwidth}
            \centering
            \includegraphics[width=0.7\textwidth]{img05/encoder_5.png} \\            
        \end{column}
    \end{columns}
\end{frame}

\section{Giros, Acelerômetros, IMUs}
\begin{frame}
    \frametitle{Sensor de orientação: Giroscópio}
    \centering
    \begin{columns}
        \begin{column}{0.5\textwidth}
            \centering
            \begin{tabular}{c}
                \includegraphics[width=0.8\textwidth]{img05/giroscópio.png} \\
            \end{tabular}
        \end{column}
        \begin{column}{0.5\textwidth}
            \begin{itemize}
                \scriptsize
                \item Os \textbf{ópticos} utilizam lasers emitidos em duas direções, uma viajando no sentido horário e outra no sentido anti-horário.
                \item Quando o aparato gira, as fases relativas dos lasers são deslocadas de acordo com a velocidade angular do movimento, uma vez que para um dos lasers o caminho ficou mais longo, e para o outro mais curto (efeito Sagnac).
                \item Mede-se a diferença de fase entre os dois feixes, e essa diferença entre as fases dos feixes é proporcional à velocidade angular do corpo e, consequentemente, ao ângulo do aparato girante.
                \item Ring Laser Gyroscope, RLG, que utiliza o efeito Sagnac
            \end{itemize}
        \end{column}    
    \end{columns}
\end{frame}

\begin{frame}
    \frametitle{Sensor de Inércia: Acelerômetros}
    \begin{columns}
        \begin{column}{0.4\textwidth}
            \centering
            \includegraphics[width=0.7\textwidth]{img05/massa_mola_amortecedor.png}
            \begin{mdframed}[%
                            backgroundcolor=blue!20,   % cor de fundo
                            linecolor=red,               % cor da borda
                            linewidth=1pt,               % espessura da borda
                            roundcorner=4pt,             % cantos arredondados
                            innertopmargin=6pt,          % espaço interno acima
                            innerbottommargin=6pt,       % espaço interno abaixo
                            innerleftmargin=6pt,         % espaço interno à esquerda
                            innerrightmargin=6pt         % espaço interno à direita
                            ]
                \scriptsize
                \begin{itemize}
                    \scriptsize
                    \item Na superfície da Terra, o acelerômetro sempre indicará, pelo menos, 1g ao longo do eixo vertical, e 0g em queda-livre.
                    \item Por isso, para obter a aceleração inercial (correspondente ao movimento do corpo), é necessário subtrair a gravidade.
                \end{itemize}
            \end{mdframed}
        \end{column}
        \begin{column}{0.6\textwidth}
            \centering
            \begin{itemize}
                \scriptsize
                \item Conforme os princípios físicos, para alterar movimentos de corpos é necessário imprimir forças sobre eles.
                \item Os acelerômetros são sensores que medem todas as forças que estejam atuando sobre eles, incluindo a gravidade.
                \item Os acelerômetros atuam como sistemas massa-mola-amortecedor, equacionado como:
            \end{itemize}
            \begin{equation*}
                F_{\text{aplicada}} = F_{\text{inercial}} + F_{\text{amortecida}} + F_{\text{elástica}} 
            \end{equation*}
            \begin{equation*}
                F_{\text{aplicada}} = m\ddot{x} + c\dot{x} + kx
            \end{equation*}
            \begin{itemize}
                \scriptsize
                \item[] Onde:
                \item \textbf{m} é a massa de prova.
                \item \textbf{c} é o coeficiente de amortecimento.
                \item \textbf{k} é a constante da mola.
            \end{itemize}
        \end{column}
    \end{columns}
\end{frame}

\begin{frame}
    \frametitle{Unidades de Medidas Inerciais (IMU)}
    \centering
    \begin{columns}
        \begin{column}{0.5\textwidth}
            \centering
            \begin{tabular}{c}
                \includegraphics[width=0.9\textwidth]{img05/IMU.png} \\
            \end{tabular}
        \end{column}
        \begin{column}{0.5\textwidth}
            \begin{itemize}
                \scriptsize
                \item As \textbf{Inertial Measurement Units (IMUs)} são dispositivos que medem posições relativas (x, y, z), orientações (roll, pitch, yaw), velocidades e acelerações de um corpo em movimento.
                \item Combinam as funções dos acelerômetros, giroscópios, bússolas além de possuírem integradores para o cálculo de hodometria.
                \item São unidades inerciais completas, sensíveis a erros, corrigindo a hodometria simples (dead reckoning) provida pelos encoders.
            \end{itemize}
        \end{column}    
    \end{columns}
\end{frame}

\section{Câmeras}
\begin{frame}
    \frametitle{Câmeras}
    \centering
    \begin{columns}
        \begin{column}{0.5\textwidth}
            \centering
            \begin{tabular}{c}
                \includegraphics[width=0.7\textwidth]{img05/câmera_2.png} \\
                \includegraphics[width=0.7\textwidth]{img05/câmera_1.png} \\
            \end{tabular}
        \end{column}
        \begin{column}{0.5\textwidth}
            \begin{itemize}
                \scriptsize
                \item As câmeras são sensores bastante populares, que podem ser utilizados das mais variadas formas.
                \item A área de pesquisa em imagens é denominada de visão computacional, em que a visão robótica é uma das áreas.
                \item Por meio de técnicas de manipulação de imagens, é possível obter distâncias, características (features), identificar objetos e pessoas, obter caminhos, dentre muitas outras aplicações.
            \end{itemize}
        \end{column}    
    \end{columns}
\end{frame}

\begin{frame}
    \frametitle{Câmeras - Knectic}
    \centering
    \includegraphics[width=0.95\textwidth]{img05/kinect.png}
\end{frame}

\begin{frame}{O que é a Câmera Kinect?}
    \begin{itemize}
        \item Sensor desenvolvido originalmente pela Microsoft para o Xbox.
        \item Usa visão computacional para captar profundidade, cor e movimento.
        \item Combina:
        \begin{itemize}
            \item Câmera RGB (imagem tradicional)
            \item Sensor de profundidade (depth camera)
            \item Microfones e acelerômetros (em versões completas)
        \end{itemize}
    \end{itemize}
    \end{frame}

\begin{frame}{Câmera Kinect: Como Funciona?}
    \begin{itemize}
        \item A câmera de profundidade projeta um padrão infravermelho (IR).
        \item O padrão refletido é captado por um sensor e processado para estimar a distância de cada ponto.
        \item Resulta em uma \textbf{imagem de profundidade} (depth map), com cada pixel representando uma distância.
    \end{itemize}
    
    \vspace{0.5cm}
    \textbf{Alternativamente (em modelos mais novos):} usa tecnologia de \textit{Time-of-Flight (ToF)} para medir o tempo que a luz leva para voltar ao sensor.
\end{frame}

\begin{frame}{Câmera Kinect: Vantagens}
    \begin{itemize}
        \item Captura simultânea de imagem colorida e profundidade.
        \item Excelente para rastreamento de corpo humano e gestos.
        \item Pronto para uso com SDKs acessíveis.
        \item Operação em tempo real com boa precisão.
    \end{itemize}
\end{frame}

\begin{frame}{Câmera Kinect: Aplicações}
    \begin{itemize}
        \item Jogos e interfaces baseadas em gestos.
        \item Robótica (percepção 3D e navegação).
        \item Mapeamento de ambientes (SLAM).
        \item Realidade aumentada e captura de movimento.
        \item Reconhecimento de pessoas e gestos em segurança.
    \end{itemize}
\end{frame}

\begin{frame}{O que são Câmeras Baseadas em Eventos?}
    \begin{itemize}
        \item Diferente das câmeras tradicionais, que capturam imagens em quadros fixos.
        \item Cada pixel funciona de forma independente e assíncrona.
        \item Gera eventos apenas quando há variação significativa de luminosidade.
    \end{itemize}
    
    \vspace{0.5cm}
    \textbf{Cada evento contém:}
    \begin{itemize}
        \item Posição do pixel $(x, y)$
        \item Tempo do evento (alta precisão temporal)
        \item Polaridade da variação (aumento ou redução de brilho)
    \end{itemize}
    \end{frame}

\begin{frame}{Câmera Baseada em Eventos: Vantagens}
    \begin{itemize}
        \item Altíssima resolução temporal (microsegundos)
        \item Baixa latência e baixo consumo de energia
        \item Alta faixa dinâmica (HDR) – ideal para ambientes com iluminação extrema
        \item Excelente desempenho com objetos rápidos
    \end{itemize}
\end{frame}

\begin{frame}{Câmera Baseada em Eventos: Aplicações}
    \begin{itemize}
        \item Robótica e drones autônomos
        \item Carros autônomos
        \item Sistemas biomiméticos (inspirados na retina)
        \item Visão computacional de alta velocidade
    \end{itemize}
    \href{https://youtu.be/0wGBpgIrd9M?si=j-tcaOsuDFyUieXV}{Link: Davide Scaramuzza }
\end{frame}

\section{GPS}
\begin{frame}
    \frametitle{Global Positioning System (GPS)}
    \centering
    \begin{columns}
        \begin{column}{0.4\textwidth}
            \centering
            \begin{tabular}{c}
                \includegraphics[width=0.8\textwidth]{img05/gps.png}\\
            \end{tabular}
        \end{column}
        \begin{column}{0.6\textwidth}
            \begin{itemize}
                \scriptsize
                \item O Sistema de Posicionamento Global é um sistema de localização global baseado em satélites e trilateração.
                \item Desenvolvido originalmente pelos USA nas décadas de 1950 e 1960 para estratégias exclusivamente militares, foi liberado  uso pela população civil na metade da década de 1990.
                \item Outros países desenvolveram seus próprios sistemas baseados em satélites, como o COMPASS (ou, BeiDou) da China, operacional a partir da década de 2000, o GLONASS russo, contemporâneo ao GPS, e o GALILEO da União Europeia e o NAVIC da Índia.
                \item O projeto original do GPS contava com 24 satélites em órbitas de 12 horas a uma altura de, aproximadamente, 21000 km acima da superfície terrestre, já no meio interplanetário.
            \end{itemize}
        \end{column}    
    \end{columns}
\end{frame}

\begin{frame}{Sistemas de Posicionamento e Segurança Cibernética}
    \textbf{Contexto:}
    \begin{itemize}
        \item O GPS foi desenvolvido pelos EUA com objetivos militares e só posteriormente liberado ao uso civil.
        \item A dependência exclusiva de um sistema estrangeiro representa um risco estratégico e cibernético.
        \item Outros países criaram seus próprios sistemas (GLONASS - Rússia, BeiDou - China, GALILEO - UE, NAVIC - Índia) para garantir autonomia.
    \end{itemize}
    
    \vspace{0.5cm}
    \textbf{Riscos associados à dependência do GPS:}
    \begin{itemize}
        \item Sinal pode ser bloqueado, degradado ou falsificado (\textit{spoofing}) em tempos de crise.
        \item Impacto direto em setores críticos: telecomunicações, transporte, energia e defesa.
        \item Falta de controle sobre atualizações, segurança e integridade dos dados recebidos.
    \end{itemize}
\end{frame}

\begin{frame}{Independência Tecnológica e Geopolítica}
    \begin{itemize}
        \item O domínio de sistemas de navegação por satélite está ligado à soberania nacional e geopolítica.
        \item Ter um sistema próprio ou acesso a múltiplos sistemas reduz vulnerabilidades.
        \item Integração com estratégias de defesa cibernética e proteção de infraestrutura crítica.
        \item Investimentos em sistemas nacionais impulsionam inovação e capacitação tecnológica interna.
    \end{itemize}
    
    \vspace{0.3cm}
    \textbf{Conclusão:} Em um mundo cada vez mais digital e conectado, o controle autônomo sobre sistemas espaciais é um ativo estratégico fundamental.
\end{frame}

\begin{frame}
    \frametitle{Global Positioning System (GPS)}
    \centering
    \begin{columns}
        \begin{column}{0.4\textwidth}
            \centering
            \begin{tabular}{c}
                \includegraphics[width=0.8\textwidth]{img05/satelite_gps.png}\\
            \end{tabular}
        \end{column}
        \begin{column}{0.6\textwidth}
            \begin{itemize}
                \scriptsize
                \item Os satélites de GPS se chamam NAVSTAR (NAVigation Satellite with Time And Ranging).
                \item Cada satélite possui um relógio atômico para sincronia de posicionamento, corrige o efeito Doppler na transmissão de seus dados por meio da Teoria da Relatividade.
                \item As órbitas desses satélites são tais que, a todo momento no horizonte de um usuário, existam, pelo menos, quatro satélites disponíveis.
            \end{itemize}
        \end{column}    
    \end{columns}
\end{frame}

\begin{frame}
    \frametitle{Global Positioning System (GPS)}
    \begin{itemize}
        \scriptsize
        \item A localização do GPS é feita por meio do princípio matemático da trilateração. No caso, trilateração em 3D.
        \item Existem estações GPS bem estabelecidas por todo o planeta.
    	\item A trilateração em 2D é facilmente explicada da seguinte maneira: 
    \end{itemize}
    \only<1>
    {
        \begin{itemize}
            \scriptsize
            \item[] Uma fonte (ex: satélite) lhe revela que está a uma distância d1 de uma estação conhecida (ex: numa cidade) denominada A.
        \end{itemize}
        \centering                
        \includegraphics[width=0.8\textwidth]{img05/gps_1.png}\\
    }
    \only<2>
    {
        \begin{itemize}
            \scriptsize
            \item[] Uma segunda fonte (outro satélite) informa que o usuário está a uma distância d2 de um outro ponto conhecido B.
        \end{itemize}
        \centering                
        \includegraphics[width=0.9\textwidth]{img05/gps_2.png}\\
    }
    \only<3>
    {
        \begin{itemize}
            \scriptsize
            \item[] Para eliminar a ambiguidade, é necessário obter a informação de uma terceira fonte (satélite), de uma distância d3 de C.
        \end{itemize}
        \centering                
        \includegraphics[width=0.9\textwidth]{img05/gps_3.png}\\
    }
    \only<4>
    {
        \centering                
        \includegraphics[width=0.95\textwidth]{img05/gps_4.png}\\
    }
\end{frame}

%\section{Arquiteturas}

\begin{frame}
    \frametitle{P3-DX}
    \only<1>
    {
    \centering
    \includegraphics[width=9cm, height=4cm]{img03/P3-DX-Aria-1.png} 
    }
    \only<2>
    {
    \centering
    \includegraphics[width=9cm, height=4cm]{img03/P3-DX-Aria-2.png} 
    }
\end{frame}

\begin{frame}
    \frametitle{Pixhalk}
    \begin{columns}
        \begin{column}{0.3\textwidth}
            \centering
            \includegraphics[width=4cm, height=3cm]{img03/cube_orange.jpg} 
        \end{column}
        \begin{column}{0.7\textwidth}
            \only<1>
            {
            \centering
            \includegraphics[width=6cm, height=4cm]{img03/rover.png} 
            }
            \only<2>
            {
            \centering
            \includegraphics[width=6cm, height=4cm]{img03/drone.png} 
            }
        \end{column}
    \end{columns}
\end{frame}

\begin{frame}
    \frametitle{Pixhalk}
    \only<1>
    {
    \centering
    \includegraphics[width=5cm, height=7.5cm]{img03/pixhalk_3.png} 
    }
    \only<2>
    {
    \centering
    \includegraphics[width=9cm, height=6cm]{img03/pixhalk_2.png} 
    }
\end{frame}

\begin{frame}
    \frametitle{Framework Completo}
    \centering
    \includegraphics[width=9cm, height=6cm]{img03/mavros.png} 
\end{frame}

\begin{frame}
    \frametitle{Mission Planner}
    \centering
    \includegraphics[width=9cm, height=6cm]{img03/missionplanner.png} 
\end{frame}

\begin{frame}
    \frametitle{QGroundControl}
    \centering
    \includegraphics[width=9cm, height=6cm]{img03/qgroundcontrol.png} 
\end{frame}

\begin{frame}
    \frametitle{Outros}
    \centering
    \includegraphics[width=9cm, height=4.5cm]{img03/outros.png} 
\end{frame}

\begin{frame}
    \frametitle{Exemplo de Configuração - Robô Real}
    \centering
    \includegraphics[width=9cm, height=4cm]{img03/exemplo_config_1.png} 
\end{frame}

\begin{frame}
    \frametitle{Software In The Loop}
    \centering
    \includegraphics[width=9cm, height=4.5cm]{img03/exemplo_config_2_sitl.png} 
\end{frame}

\section{Simuladores}
\begin{frame}
    \frametitle{Simuladores}
    \scriptsize
    \begin{itemize}
        \item Simuladores são convenientes para o desenvolvimento de aplicações robóticas sem a logística necessária para a operação de robôs móveis. Permitem testar as aplicações em ambientes de qualquer dimensão e diferentes topologias (diferentes maneiras de representar o ambiente).
        \item Simuladores implementam todo o software embarcado nos robôs reais e simulam a cinemática e dinâmica dos robôs. Podem oferecer visualização 2D ou 3D.
    \end{itemize}
    \only<1>
    {
    \begin{table}[ht]
      \centering
      \begin{tabular}{cc}
        \includegraphics[width=0.5\linewidth]{img03/v-rep.png} &
        \includegraphics[width=0.34\linewidth]{img03/mobilesim.png} \\
        (V-Rep) & (MobileSim) \\
      \end{tabular}
    \end{table}
    }
    \only<2>
    {
    \begin{table}[ht]
      \centering
      \begin{tabular}{cc}
        \includegraphics[width=0.5\linewidth]{img03/gazebo_1.png} &
        \includegraphics[width=0.34\linewidth]{img03/gazebo_2.png} \\
        (Gazebo) & (Gazebo) \\
      \end{tabular}
    \end{table}
    }
    \only<3>
    {
        \centering
        \includegraphics[width=0.35\linewidth]{img03/turtlesim.png} &
    }
\end{frame}




% (TRABALHOS)
%\section{Introdução}

\begin{frame}{O que é o YOLO?}
    \begin{itemize}
        \item YOLO (You Only Look Once) é um algoritmo de \textbf{detecção de objetos em tempo real}.
        \item Foi proposto por Joseph Redmon em 2016.
        \item Diferente de abordagens anteriores, o YOLO realiza \textbf{detecção em uma única etapa}, tratando como um problema de regressão.
        \item Detecta múltiplos objetos e suas classes em uma única imagem de forma eficiente.
    \end{itemize}
\end{frame}

\begin{frame}{Como o YOLO funciona?}
    \begin{itemize}
        \item A imagem de entrada é dividida em uma grade (ex: 7x7).
        \item Cada célula prevê:
            \begin{itemize}
                \item Bounding Boxes (caixas delimitadoras)
                \item Confiança da detecção
                \item Classes dos objetos
            \end{itemize}
            \item Toda a inferência é feita com uma única rede neural convolucional.
    \end{itemize}
\end{frame}

\begin{frame}{Arquitetura do YOLO}
    \centering
    \includegraphics[width=0.4\linewidth]{img02/yolo.png}
    \vspace{0.2cm}
    
    {\scriptsize Fonte: Redmon et al.}
\end{frame}
    
\begin{frame}{Evolução do YOLO}
    \begin{itemize}
        \item \textbf{YOLOv1 (2016)} – Primeira versão, rápida, mas com baixa precisão para objetos pequenos.
        \item \textbf{YOLOv2 (YOLO9000)} – Mais acurada e com suporte a múltiplas classes.
        \item \textbf{YOLOv3, v4, v5...} – Melhorias progressivas em velocidade e precisão.
        \item \textbf{YOLOv8 (Ultralytics)} – Implementação moderna com suporte a PyTorch e ONNX.
    \end{itemize}
\end{frame}

\begin{frame}{Aplicações do YOLO}
    \begin{itemize}
        \item Carros autônomos
        \item Câmeras de vigilância
        \item Drones
        \item Robôs móveis (como os da disciplina!)
        \item Contagem de objetos e rastreamento em tempo real
    \end{itemize}
\end{frame}

\begin{frame}{Exemplo de Detecção com YOLO}
    \centering
    \includegraphics[width=0.9\linewidth]{img02/detecao.png} \\~
    \vspace{0.2cm}
    {\scriptsize Imagem ilustrativa com caixas delimitadoras (bounding boxes)}
    \end{frame}

\begin{frame}{Vantagens e Limitações}
    \begin{columns}
        \column{0.5\textwidth}
            \textbf{Vantagens:}
            \begin{itemize}
                \item Muito rápido (tempo real)
                \item Rede única e simples de implementar
                \item Detecção multiobjeto eficiente
            \end{itemize}
            
        \column{0.5\textwidth}
            \textbf{Limitações:}
            \begin{itemize}
                \item Menor precisão para objetos pequenos
                \item Trade-off entre velocidade e acurácia
                \item Versões antigas não suportam bem múltiplas escalas
            \end{itemize}
            \end{columns}
    \end{frame}


\begin{frame}{Exemplo Prático com YOLO}
    \textbf{Objetivo:} Criar e treinar um modelo YOLO com suas próprias imagens.
    
    \begin{enumerate}
        \item Capturar ou coletar imagens (ex: celular, Google Images)
        \item Anotar os objetos com \textbf{LabelImg}
        \item Treinar o modelo YOLO com Ultralytics
        \item Testar com novas imagens
    \end{enumerate}
    
    {\small É possível usar Google Colab para evitar instalação local.}
\end{frame}

\section{Etiquetagem}

\begin{frame}{Ferramenta de Etiquetagem: LabelImg}
    \begin{itemize}
        \item Programa gratuito para anotar objetos manualmente em imagens.
        \item Gere arquivos `.xml` (formato PASCAL VOC) com os \textbf{bounding boxes}.
        \item Cada imagem `.png` ou `.jpg` terá um `.xml` com as coordenadas dos objetos.
    \end{itemize}
    
    \begin{center}
        \includegraphics[width=0.4\linewidth]{img02/labelimg_exemplo.png}
    \end{center}
\end{frame}

\begin{frame}[fragile]{Instalação do LabelImg}
    \textbf{No terminal (Linux, macOS ou Windows com WSL):}

    \begin{lstlisting}
    git clone https://github.com/tzutalin/labelImg.git
    cd labelImg
    pip install pyqt5 lxml
    python labelImg.py
    \end{lstlisting}
    
    \vspace{0.3cm}
    \textbf{Após abrir:}
    \begin{itemize}
        \item Escolha a pasta com as imagens.
        \item Clique em "Create RectBox" e salve os arquivos `.xml`.
    \end{itemize}
\end{frame}

\begin{frame}[fragile]{Estrutura de Diretório para Treinamento}
    Organize assim:
    \begin{lstlisting}
    /dataset
      ├── images
      │   ├── image1.png
      │   ├── image2.png
      └── labels
          ├── image1.xml
          ├── image2.xml
    \end{lstlisting}
    
    Você pode converter para YOLO format com o Roboflow ou `xml\_to\_yolo.py`.
\end{frame}

\section{Treinamento}
\begin{frame}[fragile]{Treinando com YOLOv8 no Google Colab}
    Instale a biblioteca Ultralytics:
    
    \begin{lstlisting}
    !pip install ultralytics
    \end{lstlisting}

    Treine o modelo com seus dados:
    \begin{lstlisting}
    from ultralytics import YOLO
    
    model = YOLO('yolov8n.pt')  # Modelo base
    model.train(data='data.yaml', epochs=50)
    \end{lstlisting}
    
    Você deve criar um `data.yaml` descrevendo:
    \begin{itemize}
        \item Caminhos para imagens de treino/teste
        \item Lista de classes
    \end{itemize}
\end{frame}

\begin{frame}[fragile]{Exemplo de Arquivo data.yaml}
    \begin{lstlisting}
    path: /content/dataset
    train: images/train
    val: images/val
    
    names:
      0: carro
      1: pessoa
    \end{lstlisting}
    
    \begin{itemize}
        \item Use Roboflow para criar isso automaticamente
        \item Ou escreva manualmente com base na sua estrutura
    \end{itemize}
    \end{frame}

\section{Teste}
\begin{frame}[fragile]{Testando seu modelo}
    Depois de treinado:
    \begin{lstlisting}
    results = model('caminho/para/imagem.jpg')
    results.show()
    \end{lstlisting}
    
    Você pode salvar a imagem com as detecções:
    \begin{lstlisting}
    results.save(filename='saida.jpg')
    \end{lstlisting}
    
    Também é possível rodar em tempo real com webcam!
\end{frame}



% Slide final

\begin{frame}

    \centering

    \begin{tcolorbox}[colback=blue!10, colframe=blue!50, width=0.6\textwidth]
        \centering
        \includegraphics[width=0.7\textwidth]{img/theend.png} \\
    \end{tcolorbox}

    \vspace{0.5cm} % Espaço entre as seções

    \begin{tcolorbox}[colback=green!10, colframe=green!50, width=0.7\textwidth]
        \centering
        \includegraphics[width=0.3\textwidth]{img/e-mail.png} \\
        \begin{center}
            {Muito Obrigado !!!} \\~ 
            {\href{amarcato@ieee.org}{amarcato@ieee.org}} \\~
        \end{center}
    \end{tcolorbox}

\end{frame}

\end{document}
