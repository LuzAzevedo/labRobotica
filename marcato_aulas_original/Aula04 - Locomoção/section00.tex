\section{Introdução}

\begin{frame}
    \frametitle{Locomoção}
    \centering
    \includegraphics[width=4.5cm, height=7cm]{img04/locomocao.png}      

\end{frame}

\begin{frame}
    \frametitle{Estratégias de Locomoção}
    \begin{columns}
        \begin{column}{0.5\textwidth}
            \centering
            \includegraphics[width=0.96\textwidth]{img04/estrategias.png}      
        \end{column}
        \begin{column}{0.5\textwidth}
            \begin{itemize}
                \scriptsize
                \item Comportamentos baseados na vida são denominados comportamentos “bio-inspirados”.
                \item Diversos grupos de robótica, ao redor do mundo, pesquisam maneiras de reproduzir os movimentos encontrados na natureza, aprimorando a locomoção robótica.
                \item Movimentos bioinspirados são bem adaptados para os principais desafios apresentados na natureza
            \end{itemize}        
        \end{column}
    \end{columns}
\end{frame}

\begin{frame}
    \frametitle{Estratégias de Locomoção}
    \centering
    \begin{tabular}{cc}
        \includegraphics[width=0.30\textwidth]{img04/Estrat_Locom_1.png} &
        \includegraphics[width=0.30\textwidth]{img04/Estrat_Locom_2.png} \\
        \includegraphics[width=0.30\textwidth]{img04/Estrat_Locom_3.png} &
        \includegraphics[width=0.30\textwidth]{img04/Estrat_Locom_4.png} \\
    \end{tabular}
\end{frame}

\begin{frame}
    \frametitle{Estratégias de Locomoção}
    \begin{itemize}
        \scriptsize
        \item Fluid Robot. \href{https://youtu.be/mhZdz1fnIEk?si=537ae7olxscC5Dcj}{\color{blue}{\underline{Link}}}.
        \item Snake like robot. \href{https://youtu.be/L9j52vjhb18?si=fDwL4c6y0JZ-Eyed}{\color{blue}{\underline{Link}}}.
        \item Crawling (ou rastejante) like robot. \href{https://youtube.com/shorts/siIaGoqD7VA?si=ZwU_DvpzKdPrydRo}{\color{blue}{\underline{Link}}}.
        \item Inchworm-inspired robot (ou lagarta). \href{https://youtu.be/WOLZTc8ptkA?si=W-UZbqxYb_0OA8U5}{\color{blue}{\underline{Link}}}.
        \item Soft autonomous earthworm robot at MIT (ou minhoca). \href{https://youtu.be/EXkf62qGFII?si=VyVwbgoziWKqhQFx}{\color{blue}{\underline{Link}}}.
        \item Single-Actuator Soft Robot for In-Pipe Crawling. \href{https://youtu.be/4KFsPYf6cxQ?si=eNmj3wTcXURfRljC}{\color{blue}{\underline{Link}}}.
        \item Boston dynamic robots. \href{https://youtu.be/-e1_QhJ1EhQ?si=z-mhZv8FGFha3oZT}{\color{blue}{\underline{Link}}}.
        \item Tennibot, The World's First Robotic Tennis Ball Collector. \href{https://youtu.be/5NLjgErJD-k?si=NAEob11gczltARKe}{\color{blue}{\underline{Link}}}.   
        \item Boston dynamic. Spot's Got an Arm! \href{https://youtu.be/6Zbhvaac68Y?si=WYzHNf3IKpHaTYnF}{\color{blue}{\underline{Link}}}.  
        \item Mecanum Wheels. \href{https://youtu.be/noqBUEgyQ8A?si=KwlOjOKvQ-tF3qrf}{\color{blue}{\underline{Link}}}.  
    \end{itemize}        
\end{frame}

\begin{frame}
    \frametitle{Rodas}
    \begin{columns}
        \begin{column}{0.5\textwidth}
            \centering
            \begin{mdframed}[%
                            backgroundcolor=blue!20,   % cor de fundo
                            linecolor=red,               % cor da borda
                            linewidth=1pt,               % espessura da borda
                            roundcorner=4pt,             % cantos arredondados
                            innertopmargin=6pt,          % espaço interno acima
                            innerbottommargin=6pt,       % espaço interno abaixo
                            innerleftmargin=6pt,         % espaço interno à esquerda
                            innerrightmargin=6pt         % espaço interno à direita
                            ]
                \scriptsize
                Não são encontradas na natureza de maneira literal, mas podem ser aproximadas ao caminhar humano, ou seja, a rolagem de um polígono se aproxima da roda.
            \end{mdframed}
            \includegraphics[width=0.6\textwidth]{img04/poligono_roda.png}  
        \end{column}
        \begin{column}{0.5\textwidth}
            \begin{itemize}
                \item Há, no entanto, uma \textbf{exceção} nos tipos de locomoção: \textbf{rodas}.
                \item Rodas são uma invenção humana, e são extremamente eficientes em solo plano e compactado (firme, não arenoso).
            \end{itemize}
            \centering\
            \includegraphics[width=0.65\textwidth]{img04/darpa.png}  
        \end{column}
    \end{columns}
\end{frame}

\begin{frame}
    \frametitle{Rodas}
    \begin{itemize}
        \item A maioria dos sistemas de locomoção terrestres hoje em dia utiliza rodas ou esteiras.
    \end{itemize}    
    \centering
    \begin{tabular}{cc}
        \includegraphics[width=0.25\textwidth]{img04/rodas_1.png} &
        \includegraphics[width=0.25\textwidth]{img04/rodas_2.png} \\
        \includegraphics[width=0.25\textwidth]{img04/rodas_3.png} &
        \includegraphics[width=0.25\textwidth]{img04/rodas_4.png} \\
    \end{tabular}    
\end{frame}

\begin{frame}
    \frametitle{Energia Locomoção}
    \begin{columns}
        \begin{column}{0.5\textwidth}
            \centering
            \includegraphics[width=0.80\textwidth]{img04/energia_locom.png}
        \end{column}
        \begin{column}{0.5\textwidth}
            \only<1>
            {
                \begin{itemize}
                    \scriptsize
                    \item Energia Gasta por Tonelada (cv/ton):
                    \begin{itemize}
                        \scriptsize
                        \item Essa métrica representa a quantidade de energia necessária para mover uma unidade de massa (tonelada) por unidade de distância. 
                        \item No contexto de robótica móvel, otimizar a energia gasta é essencial para aumentar a eficiência e autonomia dos robôs, especialmente para robôs terrestres que operam em diferentes terrenos.
                    \end{itemize}
                    \item Em robótica, escolher o modo de locomoção correto para cada ambiente é essencial para economizar energia e melhorar o desempenho.
                \end{itemize}
            }
            \only<2>
            {
                \begin{itemize}
                    \scriptsize
                    \item De fato, a eficiência dos movimentos por rodas depende intensamente do tipo de solo, em especial da dureza/firmeza e regularidade do mesmo.
                    \item Em solos irregulares, ou, com baixa compactação (exemplo: areia), rodas terão um desempenho pobre, gastando muita energia para deslocarem-se, caso consigam.
                    \item Em solos nessas características, locomoções por meio de pernas tendem a ter maior eficiência em vencer os obstáculos.
                \end{itemize}
            }
            \only<3>
            {
                \begin{itemize}
                    \scriptsize
                    \item No entanto, o controle e coordenação de pernas exige grande esforço computacional, com difícil modelagem, além do alto gasto energético.
                    \item É necessário planejar com grande cautela o tipo de locomoção de um robô móvel, uma vez que a facilidade do controle ou duração das baterias terão influência direta nessas escolhas.
                \end{itemize}
            }
        \end{column}
    \end{columns}
\end{frame}

\begin{frame}
    \frametitle{Características do Movimento}
    \begin{columns}
        \begin{column}{0.4\textwidth}
            \centering
            \includegraphics[width=0.90\textwidth]{img04/CaracteristicasMovimento.png}
        \end{column}
        \begin{column}{0.6\textwidth}
            \begin{itemize}
                \scriptsize
                \item Na robótica móvel, o ambiente é fixo e o robô se move por meio das forças que imprime no ambiente.
                \item As principais características para o movimento de robôs são:
                \begin{itemize}
                    \scriptsize
                    \item[-] \textbf{Estabilidade}: número dos pontos de contato, centro de gravidade, estabilidade estática e dinâmica, inclinação do terreno.
                    \item[-] \textbf{Características do contato}: tamanho, forma, ângulo e atrito por fricção  dos pontos de contato.
                    \item[-] \textbf{Tipo do ambiente}: estrutura, formato e meio (ex: água, ar, solo compactado ou não).
                \end{itemize}
            \end{itemize}
        \end{column}
    \end{columns}

\end{frame}
