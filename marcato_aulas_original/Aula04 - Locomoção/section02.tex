\section{Robôs com Rodas}

\begin{frame}
    \frametitle{Robôs com Rodas}
    \begin{columns}
        \begin{column}{0.5\textwidth}
                \begin{itemize}
                    \scriptsize
                    \item Rodas são a solução mais apropriada para a grande maioria das aplicações robóticas.
                    \item Com três rodas (ou mais) formando um polígono, o robô sempre possui sua estabilidade estática garantida (com projeção do CoG dentro do polígono), no entanto, requer um conjunto apropriado de amortecedores, de modo que as rodas estejam sempre em contato com o solo.
                    \item No entanto, existem diferentes tipos de rodas.
                \end{itemize}
        \end{column}
        \begin{column}{0.5\textwidth}
            \centering
            \includegraphics[width=0.8\textwidth]{img04/robocomrodas.jpg}   
        \end{column}
    \end{columns}
\end{frame}

\begin{frame}
    \frametitle{Comportamento das Rodas Tipo "Padrão"}
    \centering
    \includegraphics[width=0.8\textwidth]{img04/rodaspadrão.png}   
\end{frame}

\begin{frame}
    \frametitle{Comportamento das Rodas Tipo "Padrão"}
    \centering
    \includegraphics[width=0.8\textwidth]{img04/rodaspadrão.png}   
\end{frame}

\begin{frame}
    \frametitle{Roda \textit{Caster}}
    \centering
    \begin{tabular}{cc}
        \includegraphics[width=0.50\textwidth]{img04/roda_caster_1.png} &
        \includegraphics[width=0.25\textwidth]{img04/roda_caster_2.png} \\
    \end{tabular}         
    \begin{itemize}
        \scriptsize
        \item Rodas caster (caster wheel) possuem dois eixo de rotação, um na roda e outro na base de afixação.
        \item Possui dois graus de liberdade por definição, de rolamento e de esterçamento.
    \end{itemize}    
\end{frame}

\begin{frame}
    \frametitle{Roda \textit{Mecanum} (Suéca)}
    \centering
    \begin{tabular}{ccc}
        \includegraphics[width=0.32\textwidth]{img04/sueca_1.png} &
        \includegraphics[width=0.17\textwidth]{img04/sueca_2.png} &
        \includegraphics[width=0.18\textwidth]{img04/sueca_3.png} \\
    \end{tabular}
    \includegraphics[width=0.60\textwidth]{img04/sueca_4.png} 
    \begin{itemize}
        \scriptsize
        \item Rodas suécas (swedish mecanum wheel) possuem rolos na parte externa, fornecendo mais uma via de locomoção.
        \item Possui dois graus de liberdade por definição, ambos de rolamento, um do eixo de rotação e outro dos rolos da roda.
    \end{itemize}    
\end{frame}

\begin{frame}
    \frametitle{Roda Esférica}

    \only<1>
    {
        \centering
        \begin{tabular}{ccc}
            \includegraphics[width=0.20\textwidth]{img04/roda_esférica_2.png} &
            \includegraphics[width=0.15\textwidth]{img04/roda_esférica_3.png} &
            \includegraphics[width=0.15\textwidth]{img04/roda_esférica_4.png} \\
        \end{tabular}\\[0.1cm]
    
        \includegraphics[width=0.30\textwidth]{img04/roda_esférica_1.png}[h]
        \begin{itemize}
            \scriptsize
            \item Rodas esféricas (spherical wheel) podem se mover livremente pelo plano XY, ou seja, dois eixos livres.
            \item Possuem dois graus de liberdade por definição, ambos de rolamento, um na direção do eixo-x e outro do eixo-y.    
        \end{itemize}    
    }
    \only<2-5>
    {
        \begin{itemize}
            \item Em uma roda esférica (ou hemisférica, como no chamado HOG-wheel – Hemispherical Omnidirectional Gimbaled Wheel), o princípio de atuação baseia-se em duas ações independentes:

            \only<3>
            {
            \item Rotação contínua do corpo esférico
                \begin{itemize}
                   \item[] motor faz todo o hemisfério (ou esfera) girar em torno de seu próprio eixo. Enquanto ele estiver perfeitamente “na horizontal” (isto é, eixo de rotação vertical), a roda praticamente “patina” sobre o chão, pois não é gerado torque de acionamento sobre o ponto de contato.
                \end{itemize}
            }
            \only<4>
            {
            \item Vetorização do torque por inclinação
                \begin{itemize}
                    \item[] Uma vez que o hemisfério está girando, basta incliná-lo para que a componente de sujeição ao solo se transforme em tração. Conforme a inclinação (pitch/roll) sobre os eixos lateral e longitudinal, o torque de atrito resultante “puxa” ou “empurra” o robô em qualquer direção no plano.
                \end{itemize}
            }
            \only<5>
            {
            \item Inclinação lateral (tilt “para a direita/esquerda”): puxa/empurra o robô para frente ou para trás.
            \item Inclinação longitudinal (tilt “para frente/trás”): puxa/empurra o robô para os lados.
            }
        \end{itemize}        
    }
\end{frame}

\begin{frame}
    \frametitle{Configurações com duas rodas}
    \centering
    \includegraphics[width=0.9\textwidth]{img04/duas_rodas}\\[0.2cm]
    \includegraphics[width=0.3\textwidth]{img04/tres_rodas_legenda.png}
\end{frame}

\begin{frame}
    \frametitle{Configurações com três rodas}
    \centering
    \only<1>
    {
        \includegraphics[width=0.9\textwidth]{img04/tres_rodas_1.png}\\[0.2cm]
    }
    \only<2>
    {
        \includegraphics[width=0.9\textwidth]{img04/tres_rodas_2}\\[0.2cm]
    }
    \includegraphics[width=0.3\textwidth]{img04/tres_rodas_legenda.png}
\end{frame}


\begin{frame}
    \frametitle{Configurações com quatro rodas}
    \centering
    \only<1>
    {
        \includegraphics[width=0.9\textwidth]{img04/quatro_rodas_1.png}\\[0.2cm]
    }
    \only<2>
    {
        \includegraphics[width=0.9\textwidth]{img04/quatro_rodas_2.png}\\[0.2cm]
    }
    \only<3>
    {
        \includegraphics[width=0.9\textwidth]{img04/quatro_rodas_3.png}\\[0.2cm]
    }
    \includegraphics[width=0.3\textwidth]{img04/tres_rodas_legenda.png}
\end{frame}


\begin{frame}
    \frametitle{Robôs com rodas para ambientes não estruturados (escalada)}
    \centering
    \begin{tabular}{cc}
        \includegraphics[width=0.3\textwidth]{img04/rodas_nao_estruturados_1.png} &
        \includegraphics[width=0.25\textwidth]{img04/rodas_nao_estruturados_2.png} \\
        \includegraphics[width=0.30\textwidth]{img04/rodas_nao_estruturados_3.png} &
        \includegraphics[width=0.25\textwidth]{img04/rodas_nao_estruturados_4.jpg} \\    
    \end{tabular}         
\end{frame}

\begin{frame}
    \frametitle{Configurações de 2, 3 ou 4 rodas}
    \begin{itemize}
        \scriptsize
        \item Two wheels. \href{https://youtu.be/nfWNnqEgZ6M?si=rJqex-IPY0HnM2Nu}{\color{blue}{\underline{Link}}}.
        \item Three wheels - P3DX. \href{https://youtu.be/n3UmgCXw5lY?si=13vaPyZbJwGYQSnf}{\color{blue}{\underline{Link}}}.
        \item Synchro Drive Robot. \href{https://youtu.be/nurCA5Q4_hw?si=euqLklXB4ARirENv}{\color{blue}{\underline{Link}}}.
        \item Palm Pilot Robot. \href{https://youtu.be/jcEdac0kipI?si=ASMeT0IWqRbcNZ5R}{\color{blue}{\underline{Link}}}.
        \item Multi robot formation control - Khepera Team. \href{https://youtu.be/YRN5B37MTL4?si=93_5-Zs5Y52HRbif}{\color{blue}{\underline{Link}}}.
        \item Khepera III Robot Demo. \href{https://youtu.be/3Ku7_5Kvx2Y?si=cx3O6kNljEvL3fJd}{\color{blue}{\underline{Link}}}.
        \item Robo Cão. \href{https://youtube.com/shorts/Lr40ET1kzpU?si=R23eznL3TmZwZRk0}{\color{blue}{\underline{Link}}}.
    \end{itemize}        
\end{frame}
