\section{Robôs com Pernas}

\begin{frame}
    \frametitle{Robôs com Pernas}
    \begin{columns}
        \begin{column}{0.5\textwidth}
            \centering
            \begin{tabular}{cc}
                \includegraphics[width=0.45\textwidth]{img04/pernas_1.jpg} &
                \includegraphics[width=0.45\textwidth]{img04/pernas_2.png} \\
            \end{tabular}    
            
            \includegraphics[width=0.65\textwidth]{img04/pernas_3.png} 
            
            \begin{mdframed}[%
                            backgroundcolor=blue!20,   % cor de fundo
                            linecolor=red,               % cor da borda
                            linewidth=1pt,               % espessura da borda
                            roundcorner=4pt,             % cantos arredondados
                            innertopmargin=6pt,          % espaço interno acima
                            innerbottommargin=6pt,       % espaço interno abaixo
                            innerleftmargin=6pt,         % espaço interno à esquerda
                            innerrightmargin=6pt         % espaço interno à direita
                            ]
                \scriptsize
                Quanto menor o número de pernas, mais difícil é o controle da estabilidade do robô para que este se mantenha de pé.
            \end{mdframed}
        \end{column}
        \begin{column}{0.5\textwidth}
            \begin{itemize}
                \scriptsize
                \item A locomoção por pernas é caracterizada por uma série de pontos de contato articulados entre o robô e o ambiente.
                \item Pernas são, de fato, manipuladores robóticos e possuem diversos fatores a serem modelados e controlados.
                \item O fato é que robôs com pernas podem possuir diversas configurações diferentes, variando não somente na disposição e construção das pernas, mas também na sua quantidade
            \end{itemize}
        \end{column}
    \end{columns}

\end{frame}

\begin{frame}
    \frametitle{Estabilidade Estática}
    \begin{columns}
        \begin{column}{0.5\textwidth}
            \centering
            \begin{tabular}{c}
                \includegraphics[width=0.55\textwidth]{img04/estatica_1.png} \\
                \includegraphics[width=0.55\textwidth]{img04/estatica_2.png} \\
            \end{tabular}                
        \end{column}
        \begin{column}{0.5\textwidth}
            \only<1>
            {
                \begin{itemize}
                    \item a \textbf{estabilidade estática} – capacidade do robô manter-se de pé enquanto parado – \textbf{é cada vez mais difícil conforme o número de pernas decresce}.
                    \item Ou seja, a projeção do \textbf{Centro de Gravidade (CoG)} do robô deve estar dentro da figura geométrica formada pelos seus pontos de contato com o solo (zona de estabilidade estática).
                \end{itemize}
            }
            \only<2>
            {
                \begin{itemize}
                    \scriptsize
                    \item Além estabilidade estática, também é necessário garantir a \textbf{estabilidade dinâmica}, ou seja, durante o movimento.
                    \item Nesse caso, são permitidos menos que três pontos de contato com o solo, e boa parte da estabilidade do robô estará auxiliada pelo \textbf{momento angular}, logo, quanto mais rápido o robô se mover, melhor para a estabilidade dinâmica.
                    \item No entanto, para garanti-la, é preciso desenvolver uma gama de ações de controle para manter o \textbf{CoG} na região de contato.
                \end{itemize}
            }
        \end{column}
    \end{columns}
\end{frame}

\begin{frame}
    \frametitle{Perna é Igual a um Manipulador Robótico}
    \begin{columns}
        \begin{column}{0.5\textwidth}
                \begin{itemize}
                    \item Requer-se um mínimo de dois DoFs para mover uma perna. Um para o movimento de levantamento (lift) da perna e o outro para o movimento de balanço (swing) da perna, para frente ou para trás.
                    \item Na maior parte dos exemplos, pernas possuem três DoFs:
                \end{itemize}
        \end{column}
        \begin{column}{0.5\textwidth}
            \centering
            \includegraphics[width=0.8\textwidth]{img04/manipulador.png}   
        \end{column}
    \end{columns}
\end{frame}

\begin{frame}
    \frametitle{Robôs com Pernas Código Aberto}
    \centering
    \includegraphics[width=1.0\textwidth]{img04/robos_pernas_open_source.png}   
\end{frame}

\begin{frame}
    \frametitle{Robôs com Pernas Código Aberto}
    \begin{itemize}
        \scriptsize
        \item Stanford Pupper. \href{https://youtu.be/NIjodHA78UE?si=Aq9xUQhnpfVq5O4z}{\color{blue}{\underline{Link}}}.
        \item Solo 8 ou Solo 12. \href{https://youtu.be/MkXU-5Oupg4?si=1Vtvc7LNVfHw8Rju}{\color{blue}{\underline{Link}}}.  
        \item Vision Free Cheeta. \href{https://youtu.be/QZ1DaQgg3lE?si=e7FNd9USRKVdUAMg}{\color{blue}{\underline{Link}}}.
    \end{itemize}        
\end{frame}

\begin{frame}
    \frametitle{Movimentos de caminhada (esquerda) e galope (direita) com quatro pernas}
    \centering
    \includegraphics[width=0.7\textwidth]{img04/caminhada_e_galope.png}   
\end{frame}

\begin{frame}
    \frametitle{Movimentos mais comuns em robôs com 6 pernas para manter a estabilidade estática enquanto se movimenta}
    \centering
    \includegraphics[width=0.8\textwidth]{img04/six_legs.png}   
\end{frame}

