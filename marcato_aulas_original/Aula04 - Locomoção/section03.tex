\section{Outras Formas Locomoção}

\begin{frame}
    \frametitle{Outras Formas de Locomoção}
    \centering
    \begin{tabular}{cc}
        \includegraphics[width=0.25\textwidth]{img04/balao.png} &
        \includegraphics[width=0.35\textwidth]{img04/shark.png} \\
        \includegraphics[width=0.30\textwidth]{img04/fish.png} &
        \includegraphics[width=0.55\textwidth]{img04/planar.png} \\    
    \end{tabular}
\end{frame}

\begin{frame}
    \frametitle{Manobrabilidade}
    \begin{columns}
        \begin{column}{0.5\textwidth}
            \centering
            \includegraphics[width=0.9\textwidth]{img04/mecanum_wheels.jpeg}
        \end{column}
        \begin{column}{0.5\textwidth}
            \begin{itemize}
                \scriptsize
                \item \textbf{Manobrabilidade} refere-se a facilidade do robô em executar movimentos (manobras) ao longo do ambiente.
                \item Alguns robôs são \textbf{holonômicos / omnidirecionais}, o que significa que podem se mover a qualquer momento e para qualquer direção, independentemente da sua orientação.
                \item Isso requer uma configuração de rodas \textbf{omnidirecionais}.
            \end{itemize}
        \end{column}
    \end{columns}
\end{frame}

\begin{frame}
    \frametitle{Manobrabilidade}
    \begin{columns}
        \begin{column}{0.5\textwidth}
            \centering
            \includegraphics[width=0.9\textwidth]{img04/ackermann.png}
        \end{column}
        \begin{column}{0.5\textwidth}
            \begin{itemize}
                \scriptsize
                \item Em contraste, considere a configuração conhecida como modelo de \textbf{tração Ackermann} (exemplo: os carros de passeio).
                \item Essa configuração não permite que o robô faça movimentos omnidirecionais. 
                \item Ao contrário, obriga o veículo a se deslocar sobre círculos com raios produzidos pelo esterçamento das rodas, inclusive, círculos estes de raios maiores que o próprio veículo. 
                \item O carro sempre se movimenta sobre círculos, e \textbf{nunca caminha “de lado”, ou seja, sobre o eixo-y}.
            \end{itemize}
        \end{column}
    \end{columns}
\end{frame}


\begin{frame}
    \frametitle{Manobrabilidade}
    \begin{columns}
        \begin{column}{0.5\textwidth}
            \centering
            \begin{mdframed}[%
                            backgroundcolor=blue!20,   % cor de fundo
                            linecolor=red,               % cor da borda
                            linewidth=1pt,               % espessura da borda
                            roundcorner=4pt,             % cantos arredondados
                            innertopmargin=6pt,          % espaço interno acima
                            innerbottommargin=6pt,       % espaço interno abaixo
                            innerleftmargin=6pt,         % espaço interno à esquerda
                            innerrightmargin=6pt         % espaço interno à direita
                            ]
                \scriptsize
                Quanto maior a manobrabilidade de um robô, geralmente, maior é a dificuldade de processamento para gerar sinais de controle para levar o robô às referências desejadas.
            \end{mdframed}
        \end{column}
        \begin{column}{0.5\textwidth}
            \begin{itemize}
                \scriptsize
                \item Controlabilidade refere-se a facilidade em se gerar sinais de controle para um robô de modo a leva-lo a algum resultado.
                \item Há, em geral, uma correlação \textbf{inversa entre a manobrabilidade e a controlabilidade}.
                \item Por exemplo, robôs omnidirecionais com quatro rodas suécas requerem uma grande quantidade de cálculos para transformar o movimento desejado em comandos individuais para cada uma das rodas.
                \item Robôs simples, como o modelo diferencial (P3DX) requerem menor quantidade de cálculos para a obtenção dos comandos individuais de cada roda.
            \end{itemize}
        \end{column}
    \end{columns}
\end{frame}


