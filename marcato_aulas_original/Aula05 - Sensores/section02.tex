\section{Sonares e Lidares}
\begin{frame}
    \frametitle{Sonares}
    \centering
    \begin{columns}
        \begin{column}{0.5\textwidth}
            \centering
            \only<1-2>
            {
                \begin{tabular}{c}
                    \includegraphics[width=0.8\textwidth]{img05/ecolocalização.png} \\
                    \includegraphics[width=0.8\textwidth]{img05/ecolocalização_2.png} \\
                \end{tabular}
            }
            \only<3>
            {
                \begin{tabular}{c}
                    \includegraphics[width=0.8\textwidth]{img05/sonar.png} \\
                \end{tabular}
            }
        \end{column}
        \begin{column}{0.5\textwidth}
            \only<1>
            {
                \begin{itemize}
                    \scriptsize
                    \item O termo SONAR é um acrônimo para \textbf{So}und \textbf{Navigation} and \textbf{R}anging, que, em tradução livre, significa Determinação de Distâncias pela Navegação do Som.
                    \item Dessa maneira, SONARES são dispositivos que medem distâncias (\textbf{rangefinder}) a objetos com base nos princípios sonoros.
                    \item Na natureza, os morcegos e golfinhos se valem dos princípios sonoros para se locomoverem e caçarem. São capazes de montar mapas tridimensionais dos seus arredores.
                \end{itemize}
            }
            \only<2>
            {
                \begin{itemize}
                    \scriptsize
                    \item Por meio de distâncias é possível obter informações acerca do ambiente no qual o sensor está imerso.
                    \item Assim, o sonar é um sensor \textbf{exteroceptivo} e ativo, uma vez que envia energia (sonora) para o ambiente.
                    \item Por meio de distâncias é possível obter informações acerca do ambiente no qual o sensor está imerso.
                    Assim, o sonar é um sensor \textbf{exteroceptivo} e ativo, uma vez que envia energia (sonora) para o ambiente.
                \end{itemize}
            }
            \only<3>
            {
                \begin{itemize}
                    \scriptsize
                    \item Sonares são compostos por dois dispositivos:
                    \begin{itemize}
                        \scriptsize
                        \item Um emissor (emitter) sonoro (piezelétrico).
                        \item Um receptor (receiver) sonoro (piezelétrico).
                    \end{itemize}
                    \item Sonares, por si só, são elementos mecatrônicos completos: possuem transdutores, microcontroladores para a contagem de tempo e cálculo das distâncias, fazem o condicionamento dos sinais envolvidos, e assim por diante. 
                \end{itemize}
            }
        \end{column}    
    \end{columns}
\end{frame}

\begin{frame}
    \frametitle{Sonares}
    \centering
    \includegraphics[width=0.95\textwidth]{img05/sonar_2.png}
\end{frame}

\begin{frame}
    \frametitle{Sonares}
    \centering
    \begin{columns}
        \begin{column}{0.5\textwidth}
            \centering
            \begin{tabular}{c}
                \includegraphics[width=0.8\textwidth]{img05/sonar_3.png} \\
            \end{tabular}
        \end{column}
        \begin{column}{0.5\textwidth}
            \begin{itemize}
                \scriptsize
                \item O espalhamento típico da onda sonora é dado abaixo:
                \item Nota-se que, além do cone energético central (principal, 0o), há energia sendo emitida em outras direções radiais (ângulos).
                \item Essa energia radial pode interferir com as medidas realizadas pelos sonares (obstáculos nessas regiões podem ser obtidos).
            \end{itemize}
        \end{column}    
    \end{columns}
\end{frame}

\begin{frame}
    \frametitle{Sonares}
    \begin{itemize}
        \item O sonar funciona bem quando os obstáculos encontram-se perpendicularmente ao cone principal emitido.
        \item Caso contrário, reflexões especulares atrapalharão as medidas
    \end{itemize} 
    \centering
    \includegraphics[width=0.95\textwidth]{img05/sonar_4.png}
\end{frame}

\begin{frame}
    \frametitle{Sonares: Exemplo UHE Santo Antônio (YOLO)}
    \centering
    \begin{tabular}{c c}
        \includegraphics[width=0.45\textwidth]{img05/SAE_1.jpg} &
        \only<2>
        {
            \includegraphics[width=0.45\textwidth]{img05/SAE_2.jpg}         
        }
        \\
        \end{tabular}
\end{frame}

\begin{frame}
    \frametitle{Sonares: Exemplo UHE Santo Antônio (YOLO)}
    \centering
    \begin{tabular}{c c}
        \includegraphics[width=0.45\textwidth]{img05/SAE_3.jpg} &
        \only<2>
        {
            \includegraphics[width=0.45\textwidth]{img05/SAE_4.jpg}         
        }
        \\
        \end{tabular}
\end{frame}


\begin{frame}
    \frametitle{Lidares}
    \centering
    \begin{columns}
        \begin{column}{0.5\textwidth}
            \centering
            \only<1-2>
            {
                \begin{tabular}{c}
                    \includegraphics[width=0.4\textwidth]{img05/sick.png} \\
                    \includegraphics[width=0.4\textwidth]{img05/hokuyo.png} \\
                \end{tabular}
            }
        \end{column}
        \begin{column}{0.5\textwidth}
            \only<1>
            {
                \begin{itemize}
                    \scriptsize
                    \item O LIDAR (\textbf{LI}ght \textbf{D}etection \textbf{A}nd \textbf{R}anging), ou laser, também mede distâncias (rangefinders) a objetos.
                    \item No caso do lidar, o princípio é similar ao do sonar, envolvendo reflexão de ondas, mas, desta vez não com som, e sim com luz monocromática.
                    \item Uma luz de \textbf{LASER} (\textbf{L}ight \textbf{A}mplification by \textbf{S}timulated \textbf{E}mission of \textbf{R}adiation) é a chamada luz colimada, ou seja, os raios de luz viajam todos praticamente paralelos.
                    \item Caso contrário, reflexões especulares atrapalharão as medidas.
                \end{itemize}
            }
            \only<2>
            {
                \begin{itemize}
                    \scriptsize
                    \item No caso, os raios possuem um comprimento de onda muito bem definido (ou seja, a luz possui uma única cor, monocromática).
                    \item A princípio, o mesmo conceito de “tempo de vôo” utilizado para o sonar também poderia ser utilizado para o laser, uma vez que a velocidade da luz é bem conhecida (299.792.458 m/s).
                \end{itemize}
            }
        \end{column}    
    \end{columns}
\end{frame}

\begin{frame}
    \frametitle{Lidares}
    \centering
    \includegraphics[width=0.95\textwidth]{img05/lidar.png}
\end{frame}

\begin{frame}
    \frametitle{Lidares}
    \centering
    \begin{itemize}
        \scriptsize
        \item A distância pode ser calculada com a seguinte expressão:
    \end{itemize}

    \begin{columns}
        \begin{column}{0.5\textwidth}
            \begin{equation*}
                d = \frac{\lambda \theta}{2 \pi}
            \end{equation*}  
        \end{column}
        \begin{column}{0.5\textwidth}
            \begin{equation*}
                \begin{array}{c}
                2\pi \rightarrow \lambda \\
                d \rightarrow \theta
                \end{array}            
            \end{equation*}
        \end{column}
    \end{columns}
        
    \begin{itemize}
        \scriptsize
        \item[] Em que: 
            \begin{itemize} 
                \scriptsize
                \item $\lambda$ é o comprimento de onda da luz emitida.
                \item $\theta$ é a diferença de fase entre as luzes emitida e recebida.
            \end{itemize}
        \item Conforme percebe-se, a distância detectada pela diferença de fase (d) corresponde ao dobro da distância entre o sensor e o obstáculo (mesmo problema do tempo de voo do sonar).
        \item Dessa forma, a distância D entre o sensor e o obstáculo é:
    \end{itemize}

    \begin{equation}
        D = \frac{d}{2} = \frac{\lambda \theta}{4\pi}
    \end{equation}
    
\end{frame}

\begin{frame}
    \frametitle{Lidares}
    \begin{itemize}
        \scriptsize
        \item Como pode-se perceber, a onda eletromagnética é periódica.
        \item Por conta disso, a distância máxima (d) que pode medir é de, justamente, seu comprimento de onda $\lambda$.
        \begin{itemize}
            \scriptsize
            \item Exemplo: se $\lambda$=60 metros então, qualquer distância acima de 60 metros é ambígua e pode ser confundida com uma distância menor. Por exemplo, um objeto a 65 metros produz a mesma diferença de fase Ɵ que um outro objeto a 5 metros. Dessa forma, o sensor retornaria a distância errada.
        \end{itemize}
        \item No entanto, conforme visto, a distância D entre o sensor e o obstáculo é, na verdade, a metade da distância d.
        \item Portanto, o alcance máximo de um sensor de laser é a metade de seu comprimento de onda.
       \begin{itemize}
            \scriptsize
            \item Exemplo: se $\lambda$=60 metros então, Dmax=30 metros.
        \end{itemize}
        \item Alternativas:
        \begin{itemize}
            \scriptsize
            \item Modulação com múltiplas frequências (multitone ou multifrequência)
            \item LiDARs baseados em tempo de voo (Time-of-Flight – ToF direto)
            \item LiDARs de pulso (pulsed LiDAR)
            \item Técnicas híbridas e DSP avançado
        \end{itemize}
    \end{itemize}
\end{frame}