\section{Introdução}

\begin{frame}
    \frametitle{Introdução}
    \begin{columns}
        \begin{column}{0.4\textwidth}
            \only<1-2>
            {
		  \centering
		  \includegraphics[width=4.5cm, height=2.5cm]{img03/vantagens_e_desvantagens.png}
            }
        \end{column}
        \begin{column}{0.6\textwidth}
            \only<1>
            {
            \begin{itemize}
                \small
                \item Vantagens:
                    \begin{itemize}
                        \scriptsize
                        \item Aumento da produtividade e qualidade final do produto.
                        \item Minimização de operações.
                        \item Maior demanda de contratação de mão-de-obra especializada.
                        \item Facilidade na reprogramação para outras tarefas.
                        \item Operação em ambientes hostis, tarefas desagradáveis ou repetidoras.
                        \item Trabalho por longos períodos sem interrupções.
                    \end{itemize}
            \end{itemize}
            }
            \only<2>
            {
            \begin{itemize}
                \small
                \item Desvantagens:
                    \begin{itemize}
                        \scriptsize
                        \item Decréscimo do emprego não-especializado na indústria.
                        \item Necessidade de treinamento especializado (caro) para pessoal.
                        \item Migração de indústrias de países pobres (mão-de-obra barata e não-especializada) para outros países com mão-de-obra especializada.
                    \end{itemize}
            \end{itemize}
            }
        \end{column}
    \end{columns}
\end{frame}

\begin{frame}
    \frametitle{Robô Versus Homem}
    \centering
    \includegraphics[width=7cm, height=4cm]{img03/robovsman.png}      
    \begin{itemize}
        \scriptsize
        \item Em 1981, um estudo mostrou que um único robô custaria, no Japão, \textbf{17 mil dólares}, teria vida útil de 6 anos trabalhando 22 horas por dia (2 horas para manutenções programadas), 7 dias por semana, totalizando \textbf{48 mil horas trabalhadas}.
        \item Um operário humano custaria \textbf{13 mil dólares} por ano, trabalhando \textbf{40} horas semanais. Este trabalhador, levando em consideração, férias, finais de semana, etc..
        \item Demoraria mais de {\color{red} \textbf{30}} anos para trabalhar o mesmo número de horas do robô.
        \item O homem custa, também, {\color{red} \textbf{23}} vezes mais que o robô, para trabalhar o mesmo número de horas.
    \end{itemize}
\end{frame}

\begin{frame}
    \frametitle{Aspectos para Avaliação}
    \centering
    \includegraphics[width=5cm, height=3cm]{img03/robovsman_2.png}      
    \begin{itemize}
        \scriptsize
        \item O preço da aquisição e instalação do robô é alto.
        \item Seu payback depende do custo dos insumos utilizados, que refletirão no preço do produto final, além, claro, da demanda do produto final (capacidade de mercado).
        \item Para considerar a compra de um sistema automatizado, deve-se considerar:
        \begin{itemize}
            \item Número de empregados substituídos por máquinas;
            \item Número de turnos/dia que os robôs trabalharão;
            \item Produtividade comparada ao custo;
            \item Custo do projeto e manutenção;
            \item Custos de equipamentos periféricos e de reposição.
        \end{itemize}
    \end{itemize}
\end{frame}

\begin{frame}
    \frametitle{Mapa Mundial da Eletricidade}
	\centering
	\includegraphics[width=9cm, height=6cm]{img03/electricityworldmap.png}
\end{frame}

\begin{frame}
    \frametitle{Densidade de Robôs Industriais a cada 10.000 Empregados}
	\centering
	\includegraphics[width=9cm, height=6cm]{img03/DesindadeRobos.png}
\end{frame}

\begin{frame}
    \frametitle{Estoque Operacional Global de Robôs Industriais}
	\centering
	\includegraphics[width=9cm, height=6cm]{img03/operationalrobots.png}
\end{frame}

\begin{frame}{Robôs Industriais no Final do Século 20}
    \centering
    \includegraphics[width=0.4\linewidth]{img03/robo_industrial.png}

    \begin{itemize}
        \item Forte crescimento da automação fixa nas linhas de montagem.
        \item Robôs industriais dominavam tarefas repetitivas e perigosas.
        \item Setores como automobilístico e eletrônico foram os maiores beneficiados.
        \item Resultados: aumento de produtividade, redução de custos, e padronização.
    \end{itemize}
    \vspace{0.3cm}
\end{frame}

\section{Robótica Móvel}
\begin{frame}{Desafios Não Resolvidos pela Automação Fixa}
    \begin{columns}
        \begin{column}{0.5\textwidth}
            \centering
            \includegraphics[width=0.70\linewidth]{img03/desafios.png}
        \end{column}
        \begin{column}{0.5\textwidth}
            \begin{itemize}
                \item Ambientes dinâmicos exigem flexibilidade que robôs industriais não possuem.
                \item Transporte interno, inspeções e logística interna ainda dependiam de humanos.
                \item Crescente demanda por soluções autônomas e adaptativas.
            \end{itemize}
        \end{column}
    \end{columns}
\end{frame}

\begin{frame}{O Século 21 e os Robôs Móveis}
    \begin{itemize}
        \item Avanço em sensores, controle e inteligência artificial viabiliza robôs móveis.
        \item Robôs autônomos já operam em armazéns, hospitais, fazendas e cidades.
        \item Tendência: robôs móveis ocuparão o espaço que a automação fixa não alcançou.
        \item Caminhamos para uma convivência integrada entre humanos e robôs móveis.
    \end{itemize}
    \vspace{0.3cm}
    \centering
    \includegraphics[width=0.45\linewidth]{img03/autonomous_car.png}
\end{frame}

\begin{frame}
    \frametitle{Robótica Móvel}
    \begin{columns}
        \begin{column}{0.4\textwidth}
		  \centering
		  \includegraphics[width=4.5cm, height=6cm]{img03/autonomia.png}            
        \end{column}
        \begin{column}{0.6\textwidth}
            \begin{itemize}
                \small
                \item Um robô móvel é um veículo terrestre, aéreo ou aquático que apresenta certo grau de \textbf{\underline{autonomia}}.

                \item Autonomia: capacidade de executar determinadas ações sem necessidade de intervenção humana.
                \item Exemplos:
                    \begin{itemize}
                        \scriptsize
                        \item desviar de obstáculos para evitar colisões;
                        \item seguir para uma estação de recarga;
                        \item escolher um caminho para atingir uma posição;
                        \item reconhecer o ambiente em que se encontra.
                    \end{itemize}
            \end{itemize}
        \end{column}
    \end{columns}
\end{frame}

\begin{frame}
	\centering
	\includegraphics[width=9cm, height=6cm]{img03/autonomyvsendurance.png}
\end{frame}

\begin{frame}
    \frametitle{Graus de Autonomia}
    \centering
    \includegraphics[width=9cm, height=4cm]{img03/grausautonomia.png}            
\end{frame}

\begin{frame}
    \frametitle{Aprendizado de Máquina}
    \begin{columns}
        \begin{column}{0.5\textwidth}
		  \centering
		  \includegraphics[width=5.5cm, height=6cm]{img03/aprendizado.png}            
        \end{column}
        \begin{column}{0.5\textwidth}
            
            \begin{itemize}
                \item \scriptsize Com a tecnologia atual, autonomia plena é viável apenas para tarefas simples, como as executadas por robôs domésticos (tarefas de limpeza, de entretenimento, etc.).
                \item \scriptsize \textbf{\underline{aprendizado de máquina}}: o algoritmo executado pelo robô é ajustado para que o robô realize sua tarefa com maior eficiência (ex: treinar o robô para contornar obstáculos).
                \item \scriptsize \textbf{\underline{cooperação}}: da combinação de comportamentos simples é possível emergir um comportamento complexo (ex: encontrar um caminho até o alvo utilizando múltiplos robôs que possuem limitações individuais de sensoriamento).
            \end{itemize}
        \end{column}
    \end{columns}
\end{frame}

