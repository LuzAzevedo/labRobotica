\section{Estrutura dos Robôs Móveis}
\begin{frame}
    \frametitle{Estrutura de Robôs Móveis Terrestres}
    \begin{columns}
        \begin{column}{0.5\textwidth}
            \begin{itemize}
                \item \scriptsize \textbf{Tração/Propulsão}: Responsável pela locomoção do robô.
                \begin{itemize}
                    \item \scriptsize \textbf{\underline{Rodas}}: meio de tração mais simples, mas confina o robô a se locomover em superfícies planas. Carros..
                    \item \scriptsize \textbf{\underline{Pernas}}: meio de tração mais sofisticado, mas difícil de operar (muitos graus de liberdade) e construir (juntas eficientes). Humanóides, insetos, etc.
                    \item \scriptsize \textbf{\underline{Esteiras}}: meio de tração bom para vencer obstáculos, mas limita a capacidade de manobra do robô. Tanques de guerra.
                    \item \scriptsize \textbf{\underline{Outros Meios}}: rastejante (répteis), saltitante (canguru), combinação dos demais.
                \end{itemize}
            \end{itemize}
        \end{column}
        \begin{column}{0.5\textwidth}
		  \centering
		  \includegraphics[width=5.5cm, height=6cm]{img03/Tracao-Propulsao.png}            
        \end{column}
    \end{columns}
\end{frame}

\begin{frame}
    \frametitle{Sensoriamento}
    \scriptsize
    \begin{itemize}
        \item responsável por medir grandezas físicas do robô e do ambiente.
        \item Exemplos: distâncias, rotações, proximidades, imagens, sistema de posicionamento global, etc.
    \end{itemize}
    \begin{columns}
        \begin{column}{0.5\textwidth}
            \centering
		  \includegraphics[width=5.5cm, height=6cm]{img03/sensores.png}            
        \end{column}
        \begin{column}{0.5\textwidth}
            \centering
		  \includegraphics[width=5.5cm, height=6cm]{img03/pointcloud.png}            
        \end{column}
    \end{columns}
\end{frame}

\begin{frame}
    \frametitle{Processamento}
    \begin{columns}
        \begin{column}{0.5\textwidth}
            \only<1>
            {
                \centering
    		  \includegraphics[width=5.5cm, height=6cm]{img03/proce_1.png}
            }
            \only<2>
            {
                \centering
    		  \includegraphics[width=5.5cm, height=6cm]{img03/arduino.jpg}
            }
        
        \end{column}
        \begin{column}{0.5\textwidth}
            \only<1>
            {
                \centering
    		  \includegraphics[width=5cm, height=5.2cm]{img03/proce_2.png} 
            }
            \only<2>
            {
                \centering
    		  \includegraphics[width=6cm, height=5.2cm]{img03/odroid.jpg} 
            }
        \end{column}
    \end{columns}
    \scriptsize
    \begin{itemize}
        \item responsável pela execução do controle de alto nível. Exemplo: Grandezas físicas do robô e do ambiente. Medir distâncias, rotações, proximidades, imagens, sistema de posicionamento global, SLAM, YOLO, etc.
    \end{itemize}
\end{frame}

\begin{frame}{Flight Control Unit (FCU)}
    \begin{itemize}
        \scriptsize
        \item \textbf{Definição:} A Flight Control Unit (FCU) é o "cérebro" de controle dos robôs móveis aéreos, sendo responsável por interpretar sensores, calcular comandos e estabilizar o veículo.

        \only<1>
        {
            \item \textbf{Funções principais:}
            \begin{itemize}
                \scriptsize
                \item Leitura e fusão de dados de sensores (IMU, GPS, barômetro, magnetômetro).
                \item Execução de algoritmos de controle de baixo nível (PID, MPC, LQR).
                \item Comando de atuadores (motores, servos, superfícies de controle).
                \item Comunicação com a estação de solo (via MAVLink ou similar).
            \end{itemize}
        }

        \only<2>
        {
            \item \textbf{Exemplos de FCUs populares:}
            \begin{itemize}
                \item Pixhawk e Cube Orange
            \end{itemize}
            
            \item \textbf{Aplicações:}
            \begin{itemize}
                \item Drones multirrotores, aviões VTOL, carros autônomos, robôs anfíbios.
            \end{itemize}
        }
        \end{itemize}
    
    \vspace{0.3cm}
    \centering
    \includegraphics[width=0.35\linewidth]{img03/cube_orange.jpg} % Substitua pela imagem desejada
\end{frame}

\begin{frame}
    \frametitle{Comunicação}
    \scriptsize
    \begin{itemize}
        \item responsável pelo tráfego de informações e pela comunicação do robô com o mundo exterior.
        \item Exemplos: displays, redes sem fios, interpretadores e sintetizadores de fala, etc.
    \end{itemize}
    \begin{columns}
        \begin{column}{0.33\textwidth}
            \centering
		  \includegraphics[width=3.5cm, height=4cm]{img03/telemetria_1.png}  
        \end{column}
        \begin{column}{0.33\textwidth}
            \centering
		  \includegraphics[width=3.5cm, height=6cm]{img03/telemetria_2.png} 
        \end{column}
        \begin{column}{0.33\textwidth}
            \centering
		  \includegraphics[width=3.5cm, height=3.5cm]{img03/telemetria_3.png} 
        \end{column}
    \end{columns}    
\end{frame}


\begin{frame}
    \frametitle{Alimentação}
    \scriptsize
    \begin{itemize}
        \item responsável por fornecer energia aos demais subsistemas, mantendo as funcionalidades ativas.
        \item Exemplos: baterias, motogeradores, cordões umbilicais, slip ring, etc.
    \end{itemize}
    \only<1>
    {
    \begin{columns}
        \begin{column}{0.5\textwidth}
            \centering
		  \includegraphics[width=3.5cm, height=4cm]{img03/bateria_1.png}  
        \end{column}
        \begin{column}{0.5\textwidth}
            \centering
		  \includegraphics[width=3.5cm, height=3.5cm]{img03/bateria_3.png} 
        \end{column}
    \end{columns}   
    }
    \only<2>
    {
    \begin{columns}
        \begin{column}{0.5\textwidth}
            \centering
		  \includegraphics[width=3.5cm, height=4cm]{img03/bateria_2.png}  
        \end{column}
        \begin{column}{0.5\textwidth}
            \centering
		  \includegraphics[width=3.5cm, height=3.5cm]{img03/bateria_4.png} 
        \end{column}
    \end{columns}   
    }
    \only<3>
    {
    \begin{columns}
        \begin{column}{0.6\textwidth}
            \centering
		  \includegraphics[width=6cm, height=4cm]{img03/slip_ring_1.png}  
        \end{column}
        \begin{column}{0.4\textwidth}
            \centering
		  \includegraphics[width=3.5cm, height=2.5cm]{img03/slip_ring_2.png} 
        \end{column}
    \end{columns}   
    }

\end{frame}

