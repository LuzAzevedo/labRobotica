\section{Vision}

\begin{frame}
	\frametitle{Quase todos animais possuem olhos}
	\only<1>
	{
		\begin{figure}[!htb]
			\centering
			\includegraphics[width=6cm, height=6cm]{Figuras/inseto.png}
			\caption{Olhos compostos (Insetos)}
		\end{figure}
	}
	\only<2>
	{
		\begin{figure}[!htb]
			\centering
			\includegraphics[width=6cm, height=6cm]{Figuras/aranha.png}
			\caption{Olhos principais e secundários (aranha)}
		\end{figure}
	}
	\only<3>	
	{
		\begin{figure}[!htb]
			\centering
			\includegraphics[width=6cm, height=6cm]{Figuras/olhos-azuis.jpg}
			\caption{Olhos baseados em lente (seres humanos)}
		\end{figure}
	}	
	\only<4>	
	{
		\begin{itemize}
		\item Explora ambientes desconhecidos
		\item Negocia espaço físico com os outros
		\item Detecta e reconhece a presa à distância, para então caça-la
		\item E assim por diante...
		\end{itemize}
	}		
\end{frame}

\begin{frame}
	\begin{beamerboxesrounded}{}
		\centering
		\begin{LARGE}
			"A picture is worth a thounsand words"
		\end{LARGE}
	\end{beamerboxesrounded}
\end{frame}

\begin{frame}
	\frametitle{Por que é tão difícil fazer um computador enxergar?}
	\only<1>
	{
		\begin{figure}[!htb]
			\centering
			\includegraphics[width=6cm, height=6cm]{Figuras/camera.jpg}
			\caption{USB 2.0 board-level camera - mvBlueFOX-MLC - Global Shutter - 90 fps}
		\end{figure}	
	}
	\only<2>
	{
		\begin{figure}[!htb]
			\centering
			\includegraphics[width=8cm, height=6cm]{Figuras/backhand2-sharapova.jpg}
			\caption{Backhand tenista Sharapova}
		\end{figure}	
	}
	\only<3>
	{
		\begin{itemize}
		\item Uma câmera pode entregar para o computador muitos \emph{frames} por segundo, tal como a retina do olho entrega para o cérebro humano
		\item No entanto, cada \emph{frame} é uma coleção de números positivos que mede a quantidade de luz incidente sobre uma localização em particular (\emph{pixel}) sobre uma superfície foto-sensível.
		\item Geralmente são utilizadas duas formas de representação:
			\begin{itemize}
				\item Canal simples: Escala de Cinza
				\item Colorido (RGB)
			\end{itemize}
		\item Diferentes resoluções
			\begin{itemize}
				\item Exemplo: 1000 x 750 (Largura x Altura)
			\end{itemize}
		\end{itemize}	
	}
	\only<4>
	{
		\begin{block}{Gradiente de Imagem mostrando os valores do pixel desde o preto (0) até o branco (255)}
			\begin{figure}[!htb]
				\centering
				\includegraphics[width=10cm, height=2cm]{Figuras/EscalaCinza.png}
			\end{figure}	
		\end{block}
	}	
	\only<5>
	{
		\begin{block}{O Espaço de Cores RGB é aditivo}
			\begin{figure}[!htb]
				\centering
				\includegraphics[width=5cm, height=3cm]{Figuras/RGB_1.png} \\~\\
				\includegraphics[width=8cm, height=3cm]{Figuras/RGB_2.png}				
			\end{figure}	
		\end{block}
	}	
	
	\only<6>
	{
		\begin{block}{Cada pixel tem três números representando o \emph{red, green e blue} respectivamente}
			\begin{figure}[!htb]
				\centering
				\includegraphics[width=9cm, height=6cm]{Figuras/lenna.png}
			\end{figure}	
		\end{block}
	}
	\only<7>
	{
		\begin{block}{Para um computador, o espelho lateral de um carro é somente um matriz com números positivos}
			\begin{figure}[!htb]
				\centering
				\includegraphics[width=9cm, height=6cm]{Figuras/darpa.png}
			\end{figure}	
		\end{block}
	}
	
\end{frame}

\begin{frame}
	\frametitle{Como interpretar os valores dos pixels?}
	\only<1>
	{
			\begin{figure}[!htb]
				\centering
				\includegraphics[width=10cm, height=8cm]{Figuras/identificacao_imagem.png}
			\end{figure}	
	}	
	\only<2>
	{
			\begin{figure}[!htb]
				\centering
				\includegraphics[width=6cm, height=3cm]{Figuras/apple_red.png}
			\end{figure}	
	}	
	\only<3>
	{
			\begin{figure}[!htb]
				\centering
				\includegraphics[width=6cm, height=3cm]{Figuras/apple_green.png}
			\end{figure}	
	}	
	\only<4>
	{
			\begin{figure}[!htb]
				\centering
				\includegraphics[width=8cm, height=8cm]{Figuras/pontodevista.jpg}
			\end{figure}	
	}	
	\only<5>
	{
			\begin{figure}[!htb]
				\centering
				\includegraphics[width=8cm, height=7.5cm]{Figuras/background.png}
			\end{figure}	
	}
	\only<6>
	{
		\begin{figure}[!htb]
			\centering
			\includegraphics[width=6cm, height=8cm]{Figuras/Challenges.png}
		\end{figure}	
	}
	\only<7>
	{
		\begin{itemize}
		\item O mesmo objeto de interesse pode estar mais próximo ou mais distante
		\item O mesmo objeto pode ser capturado de um ângulo diferente
		\item As condições de iluminação ambiente mudam completamente os valores dos pixels
		\item Entre outros desafios
		\end{itemize}
	}

\end{frame}

\begin{frame}
    \frametitle{YOLO - You Only Look Once}
    \centering
    \includegraphics[width=8cm, height=4cm]{Figuras/yolo.png}
\end{frame}

	
