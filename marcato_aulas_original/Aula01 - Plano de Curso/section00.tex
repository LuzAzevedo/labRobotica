\section{Introdução}

%----------------------------------------------------------------------------------------

\begin{frame}{YouTube}
	\frametitle{YouTube}

    \begin{figure}
        \includegraphics[width=0.75\linewidth]{img/youtube.png}
    \end{figure}

    \vspace{0.5cm}

    \begin{center}
        \href{https://www.youtube.com/@AndreMarcato}{https://www.youtube.com/@AndreMarcato}    
    \end{center}
    
\end{frame}

\begin{frame}{Aviso de Uso Restrito}
    \begin{columns}
        \footnotesize
        \column{0.7\textwidth}
        \textbf{Esta apresentação é de uso interno e destina-se exclusivamente a fins de treinamento.}
        \\~ 
        \\ 
        Não deve ser distribuída ou compartilhada externamente.
        \\~
        \\
        Algumas imagens e conteúdos foram retirados de fontes diversas para fins educacionais.
        \\~
        \\
        \textbf{Fontes principais:}
        \begin{itemize}
            \item Artigos e apresentações acadêmicas
            \item Materiais técnicos disponíveis na internet
            \item Documentos internos e treinamentos anteriores
            \item Consulta a materiais elaborados por diversos autores e/ou empresas 
        \end{itemize}
        
        \column{0.3\textwidth}
        \includegraphics[width=0.7\textwidth]{img/forinternaluseonly.png} % Substitua "example-image" pelo caminho da sua imagem
    \end{columns}
\end{frame}

\begin{frame}
    \frametitle{Identificação da Disciplina}
    \begin{itemize}
        \item \textbf{\underline{Disciplina}:} Robótica Móvel
        \item \textbf{\underline{Código}:} ENE122
        \item \textbf{\underline{Turma}:} A
        \item \textbf{\underline{Horário}:} Terças e Quintas das 15h às 17h
        \item \textbf{\underline{Professor}:} André Luís Marques Marcato
        \item \textbf{\underline{E-mails}:} amarcato@ieee.org / almmarcato@gmail.com
    \end{itemize}
\end{frame}

\begin{frame}{Contexto Geral}
    {\small
    A disciplina de Robótica Móvel visa introduzir os alunos aos conceitos fundamentais da robótica aplicada ao movimento e controle de robôs móveis. 
    }
    \begin{itemize}
        \item Locomoção
        \item Sensores
        \item Teleoperação
        \item Cinemática
        \item Controle
        \item Planejamento de caminhos
        \item Localização
    \end{itemize}
    {\scriptsize \color{red}
    Finalmente, espera-se que o aluno aprimore sua capacidade de pesquisa e estudo de forma individual e em grupo, expressão escrita e oral e, também, suas habilidades de programação.
    }
\end{frame}

\begin{frame}{Equipamentos e Aplicativos Necessários}
    Os alunos matriculados na disciplina deverão possuir:
    \begin{enumerate}[a] 
		\item Acesso a computador desktop ou laptop ou tablet ou celular (trabalhos poderão ser propostos no computador ou laptop);
		\item Acesso a \textit{link} de internet com velocidade compatível para assistir vídeo aulas, assistir vídeos e realizar atividades \textit{on-line};
		\item Navegador de internet compatível com aplicativos/ferramentas do Google: \textit{Google ClassRoom, Google Drive, Gmail, Google Colab, entre outras}.
        \item Softwares: ROS, Gazebo, Mission Planer, Máquinas Virtuais (Ex. Virtual Box)
    \end{enumerate}
\end{frame}

\begin{frame}{Ementa da Disciplina}
    \begin{itemize}
        \item Introdução à robótica móvel; 
        \item Robótica e automação industrial; 
        \item Sensores e atuadores aplicados à robótica móvel; 
        \item Locomoção de robôs móveis; 
        \item Cinemática de robôs móveis; 
        \item Controle de robôs móveis; 
        \item Planejamento de caminhos e trajetórias para robôs móveis; 
        \item Técnicas de localização para robôs móveis.
    \end{itemize}
\end{frame}


\begin{frame}{Metodologias de Ensino:}
    \begin{enumerate}[a]
		\item \textbf{Aulas Expositivas}: O professor utilizará aulas expositivas tradicionais empregando recursos audiovisuais (multimidia, google colab, projeção da tela do tablet, entre outros) e quadro negro.
		\item \textbf{Aprendizagem Ativa:} Resolução de exemplos em conjunto ou individualmente.;
		\item \textbf{Projetos Práticos:} Aulas ou textos informados com antecedência para que os alunos possam fazer a internalização dos conceitos essenciais antes da aula e depois, junto à turma, resolver exercícios, discutir o conhecimento adquirido e tirar possíveis dúvidas com a ajuda e orientação do professor.
    \end{enumerate}
\end{frame}

\section{Critérios de Avaliação}
\begin{frame}{Critérios de Avaliação}
    \small
    \begin{itemize}
		\item Serão realizadas quatro provas individuais com conteúdo programático cumulativo nas quais poderão ser propostas atividades adicionais que serão incorporadas na avaliação. Das quatro provas, serão selecionadas as 3 melhores notas para performar o total de 75\% da nota final total. 
        \item Para estas provas poderão haver segunda chamada de acordo com os critérios estabelecidos no RAG. Será realizada uma única prova de segunda chamada no final do período com 4 questões, uma referente a cada prova. Se o(s) pedido(s) de segunda chamada for(em) julgado(s) procedente(s), o aluno fará somente a(s) questão(ões) relativa(s) à(s) prova(s) que não compareceu. 
		\item Em grupos de até 4 alunos, será apresentado um trabalho final da disciplina, performando os 25\% restantes da nota final total. 
    \end{itemize}
\end{frame}

\begin{frame}{Segunda Chamada}

    \centering
    \includegraphics[width=0.95\paperwidth]{img/seg_chamada_a.png}
    \includegraphics[width=0.95\paperwidth]{img/seg_chamada_b.png}
    
\end{frame}

\begin{frame}{Controle de Frequência}
    \begin{itemize}
        \item A frequência dos alunos será avaliada de acordo com estabelecido na RAG;
        \item A presença e interatividade durante as aulas poderão ser bonificadas através de um "\textit{plus}" nas notas das avaliações.
		\item \textbf{Critério de aprovação:} É aprovado, quanto ao aproveitamento, em todas as disciplinas ou conjunto de atividades acadêmicas curriculares, a discente ou o discente que alcançar a nota final igual ou superior a 60\% (sessenta por cento) da nota máxima;
		\item \textbf{Trancamento:} O trancamento da disciplina poderá ser realizado de acordo com estabelecido no RAG e em conformidade com o calendário acadêmico da graduação. 
    \end{itemize}
\end{frame}

\begin{frame}{Calendário do Curso}
    \begin{center}
		\begin{tabular}{l|cccccccccc}\hline
			\multicolumn{1}{c|}{\textbf{Mês}} & \multicolumn{10}{c}{\textbf{Dia}}\\\hline\hline
			Abril   & 22        & 24 & 29 &           &           &           &    &    &    & \\
			Maio    & \sout{01} & 06 & 08 & 13        & 15        & 20        & 22 & 27 & 29 & \\
			Junho   & 03        & 05 & 10 & 12        & 17        & \sout{19} & 24 & 26 &    & \\
			Julho   & 01        & 03 & 08 & 10        & 15        & 17        & 22 & 24 & 29 & 31 \\
			Agosto  & 05        & 07 & 12 & 14        & 19        & 21        &    &    &    & \\
			\hline
		\end{tabular}
    \end{center}
\end{frame}

\begin{frame}{Cronograma de Apresentação dos Conteúdos}

    \scriptsize
    
    \begin{center}
		\begin{tabular}{c|l}\hline
		  \textbf{Data} & \multicolumn{1}{|c}{\textbf{Atividade/Conteúdo}}\\ \hline\hline
            22/04/2025 (Ter) & Plano de Curso \\
            24/04/2025 (Qui) & Python e Google Colab\\
            29/04/2025 (Ter) & Introdução \\ \hline
            01/05/2025 (Qui) & FERIADO\\ 
            06/05/2025 (Ter) & Locomoção e Sensores\\ 
            08/05/2025 (Qui) & ROS (1) \\  
            13/05/2025 (Ter) & Cinemática\\
            15/05/2025 (Qui) & \color{red} PRIMEIRO TVC \\
            20/05/2025 (Ter) & Cinemática \\
            22/05/2025 (Qui) & ROS (2) \\ 
            27/05/2025 (Ter) & Controle\\ 
            29/05/2025 (Qui) & ROS (3) \\ \hline 
            03/06/2025 (Ter) & Controle\\
            05/06/2025 (Qui) & ROS (4) \\ 
            10/06/2025 (Ter) & \color{red} SEGUNDO TVC \\
            12/06/2025 (Qui) & ROS (5) \\
            17/06/2025 (Ter) & Planej. Trajetória\\
            19/06/2025 (Qui) & FERIADO \\ 
            24/06/2025 (Ter) & Otimização\\
            26/06/2025 (Qui) & ROS (6) \\ \hline
		\end{tabular}
    \end{center}
\end{frame}

\begin{frame}{Cronograma de Apresentação dos Conteúdos}

    \scriptsize
    
    \begin{center}
		\begin{tabular}{c|l}\hline
		  \textbf{Data} & \multicolumn{1}{|c}{\textbf{Atividade/Conteúdo}}\\ \hline\hline
            01/07/2025 (Ter) & Campos Potenciais e A \* \\ 
            03/07/2025 (Qui) & ROS (7) \\ 
            08/07/2025 (Ter) & \color{red} TERCEIRO TVC \\ 
            10/07/2025 (Qui) & ROS (8) \\ 
            15/07/2025 (Ter) & Localização\\
            17/07/2025 (Qui) & ROS (9) \\ 
            22/07/2025 (Ter) & Localização\\
            24/07/2025 (Qui) & ROS (10) \\
            29/07/2025 (Ter) &  Localização \\
            31/07/2025 (Qui) &  \color{red} QUARTO TVC \\ \hline
            05/08/2025 (Ter) &  Apresentação Trabalhos Finais \\
            07/08/2025 (Qui) &  Apresentação Trabalhos Finais \\
            12/08/2025 (Ter) &  Apresentação Trabalhos Finais \\
            14/08/2025 (Qui) &  Apresentação Trabalhos Finais \\
            19/08/2025 (Ter) &  Apresentação Trabalhos Finais \\
            21/07/2025 (Qui) &  \color{red} SEGUNDA CHAMADA \\ \hline \hline
		\end{tabular}
    \end{center}
\end{frame}





