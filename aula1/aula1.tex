\title[Introdução ao ROS 2]{Aula 1 — Introdução ao ROS 2}
\author[Marcato]{Professor: André L. M. Marcato}
\institute[UFJF]{Universidade Federal de Juiz de Fora \\ Curso de Engenharia Elétrica — Robótica e Automação Industrial}
\date{}

% ---------- Title ----------
\begin{frame}
  \titlepage
\end{frame}

\section{Bibliografia}

\begin{frame}
	\frametitle{Livros Texto}
	\only<1>
	{
		  \centering
		  \includegraphics[width=4.5cm, height=6cm]{RobotProgramming.png}
	}

        \only<2>
	{
		\begin{columns}
			\begin{column}{0.5\textwidth}
				\centering
				\includegraphics[width=4.5cm, height=6cm]{ROS_Springer.jpg}
			\end{column} 
			\begin{column}{0.5\textwidth}
				\centering
				\includegraphics[width=4.5cm, height=6cm]{Learning.jpg}
			\end{column} 
		\end{columns}
	}	

        \only<3>
	{
		\begin{columns}
			\begin{column}{0.5\textwidth}
				\centering
				\includegraphics[width=4.5cm, height=6cm]{LivroPython1.png}
			\end{column} 
			\begin{column}{0.5\textwidth}
				\centering
				\includegraphics[width=4.5cm, height=6cm]{LivroPython2.png}
			\end{column} 
		\end{columns}
	}	
\end{frame}

% ---------- ROS2 Website ----------
\begin{frame}{ROS Website}
\begin{itemize}
  \item \url{https://www.ros.org/}
\end{itemize}

% here add image of the ros2 website, the image is in the images folder
\begin{figure}
  \centering
  \includegraphics[width=0.85\textwidth]{ros2website.png}
\end{figure}
\end{frame}

% ---------- Overview ----------
\begin{frame}{Visão Geral}
\justifying
\begin{itemize}
  \item Entender o ecossistema \textbf{ROS 2} \emph{(nós, tópicos, serviços, actions, parâmetros, etc.)}.
  \item Desenvolver aplicações para controlar o movimento de um robô.
  \item Organização do workspace (\texttt{colcon}, pacotes, \texttt{launch}).
  \item Ferramentas essenciais: \texttt{ros2 CLI}, RViz2, rqt, tf2, simuladores.
  \item Usar os simuladores (exemplo: \texttt{Gazebo}) para testar as aplicações.
\end{itemize}
\end{frame}

% ---------- Objetivos ----------
\begin{frame}{Objetivos da Aula}
\begin{block}{Conceitos que serão abordados:}
\begin{itemize}
  \item \textbf{Áreas de aplicação do ROS}
  \item \textbf{Como funciona um processo robótico}
  \item \textbf{O que é ROS} - definição, conceitos fundamentais e arquitetura básica
  \item \textbf{Principais diferenças entre ROS 1 e ROS 2}
  \item \textbf{Evolução do ROS, história e distros}
\end{itemize}
\end{block}
\end{frame}

% ---------- Aplicações Industriais ----------
\begin{frame}{Aplicações Industriais}
\begin{columns}[c,onlytextwidth]
\column{.5\textwidth}
% Vídeo do Nav2
\vfill
\begin{minipage}{\textwidth}
\centering
\href{https://www.youtube.com/watch?v=CYaN43TJANc}{\includegraphics[width=0.9\textwidth]{nav2_thumbnail.jpg}}
\end{minipage}
\begin{center}
\textit{\small Robô feito com \href{https://docs.nav2.org/}{\textcolor{UFJFWine}{\textbf{Nav2}}}}
\end{center}
\vfill

\column{.5\textwidth}
\vfill
\begin{itemize}
  \item \textbf{Veículos Autônomos}
  \item \textbf{Robôs de Serviço/Assistência}
  \item \textbf{Robótica Móvel em Logística}
  \item \textbf{Robôs Industriais Colaborativos}
\end{itemize}
\vfill
\end{columns}
\end{frame}

% ---------- Aplicações Especializadas ----------
\begin{frame}{Aplicações Especializadas}
\begin{columns}[c,onlytextwidth]
\column{.5\textwidth}
% Vídeo de Drones
\vfill
\begin{minipage}{\textwidth}
\centering
\href{https://www.youtube.com/watch?v=J_-m0ZgxgFk&t=201s}{\includegraphics[width=0.9\textwidth]{drones_thumbnail.jpg}}
\end{minipage}
\begin{center}
\textit{\small \href{https://amiga.farm-ng.com/docs/amiga_quick_start/amiga-quick-start/}{\textcolor{UFJFWine}{\textbf{Robô agrícola}}} com integração com o ROS}
\end{center}
\vfill

\column{.5\textwidth}
\vfill
\begin{itemize}
  \item \textbf{Drones e UxVs}
  \item \textbf{Agricultura de Precisão}
  \item \textbf{Saúde e Reabilitação}
  \item \textbf{Robótica Subaquática}
\end{itemize}
\vfill
\end{columns}
\end{frame}

% ---------- ROS Robots ----------
\begin{frame}{ROS Robots}
\begin{center}
\includegraphics[width=0.85\textwidth]{rosrobots.png}
\end{center}
\begin{center}
\href{https://robots.ros.org/}{\textcolor{UFJFWine}{\textbf{robots.ros.org}}}
\end{center}
\end{frame}

% ---------- Ciclo em Robótica ----------
\begin{frame}{Ciclo de um Processo em Robótica}
\begin{center}
\begin{tikzpicture}[
    box/.style={rectangle, draw=UFJFWine, fill=UFJFLightGray, text width=3.4cm, minimum height=2.2cm, align=center, font=\footnotesize, inner sep=0.1cm},
    arrow/.style={->, thick, UFJFWine, line width=1pt}
]

% Caixas principais
\node[box] (sensoriamento) at (-1.0,0) {
    \textbf{Sensoriamento}\\[0.2cm]
    \begin{itemize}
        \item Câmeras, LiDAR, Radar
        \item IMU, GPS/RTK
        \item Ultrassom, ToF
    \end{itemize}
};

\node[box] (processamento) at (3.2,0) {
    \textbf{Processamento}\\[0.2cm]
    \begin{itemize}
        \item Percepção (Visão, SLAM)
        \item Localização e Mapas
        \item Planejamento e Controle
    \end{itemize}
};

\node[box] (acao) at (7.4,0) {
    \textbf{Atuação}\\[0.2cm]
    \begin{itemize}
        \item Motores/Servos
        \item Manipuladores
        \item Motores de Drones
    \end{itemize}
};

% Setas (mais curtas)
\draw[arrow] (sensoriamento.east) -- (processamento.west);
\draw[arrow] (processamento.east) -- (acao.west);

\end{tikzpicture}
\end{center}
\end{frame}

% ---------- Diagrama de Sensores ----------
\begin{frame}{Sensores, processamento e atuação em um veículo autônomo}
\begin{center}
\includegraphics[width=0.65\textwidth]{autonomous-vehicle-sensors-components-expert-article-diagram.png}
\end{center}
\end{frame}

% ---------- O que é ROS ----------
\section{O que é ROS?}
\begin{frame}{O que é ROS?}
\begin{block}{Robot Operating System}
\begin{itemize}
  \item \textbf{Framework de software} para desenvolvimento de aplicações robóticas.
  \item \textbf{Conjunto de ferramentas}, bibliotecas e convenções.
  \item \textbf{Plataforma distribuída} para comunicação entre processos.
  \item \textbf{Ecossistema} de pacotes e drivers para robótica.
\end{itemize}
\end{block}
\begin{center}
\includegraphics[width=0.6\textwidth]{ROSINTRO.png}
\end{center}
\end{frame}

\begin{frame}{Filosofia do ROS}
\begin{itemize}
  \item \textbf{Peer to peer}: Programas individuais se comunicam através de APIs definidas (mensagens ROS, serviços, etc.).
  \item \textbf{Distribuído}: Programas podem ser executados em múltiplos computadores e se comunicar pela rede.
  \item \textbf{Multi-linguagem}: Módulos ROS podem ser escritos em qualquer linguagem para a qual existe uma biblioteca cliente (C++, Python, MATLAB, Java, etc.).
  \item \textbf{Leve}: Bibliotecas independentes são envolvidas com uma camada fina do ROS.
  \item \textbf{Livre e open-source}: A maioria do software ROS é open-source e gratuito para uso.
\end{itemize}
\end{frame}

\begin{frame}{Características Principais do ROS}
\begin{itemize}
  \item \textbf{Modular}: sistema distribuído de nós independentes.
  \item \textbf{Reutilizável}: bibliotecas e ferramentas compartilhadas.
  \item \textbf{Multi-linguagem}: C++, Python, Java, etc.
  \item \textbf{Multi-plataforma}: Linux, Windows, macOS.
  \item \textbf{Open Source}: desenvolvimento colaborativo.
\end{itemize}
\end{frame}

% ---------- De ROS1 a ROS2 ----------
\section{Do ROS 1 ao ROS 2}
\begin{frame}{Motivações para o ROS 2}
\begin{itemize}
    \item \textbf{ROS 1 não foi projetado} com todos os casos de uso atuais em mente
    \item \textbf{Projetado principalmente para pesquisa}
    \item \textbf{Inicialmente projetado apenas para PR2}
    \item \textbf{Nunca realmente projetado para tempo real}
\end{itemize}
\end{frame}

\begin{frame}{Motivações para o ROS 2 - Novos Usos}
\begin{itemize}
    \item \textbf{Equipes de robôs}
    \item \textbf{Não adequado para tipos de microcontroladores bare-metal}
    \item \textbf{Comunicação sem ser em tempo real}
    \item \textbf{Redes com perda de dados}
    \item \textbf{Usos em pesquisa \& indústria} (ex: certificação)
    \item \textbf{Redesign da API}
\end{itemize}
\end{frame}

\begin{frame}{Principais Diferenças (ROS 1 vs ROS 2)}
\begin{itemize}
  \item \textbf{Suporte para outros SOs}: ROS 2 funciona em Windows, macOS e Linux
  \item \textbf{Pelo menos C++11 e Python 3}: (C++03 antes do Noetic e Python 2 no ROS 1)
  \item \textbf{Uso de middleware pronto}: (em oposição ao middleware customizado)
  \item \textbf{Integração mais próxima com Python}: (ex: arquivos de launch são em Python)
  \item \textbf{Nós em tempo real}
  \item \textbf{Mudanças técnicas internas}: menos mudanças conceituais
\end{itemize}
\end{frame}





% ---------- DDS ----------
\begin{frame}{DDS (Data Distribution Service)}
\begin{columns}[T,onlytextwidth]
\column{.4\textwidth}
\begin{itemize}
  \item \textbf{Padrão} que permite troca de dados usando um padrão publish-subscribe
  \item \textbf{Middleware} que fornece alto desempenho, escalável e forma segura para nós trocarem dados e se comunicarem
  \item \textbf{Usado como camada de comunicação} para ROS 2
\end{itemize}
\column{.6\textwidth}
\begin{center}
\includegraphics[width=0.55\textwidth]{rosstructure.png}
\end{center}
\end{columns}
\end{frame}

% ---------- Arquitetura ROS2 / DDS ----------
\section{Arquitetura DDS do ROS 2}
\begin{frame}{Arquitetura DDS do ROS 2}
\begin{itemize}
  \item \textbf{DDS (Data Distribution Service)}: padrão OMG para sistemas distribuídos.
  \item \textbf{Descoberta automática}: nós se encontram dinamicamente.
  \item \textbf{QoS configurável}: confiabilidade, durabilidade, deadlines.
  \item \textbf{Transporte flexível}: TCP, UDP, shared memory, etc.
  % DDS diagram image
  \begin{center}
  \includegraphics[width=0.45\textwidth]{dds.png}
  \end{center}
\end{itemize}
\end{frame}

\begin{frame}{Camadas de Software}
\begin{itemize}
  \item \textbf{Aplicação}: nós (C++ \texttt{rclcpp}, Python \texttt{rclpy}), launch, parâmetros.
  \item \textbf{ROS 2 Client Libraries}: rclcpp, rclpy.
  \item \textbf{RMW}: abstração do middleware DDS.
  \item \textbf{DDS}: descoberta, \emph{pub/sub}, QoS, \emph{transport}.
\end{itemize}
\end{frame}

% ---------- Evolução e História ----------

\section{Evolução, História e Distros}
\begin{frame}
	\frametitle{História do ROS}
	\begin{itemize}
		\scriptsize
		\item \href{https://www.linkedin.com/in/eric-berger-806b3b3/}{Eric Berger} e {Keenan Wyrobek} começaram o Doutorado (Ph.D) em Stanford...
		\item Buscaram levantar fundos para o desenvolvimento do projeto o Linux da Robótica.
		\item PR2 (Personal Robotics) - \href{https://www.youtube.com/watch?v=oyHWkQcin7I&feature=youtu.be}{\underline{ver vídeo}}
		\item Atualmente: Ecossistema ROS - Qualquer grupo pode iniciar um repositório de código do ROS ("\textit{federated model}")
	\end{itemize}
			
	\begin{table}
		\begin{tabular}{ | >{\centering\arraybackslash}m{1cm}  >{\centering\arraybackslash}m{7cm} | }  
			\hline
			\includegraphics[width=0.6cm,height=0.6cm]{stanford.png} & Stanford Personal Robotics Program (janeiro de 2007)\\
			\hline
			\includegraphics[width=0.6cm,height=0.6cm]{willowgarage.jpg} & Laboratório de pesquisa em robótica e incubadora tecnológica (Novembro de 2007) \\
			\hline
			\includegraphics[width=0.6cm,height=0.6cm]{openrobotics.png} & Open Source Robotics Foundation (OSRF) ou Open Robotics (Fevereiro de 2013)\\
			\hline
		\end{tabular}
	\end{table}
	
	\centering
	\scriptsize
	"Don't let anyone crush your crazy" ou "Não deixe ninguém esmagar sua loucura"
\end{frame}


\begin{frame}
	\frametitle{RosCon - ROS Conference (\href{https://roscon.ros.org/2024/} {https://roscon.ros.org/2024/})} 
        \begin{columns}
            \begin{column}{0.5\textwidth}
        	\centering
                \only<1>
                {
    	          \includegraphics[width=0.75\textwidth]{ROSCon1.png} \\~\\
                }
                \only<2>
                {
    	          \includegraphics[width=6cm,height=3.5cm]{ROSCon2.png} \\~\\
                }
            \end{column}
            \begin{column}{0.5\textwidth}
        	\scriptsize
        	Conjugada com o IROS (ou ICRA):
        	\begin{itemize}
            		\item \href{https://www.iros25.org/}{IEEE/RSJ International Conference on Intelligent Robots and Systems}
				\item \href{https://2025.ieee-icra.org/}{International Conference on Robotics and Automation}
            	\end{itemize}
            \end{column}
        \end{columns}
\end{frame}

---------- Distribuições ----------
\begin{frame}{Distribuições ROS}
\begin{center}
\includegraphics[width=0.85\textwidth]{rosdistros.png}
\end{center}
\end{frame}

% \begin{frame}
	\frametitle{Distribuições do ROS 1}
	\newcommand{\cor}{\cellcolor{red!50}}
	\only<1>
	{
	\begin{table}
		\tiny
		\begin{tabular}{| c |  c | c | c | c | c | c | c | c | c | c | c |  c |}
			\hline
			\cor 2020 & 2021 & 2022 & 2023 & 2012 & 2012 & 2013 & 2014 & 2015 & 2016 & 2017 & 2018 & 2020 \\
			\hline
			\cor Mar  & Ago  & Mar  & Ago  & Abr  & Dez  & Set  & Jul  & Mai  & May  & May  & May  & May \\
			\cor 02   & 02   & 02   & 30   & 23   & 31   & 04   & 22   & 23   & 23   & 23   & 23   & 23  \\
			\hline
		\end{tabular}
	\end{table}
	}
	\only<2>
	{
	\begin{table}
		\tiny
		\begin{tabular}{| c |  c | c | c | c | c | c | c | c | c | c | c |  c |}
			\hline
			2010 & \cor 2010 & 2011 & 2011 & 2012 & 2012 & 2013 & 2014 & 2015 & 2016 & 2017 & 2018 & 2020\\
			\hline
			Mar  & \cor Ago  & Mar  & Ago  & Abr  & Dez  & Set  & Jul  & Mai  & May  & May  & May &  May \\
			02   & \cor 02   & 02   & 30   & 23   & 31   & 04   & 22   & 23   & 23   & 23   & 23    & 23   \\
			\hline
		\end{tabular}
	\end{table}
	}
	\only<3>
	{
	\begin{table}
		\tiny
		\begin{tabular}{| c |  c | c | c | c | c | c | c | c | c | c | c |  c |}
			\hline
			2010 & 2010 & \cor 2011 & 2011 & 2012 & 2012 & 2013 & 2014 & 2015 & 2016 & 2017 & 2018 & 2020 \\
			\hline
			Mar  & Ago  & \cor Mar  & Ago  & Abr  & Dez  & Set  & Jul  & Mai  & May  & May  & May &  May  \\
			02   & 02   & \cor 02   & 30   & 23   & 31   & 04   & 22   & 23   & 23   & 23   & 23    & 23   \\
			\hline
		\end{tabular}
	\end{table}
	}
	\only<4>
	{
	\begin{table}
		\tiny
		\begin{tabular}{| c |  c | c | c | c | c | c | c | c | c | c | c |  c |}
			\hline
			2010 & 2010 & 2011 & \cor 2011 & 2012 & 2012 & 2013 & 2014 & 2015 & 2016 & 2017 & 2018 & 2020 \\
			\hline
			Mar  & Ago  & Mar  & \cor Ago  & Abr  & Dez  & Set  & Jul  & Mai  & May  & May  & May &  May  \\
			02   & 02   & 02   & \cor 30   & 23   & 31   & 04   & 22   & 23   & 23   & 23   & 23    & 23   \\
			\hline
		\end{tabular}
	\end{table}
	}
	\only<5>
	{
	\begin{table}
		\tiny
		\begin{tabular}{| c |  c | c | c | c | c | c | c | c | c | c | c |  c |}
			\hline
			2010 & 2010 & 2011 & 2011 & \cor 2012 & 2012 & 2013 & 2014 & 2015 & 2016 & 2017 & 2018  & 2020\\
			\hline
			Mar  & Ago  & Mar  & Ago  & \cor Abr  & Dez  & Set  & Jul  & Mai  & May  & May  & May &  May  \\
			02   & 02   & 02   & 30   & \cor 23   & 31   & 04   & 22   & 23   & 23   & 23   & 23    & 23   \\
			\hline
		\end{tabular}
	\end{table}
	}
	\only<6>
	{
	\begin{table}
		\tiny
		\begin{tabular}{| c |  c | c | c | c | c | c | c | c | c | c | c |  c |}
			\hline
			2010 & 2010 & 2011 & 2011 & 2012 & \cor 2012 & 2013 & 2014 & 2015 & 2016 & 2017 & 2018  & 2020\\
			\hline
			Mar  & Ago  & Mar  & Ago  & Abr  & \cor Dez  & Set  & Jul  & Mai  & May  & May  & May  &  May \\
			02   & 02   & 02   & 30   & 23   & \cor 31   & 04   & 22   & 23   & 23   & 23   & 23    & 23   \\
			\hline
		\end{tabular}
	\end{table}
	}
	\only<7>
	{
	\begin{table}
		\tiny
		\begin{tabular}{| c |  c | c | c | c | c | c | c | c | c | c | c |  c |}
			\hline
			2010 & 2010 & 2011 & 2011 & 2012 & 2012 & \cor 2013 & 2014 & 2015 & 2016 & 2017 & 2018  & 2020\\
			\hline
			Mar  & Ago  & Mar  & Ago  & Abr  & Dez  & \cor Set  & Jul  & Mai  & May  & May  & May  &  May \\
			02   & 02   & 02   & 30   & 23   & 31   & \cor 04   & 22   & 23   & 23   & 23   & 23    & 23   \\
			\hline
		\end{tabular}
	\end{table}
	}
	\only<8>
	{
	\begin{table}
		\tiny
		\begin{tabular}{| c |  c | c | c | c | c | c | c | c | c | c | c |  c |}
			\hline
			2010 & 2010 & 2011 & 2011 & 2012 & 2012 & 2013 & \cor 2014 & 2015 & 2016 & 2017 & 2018  & 2020\\
			\hline
			Mar  & Ago  & Mar  & Ago  & Abr  & Dez  & Set  & \cor Jul  & Mai  & May  & May  & May &  May  \\
			02   & 02   & 02   & 30   & 23   & 31   & 04   & \cor 22   & 23   & 23   & 23   & 23   & 23   \\
			\hline
		\end{tabular}
	\end{table}
	}
	\only<9>
	{
	\begin{table}
		\tiny
		\begin{tabular}{| c |  c | c | c | c | c | c | c | c | c | c | c |  c |}
			\hline
			2010 & 2010 & 2011 & 2011 & 2012 & 2012 & 2013 & 2014 & \cor 2015 & 2016 & 2017 & 2018  & 2020\\
			\hline
			Mar  & Ago  & Mar  & Ago  & Abr  & Dez  & Set  & Jul  & \cor Mai  & May  & May  & May  &  May \\
			02   & 02   & 02   & 30   & 23   & 31   & 04   & 22   & \cor 23   & 23   & 23   & 23   &  23 \\
			\hline
		\end{tabular}
	\end{table}
	}
	\only<10>
	{
	\begin{table}
		\tiny
		\begin{tabular}{| c |  c | c | c | c | c | c | c | c | c | c | c |  c |}
			\hline
			2010 & 2010 & 2011 & 2011 & 2012 & 2012 & 2013 & 2014 & 2015 & \cor 2016 & 2017 & 2018  & 2020\\
			\hline
			Mar  & Ago  & Mar  & Ago  & Abr  & Dez  & Set  & Jul  & Mai  & \cor May  & May  & May &  May  \\
			02   & 02   & 02   & 30   & 23   & 31   & 04   & 22   & 23   & \cor 23   & 23   & 23   & 23    \\
			\hline
		\end{tabular}
	\end{table}
	}
	\only<11>
	{
	\begin{table}
		\tiny
		\begin{tabular}{| c |  c | c | c | c | c | c | c | c | c | c | c |  c |}
			\hline
			2010 & 2010 & 2011 & 2011 & 2012 & 2012 & 2013 & 2014 & 2015 & 2016 & \cor 2017 & 2018  & 2020\\
			\hline
			Mar  & Ago  & Mar  & Ago  & Abr  & Dez  & Set  & Jul  & Mai  & May  & \cor May  & May  &  May \\
			02   & 02   & 02   & 30   & 23   & 31   & 04   & 22   & 23   & 23   & \cor 23   & 23    & 23   \\
			\hline
		\end{tabular}
	\end{table}
	}
	\only<12>
	{
	\begin{table}
		\tiny
		\begin{tabular}{| c |  c | c | c | c | c | c | c | c | c | c | c |  c |}
			\hline
			2010 & 2010 & 2011 & 2011 & 2012 & 2012 & 2013 & 2014 & 2015 & 2016 & 2017 & \cor 2018  & 2020\\
			\hline
			Mar  & Ago  & Mar  & Ago  & Abr  & Dez  & Set  & Jul  & Mai  & May  & May  & \cor May &  May  \\
			02   & 02   & 02   & 30   & 23   & 31   & 04   & 22   & 23   & 23   & 23   & \cor 23   & 23    \\
			\hline
		\end{tabular}
	\end{table}
	}
	\only<13>
	{
	\begin{table}
		\tiny
		\begin{tabular}{| c |  c | c | c | c | c | c | c | c | c | c | c |  c |}
			\hline
			2010 & 2010 & 2011 & 2011 & 2012 & 2012 & 2013 & 2014 & 2015 & 2016 & 2017 &  2018  & \cor 2020\\
			\hline
			Mar  & Ago  & Mar  & Ago  & Abr  & Dez  & Set  & Jul  & Mai  & May  & May  & May    & \cor May \\
			02   & 02   & 02   & 30   & 23   & 31   & 04   & 22   & 23   & 23   & 23   & 23     & \cor 23\\
			\hline
		\end{tabular}
	\end{table}
	}

	\only<1>
	{
		\textbf{\href{http://wiki.ros.org/boxturtle}{\color{red} ROS \color{blue}B\color{red}ox Turtle}} \\~\\
		\begin{columns}
			\begin{column}{0.3\textwidth}
				\centering
				\includegraphics[width=3cm,height=3cm]{BoxTurtle_Poster.png}
			\end{column}
			\begin{column}{0.3\textwidth}
				\centering
				\includegraphics[width=1cm,height=1cm]{BoxTurtle.png}
			\end{column}
			\begin{column}{0.3\textwidth}
	`			\centering
			\end{column}		
		\end{columns}
	}
	\only<2>
	{
		\textbf{\href{http://wiki.ros.org/cturtle}{\color{red} ROS \color{blue}C \color{red} Turtle}}\\~\\
		\begin{columns}
			\centering
			\begin{column}{0.3\textwidth}
				\includegraphics[width=3cm,height=3cm]{CTurtle_Poster.jpg}
			\end{column}
			\begin{column}{0.3\textwidth}
				\centering
				\includegraphics[width=1cm,height=1cm]{CTurtle.png}
			\end{column}
			\begin{column}{0.3\textwidth}
				\centering
			\end{column}		
		\end{columns}
	}
	\only<3>
	{
		\textbf{\href{http://wiki.ros.org/diamondback}{\color{red} ROS \color{blue}D\color{red}iamondback}}\\~\\
		\begin{columns}
			\begin{column}{0.3\textwidth}
				\centering
				\includegraphics[width=3cm,height=3cm]{Diamondback_Poster.jpg}
			\end{column}
			\begin{column}{0.3\textwidth}
				\centering
				\includegraphics[width=1cm,height=1cm]{Diamondback.png}
			\end{column}
			\begin{column}{0.3\textwidth}
				\centering
			\end{column}		
		\end{columns}
	}
	\only<4>
	{
		\textbf{\href{http://wiki.ros.org/electric}{\color{red} ROS \color{blue}E\color{red}lectric Emys}}\\~\\
		\begin{columns}
			\begin{column}{0.3\textwidth}
				\centering
				\includegraphics[width=3cm,height=3cm]{ElectricEmys_Poster.png}
			\end{column}
			\begin{column}{0.3\textwidth}
				\centering
				\includegraphics[width=1cm,height=1cm]{ElectricEmys.png}
			\end{column}
			\begin{column}{0.3\textwidth}
				\centering
			\end{column}		
		\end{columns}
	}
	\only<5>
	{
		\textbf{\href{http://wiki.ros.org/fuerte}{\color{red} ROS \color{blue}F\color{red}uerte Turtle}}\\~\\
		\begin{columns}
			\begin{column}{0.3\textwidth}
				\centering
				\includegraphics[width=3cm,height=3cm]{FuerteTurtle_Poster.jpg}
			\end{column}
			\begin{column}{0.3\textwidth}
				\centering
				\includegraphics[width=1cm,height=1cm]{FuerteTurtle.png}
			\end{column}
			\begin{column}{0.3\textwidth}
				\centering
			\end{column}		
		\end{columns}
	}
	\only<6>
	{
		\textbf{\href{http://wiki.ros.org/groovy}{\color{red} ROS \color{blue}G\color{red}roovy Galapagos}}\\~\\
		\begin{columns}
			\begin{column}{0.3\textwidth}
				\centering
				\includegraphics[width=3cm,height=3cm]{GroovyGalapagos_Poster.jpg}
			\end{column}
			\begin{column}{0.3\textwidth}
				\centering
				\includegraphics[width=1cm,height=1cm]{GroovyGalapagos.png}
			\end{column}
			\begin{column}{0.3\textwidth}
				\centering
				\begin{table}
					\tiny
					\begin{tabular}{|c|}
						\hline
						End-of-Life (EOL) \\
						\hline
						Julho, 2014 \\
						\hline
					\end{tabular}
				\end{table}	
			\end{column}		
		\end{columns}
	}
	\only<7>
	{
		\textbf{\href{http://wiki.ros.org/hydro}{\color{red} ROS \color{blue}H\color{red}ydro Medusa}}\\~\\
		\begin{columns}
			\begin{column}{0.3\textwidth}
				\centering
				\includegraphics[width=3cm,height=3cm]{HydroMedusa_Poster.png}
			\end{column}
			\begin{column}{0.3\textwidth}
				\centering
				\includegraphics[width=1cm,height=1cm]{HydroMedusa.png}
			\end{column}
			\begin{column}{0.3\textwidth}
				\centering
				\begin{table}
					\tiny
					\begin{tabular}{|c|}
						\hline
						End-of-Life (EOL) \\
						\hline
						Maio, 2015 \\
						\hline
					\end{tabular}
				\end{table}		
			\end{column}		
		\end{columns}
	}
	\only<8>
	{
		\textbf{\href{http://wiki.ros.org/indigo}{\color{red} ROS \color{blue}I\color{red}ndigo Igloo}}\\~\\
		\begin{columns}
			\begin{column}{0.3\textwidth}
				\centering
				\includegraphics[width=3cm,height=3cm]{IndigoIgloo_Poster.png}
			\end{column}
			\begin{column}{0.3\textwidth}
				\centering
				\includegraphics[width=1cm,height=1cm]{IndigoIgloo.png}
			\end{column}
			\begin{column}{0.3\textwidth}
				\centering
				\begin{table}
					\tiny
					\begin{tabular}{|c|}
						\hline
						End-of-Life (EOL) \\
						\hline
						Abril, 2019 \\
						\hline
					\end{tabular}
				\end{table}		
			\end{column}		
		\end{columns}
	}
	\only<9>
	{
		\textbf{\href{http://wiki.ros.org/jade}{\color{red} ROS \color{blue}J\color{red}ade Turtle}}\\~\\
		\begin{columns}
			\begin{column}{0.3\textwidth}
				\centering
				\includegraphics[width=3cm,height=3cm]{JadeTurtle_Poster.png}
			\end{column}
			\begin{column}{0.3\textwidth}
				\centering
				\includegraphics[width=1cm,height=1cm]{JadeTurtle.png}
			\end{column}
			\begin{column}{0.3\textwidth}
				\centering
				\begin{table}
					\tiny
					\begin{tabular}{|c|}
						\hline
						End-of-Life (EOL) \\
						\hline
						Maio, 2017 \\
						\hline
					\end{tabular}
				\end{table}		
			\end{column}		
		\end{columns}
	}
	\only<10>
	{
		\textbf{\href{http://wiki.ros.org/kinetic}{\color{red} ROS \color{blue}K\color{red}inetic Kame}}\\~\\
		\begin{columns}
			\begin{column}{0.3\textwidth}
				\centering
				\includegraphics[width=3cm,height=3cm]{KineticKame_Poster.png}
			\end{column}
			\begin{column}{0.3\textwidth}
				\centering
				\includegraphics[width=1cm,height=1cm]{KineticKame.png}
			\end{column}
			\begin{column}{0.3\textwidth}
				\centering
				\begin{table}
					\tiny
					\begin{tabular}{|c|}
						\hline
						End-of-Life (EOL) \\
						\hline
						Abril, 2021 \\
						\hline
					\end{tabular}
				\end{table}		
			\end{column}		
		\end{columns}
	}
	\only<11>
	{
		\textbf{\href{http://wiki.ros.org/lunar}{\color{red} ROS \color{blue}L\color{red}unar LoggerHead}}\\~\\
		\begin{columns}
			\begin{column}{0.3\textwidth}
				\centering
				\includegraphics[width=3cm,height=3cm]{LugarLoggerhead_Poster.png}
			\end{column}
			\begin{column}{0.3\textwidth}
				\centering
				\includegraphics[width=1cm,height=1cm]{LugarLoggerhead.png}
			\end{column}
			\begin{column}{0.3\textwidth}
				\centering
				\begin{table}
					\tiny
					\begin{tabular}{|c|}
						\hline
						End-of-Life (EOL) \\
						\hline
						Maio, 2019 \\
						\hline
					\end{tabular}
				\end{table}		
			\end{column}		
		\end{columns}
	}
	\only<12>
	{
		\textbf{\href{http://wiki.ros.org/melodic}{\color{red} ROS \color{blue}M\color{red}elodic Morenia}} \\~\\
		\begin{columns}
			\begin{column}{0.3\textwidth}
				\centering
				\includegraphics[width=3cm,height=3cm]{MelodicMorenia_Poster.png}
			\end{column}
			\begin{column}{0.3\textwidth}
				\centering
				\includegraphics[width=1cm,height=1cm]{MelodicMorenia.png}
			\end{column}
			\begin{column}{0.3\textwidth}
				\centering
				\begin{table}
					\tiny
					\begin{tabular}{|c|}
						\hline
						End-of-Life (EOL) \\
						\hline
						Maio, 2023 \\
						\hline
					\end{tabular}
				\end{table}		
			\end{column}		
		\end{columns}
	}	

	\only<13>
	{
		\textbf{\href{https://wiki.ros.org/noetic}{\color{red} ROS \color{blue}N\color{red}oetic Ninjemys}} \\~\\
		\begin{columns}
			\begin{column}{0.3\textwidth}
				\centering
				\includegraphics[width=3cm,height=3cm]{Noetic.png}
			\end{column}
			\begin{column}{0.3\textwidth}
				\centering
				\includegraphics[width=1cm,height=1cm]{Noetic.png}
			\end{column}
			\begin{column}{0.3\textwidth}
				\centering
				\begin{table}
					\tiny
					\begin{tabular}{|c|}
						\hline
						End-of-Life (EOL) \\
						\hline
						Maio, 2025 \\
						\hline
					\end{tabular}
				\end{table}		
			\end{column}		
		\end{columns}
	}	

\end{frame}

% \begin{frame}
	\frametitle{Distribuições do ROS 2}
        
	\newcommand{\cor}{\cellcolor{red!50}}
	\only<1>
	{
	\begin{table}
		\tiny
		\begin{tabular}{| c |  c | c | c | c | c | c | c | c | c | }
			\hline
			\cor 2017 & 2018 & 2018 & 2019 & 2019 & 2020 & 2021 & 2022 & 2023 & 2024  \\
			\hline
			\cor Dec  & Jul  & Dec  & May  & Nov  & Jun  & May  & May  & May  & May   \\
			\cor 08   & 02   & 14   & 31   & 22   & 05   & 23   & 23   & 23   & 23     \\
			\hline
		\end{tabular}
	\end{table}
	}
	\only<2>
	{
	\begin{table}
		\tiny
		\begin{tabular}{| c |  c | c | c | c | c | c | c | c | c | }
			\hline
			2017 & \cor 2018 & 2018 & 2019 & 2019 & 2020 & 2021 & 2022 & 2023 & 2024 \\
			\hline
			Dec  & \cor Jul  & Dec  & May  & Nov  & Jun  & May  & May  & May  & May   \\
			08   & \cor 02   & 14   & 31   & 22   & 05   & 23   & 23   & 23   & 23     \\
			\hline
		\end{tabular}
	\end{table}
	}
	\only<3>
	{
	\begin{table}
		\tiny
		\begin{tabular}{| c |  c | c | c | c | c | c | c | c | c | }
			\hline
			2017 & 2018 & \cor 2018 & 2019 & 2019 & 2020 & 2021 & 2022 & 2023 & 2024 \\
			\hline
			Dec  & Jul  & \cor Dec  & May  & Nov  & Jun  & May  & May  & May  & May   \\
			08   & 02   & \cor 14   & 31   & 22   & 05   & 23   & 23   & 23   & 23     \\
			\hline
		\end{tabular}
	\end{table}
	}
	\only<4>
	{
	\begin{table}
		\tiny
		\begin{tabular}{| c |  c | c | c | c | c | c | c | c | c | }
			\hline
			2017 & 2018 & 2018 & \cor 2019 & 2019 & 2020 & 2021 & 2022 & 2023 & 2024 \\
			\hline
			Dec  & Jul  & Dec  & \cor May  & Nov  & Jun  & May  & May  & May  & May   \\
			08   & 02   & 14   & \cor 31   & 22   & 05   & 23   & 23   & 23   & 23     \\
			\hline
		\end{tabular}
	\end{table}
	}
	\only<5>
	{
	\begin{table}
		\tiny
		\begin{tabular}{| c |  c | c | c | c | c | c | c | c | c | }
			\hline
			2017 & 2018 & 2018 & 2019 & \cor 2019 & 2020 & 2021 & 2022 & 2023 & 2024 \\
			\hline
			Dec  & Jul  & Dec  & May  & \cor Nov  & Jun  & May  & May  & May  & May   \\
			08   & 02   & 14   & 31   & \cor 22   & 05   & 23   & 23   & 23   & 23     \\
			\hline
		\end{tabular}
	\end{table}
	}
	\only<6>
	{
	\begin{table}
		\tiny
		\begin{tabular}{| c |  c | c | c | c | c | c | c | c | c | }
			\hline
			2017 & 2018 & 2018 & 2019 & 2019 & \cor 2020 & 2021 & 2022 & 2023 & 2024 \\
			\hline
			Dec  & Jul  & Dec  & May  & Nov  & \cor Jun  & May  & May  & May  & May   \\
			08   & 02   & 14   & 31   & 22   & \cor 05   & 23   & 23   & 23   & 23     \\
			\hline
		\end{tabular}
	\end{table}
	}
	\only<7>
	{
	\begin{table}
		\tiny
		\begin{tabular}{| c |  c | c | c | c | c | c | c | c | c | }
			\hline
			2017 &  2018 & 2018 & 2019 & 2019 & 2020 & \cor 2021 & 2022 & 2023 & 2024 \\
			\hline
			Dec  & Jul  & Dec  & May  & Nov  & Jun  & \cor May  & May  & May  & May   \\
			08   & 02   & 14   & 31   & 22   & 05   & \cor 23   & 23   & 23   & 23     \\
			\hline
		\end{tabular}
	\end{table}
	}
	\only<8>
	{
	\begin{table}
		\tiny
		\begin{tabular}{| c |  c | c | c | c | c | c | c | c | c | }
			\hline
			2017 &  2018 & 2018 & 2019 & 2019 & 2020 & 2021 & \cor 2022 & 2023 & 2024 \\
			\hline
			Dec  & Jul  & Dec  & May  & Nov  & Jun  & May  & \cor May  & May  & May   \\
			08   & 02   & 14   & 31   & 22   & 05   & 23   & \cor 23   & 23   & 23     \\
			\hline
		\end{tabular}
	\end{table}
	}
	\only<9>
	{
	\begin{table}
		\tiny
		\begin{tabular}{| c |  c | c | c | c | c | c | c | c | c | }
			\hline
			2017 &  2018 & 2018 & 2019 & 2019 & 2020 & 2021 & 2022 & \cor 2023 & 2024 \\
			\hline
			Dec  & Jul  & Dec  & May  & Nov  & Jun  & May  & May  & \cor May  & May   \\
			08   & 02   & 14   & 31   & 22   & 05   & 23   & 23   & \cor 23   & 23     \\
			\hline
		\end{tabular}
	\end{table}
	}
	\only<10>
	{
	\begin{table}
		\tiny
		\begin{tabular}{| c |  c | c | c | c | c | c | c | c | c | }
			\hline
			2017 & 2018 & 2018 & 2019 & 2019 & 2020 & 2021 & 2022 & 2023 & \cor  2024 \\
			\hline
			Dec  & Jul  & Dec  & May  & Nov  & Jun  & May  & May  & May  & \cor May   \\
			08   & 02   & 14   & 31   & 22   & 05   & 23   & 23   & 23   & \cor 23     \\
			\hline
		\end{tabular}
	\end{table}
	}

	\only<1>
	{
		\textbf{\href{https://docs.ros.org/en/jazzy/Releases/Release-Ardent-Apalone.html}{\color{red} ROS \color{blue}A\color{red}rdent Apalone}} \\~\\
		\begin{columns}
			\begin{column}{0.3\textwidth}
				\centering
				\includegraphics[width=3cm,height=3cm]{Ardent Apalone.png}
			\end{column}
			\begin{column}{0.3\textwidth}
				\centering
			\end{column}
			\begin{column}{0.3\textwidth}
				\centering
				\begin{table}
					\tiny
					\begin{tabular}{|c|}
						\hline
						End-of-Life (EOL) \\
						\hline
						December, 2018 \\
						\hline
					\end{tabular}
				\end{table}	

			\end{column}		
		\end{columns}
	}
	\only<2>
	{
		\textbf{\href{https://docs.ros.org/en/jazzy/Releases/Release-Bouncy-Bolson.html}{\color{red} ROS \color{blue}B\color{red}ouncy Bolson}}\\~\\
		\begin{columns}
			\centering
			\begin{column}{0.3\textwidth}
				\includegraphics[width=3cm,height=3cm]{Bouncy Bolson.png}
			\end{column}
			\begin{column}{0.3\textwidth}
				\centering
			\end{column}
			\begin{column}{0.3\textwidth}
				\centering
				\begin{table}
					\tiny
					\begin{tabular}{|c|}
						\hline
						End-of-Life (EOL) \\
						\hline
						July, 2019 \\
						\hline
					\end{tabular}
				\end{table}	

			\end{column}		
		\end{columns}
	}
	\only<3>
	{
		\textbf{\href{https://docs.ros.org/en/jazzy/Releases/Release-Crystal-Clemmys.html}{\color{red} ROS \color{blue}C\color{red}rystal Clemmys}}\\~\\
		\begin{columns}
			\begin{column}{0.3\textwidth}
				\centering
				\includegraphics[width=3cm,height=3cm]{Crystal Clemmys.png}
			\end{column}
			\begin{column}{0.3\textwidth}
				\centering
			\end{column}
			\begin{column}{0.3\textwidth}
				\centering
				\begin{table}
					\tiny
					\begin{tabular}{|c|}
						\hline
						End-of-Life (EOL) \\
						\hline
						December, 2019 \\
						\hline
					\end{tabular}
				\end{table}	

			\end{column}		
		\end{columns}
	}
	\only<4>
	{
		\textbf{\href{https://docs.ros.org/en/jazzy/Releases/Release-Dashing-Diademata.html}{\color{red} ROS \color{blue}D\color{red}ashing Diademata}}\\~\\
		\begin{columns}
			\begin{column}{0.3\textwidth}
				\centering
				\includegraphics[width=3cm,height=3cm]{Dashing Diademata.png}
			\end{column}
			\begin{column}{0.3\textwidth}
				\centering
			\end{column}
			\begin{column}{0.3\textwidth}
				\centering
				\begin{table}
					\tiny
					\begin{tabular}{|c|}
						\hline
						End-of-Life (EOL) \\
						\hline
						May, 2021 \\
						\hline
					\end{tabular}
				\end{table}	
			\end{column}		
		\end{columns}
	}
	\only<5>
	{
		\textbf{\href{https://docs.ros.org/en/jazzy/Releases/Release-Eloquent-Elusor.html}{\color{red} ROS \color{blue}E\color{red}loquent Elusor}}\\~\\
		\begin{columns}
			\begin{column}{0.3\textwidth}
				\centering
				\includegraphics[width=3cm,height=3cm]{Eloquent Elusor.png}
			\end{column}
			\begin{column}{0.3\textwidth}
				\centering
			\end{column}
			\begin{column}{0.3\textwidth}
				\centering
				\begin{table}
					\tiny
					\begin{tabular}{|c|}
						\hline
						End-of-Life (EOL) \\
						\hline
						November, 2020 \\
						\hline
					\end{tabular}
				\end{table}	
			\end{column}		
		\end{columns}
	}
	\only<6>
	{
		\textbf{\href{https://docs.ros.org/en/jazzy/Releases/Release-Foxy-Fitzroy.html}{\color{red} ROS \color{blue}F\color{red}oxy Fitzroy}}\\~\\
		\begin{columns}
			\begin{column}{0.3\textwidth}
				\centering
				\includegraphics[width=3cm,height=3cm]{Foxy Fitzroy.png}
			\end{column}
			\begin{column}{0.3\textwidth}
				\centering
			\end{column}
			\begin{column}{0.3\textwidth}
				\centering
				\begin{table}
					\tiny
					\begin{tabular}{|c|}
						\hline
						End-of-Life (EOL) \\
						\hline
						June, 2023 \\
						\hline
					\end{tabular}
				\end{table}	
			\end{column}		
		\end{columns}
	}
	\only<7>
	{
		\textbf{\href{https://docs.ros.org/en/jazzy/Releases/Release-Galactic-Geochelone.html}{\color{red} ROS \color{blue}G\color{red}alactic Geochelone}}\\~\\
		\begin{columns}
			\begin{column}{0.3\textwidth}
				\centering
				\includegraphics[width=3cm,height=3cm]{Galactic Geochelone.png}
			\end{column}
			\begin{column}{0.3\textwidth}
				\centering
			\end{column}
			\begin{column}{0.3\textwidth}
				\centering
				\begin{table}
					\tiny
					\begin{tabular}{|c|}
						\hline
						End-of-Life (EOL) \\
						\hline
						December, 2022 \\
						\hline
					\end{tabular}
				\end{table}		
			\end{column}		
		\end{columns}
	}
	\only<8>
	{
		\textbf{\href{https://docs.ros.org/en/jazzy/Releases/Release-Humble-Hawksbill.html}{\color{red} ROS \color{blue}H\color{red}umble Hawksbill}}\\~\\
		\begin{columns}
			\begin{column}{0.3\textwidth}
				\centering
				\includegraphics[width=3cm,height=3cm]{Humble Hawksbill.png}
			\end{column}
			\begin{column}{0.3\textwidth}
				\centering
			\end{column}
			\begin{column}{0.3\textwidth}
				\centering
				\begin{table}
					\tiny
					\begin{tabular}{|c|}
						\hline
						End-of-Life (EOL) \\
						\hline
						May, 2027 \\
						\hline
					\end{tabular}
				\end{table}		
			\end{column}		
		\end{columns}
	}
	\only<9>
	{
		\textbf{\href{https://docs.ros.org/en/jazzy/Releases/Release-Iron-Irwini.html}{\color{red} ROS \color{blue}I\color{red}ron Irwini}}\\~\\
		\begin{columns}
			\begin{column}{0.3\textwidth}
				\centering
				\includegraphics[width=3cm,height=3cm]{Iron Irwini.png}
			\end{column}
			\begin{column}{0.3\textwidth}
				\centering
			\end{column}
			\begin{column}{0.3\textwidth}
				\centering
				\begin{table}
					\tiny
					\begin{tabular}{|c|}
						\hline
						End-of-Life (EOL) \\
						\hline
						December, 2024 \\
						\hline
					\end{tabular}
				\end{table}		
			\end{column}		
		\end{columns}
	}
	\only<10>
	{
		\textbf{\href{https://docs.ros.org/en/jazzy/Releases/Release-Jazzy-Jalisco.html}{\color{red} ROS \color{blue}J\color{red}azzy Jalisco}}\\~\\
		\begin{columns}
			\begin{column}{0.3\textwidth}
				\centering
				\includegraphics[width=3cm,height=3cm]{Jazzy Jalisco.png}
			\end{column}
			\begin{column}{0.3\textwidth}
				\centering
			\end{column}
			\begin{column}{0.3\textwidth}
				\centering
				\begin{table}
					\tiny
					\begin{tabular}{|c|}
						\hline
						End-of-Life (EOL) \\
						\hline
						May, 2029 \\
						\hline
					\end{tabular}
				\end{table}		
			\end{column}		
		\end{columns}
	}


\end{frame}
% \section{Virtual Box}

\begin{frame}{Passo a Passo - Virtual Box}
    \begin{itemize}
        \item Instalar o VirtualBox no Windows (Host)
        \item Configurar uma máquina virtual para rodar o Ubuntu (Guest)
        \item Preparar o ambiente para desenvolvimento em ROS
    \end{itemize}
\end{frame}

\begin{frame}{Passo 1 - Baixar o VirtualBox}
    \begin{itemize}
        \item Acesse o site oficial: \texttt{https://www.virtualbox.org/}
        \item Clique em \textbf{Downloads}
        \item Selecione a versão para Windows hosts
    \end{itemize}
    \includegraphics[width=0.7\textwidth]{vitrual_box.png} 
\end{frame}

\begin{frame}{Passo 2 - Instalar o VirtualBox}
    \begin{itemize}
        \item Execute o instalador baixado
        \item Siga as instruções padrão (Next, Next, Install)
        \item Permita instalação dos drivers de rede quando solicitado
    \end{itemize}
\end{frame}

\begin{frame}{Passo 3 - Baixar a imagem do Ubuntu}
    \begin{itemize}
        \item Acesse: \texttt{https://ubuntu.com/download/desktop}
        \item Baixe o arquivo \texttt{.iso} da versão desejada (recomendado: LTS)
        \item Compatível com o ROS Noetic: Ubuntu 20.04.6 LTS (Focal Fossa)
    \end{itemize}
    \includegraphics[width=0.7\textwidth]{ubuntu.png}
    \end{frame}

\begin{frame}{Passo 4 - Criar Máquina Virtual}
    \begin{itemize}
        \item Abra o VirtualBox e clique em \textbf{Novo}
        \item Nome: Ubuntu
        \item Tipo: Linux
        \item Versão: Ubuntu (64-bit)
    \end{itemize}
\end{frame}

\begin{frame}{Passo 5 - Configurar a Máquina Virtual}
    \begin{itemize}
        \item Memória RAM: recomendado pelo menos 4096 MB
        \item Disco rígido virtual: \textbf{Criar novo}, formato VDI, dinamicamente alocado
        \item Tamanho do disco: pelo menos 20 GB
    \end{itemize}
\end{frame}

\begin{frame}{Passo 6 - Instalar o Ubuntu}
    \begin{itemize}
        \item Inicie a máquina virtual
        \item Selecione o arquivo \texttt{.iso} do Ubuntu como mídia de boot
        \item Siga o assistente de instalação do Ubuntu
        \item Defina um nome de usuário e senha
    \end{itemize}
\end{frame}

\begin{frame}{Dicas Finais}
    \begin{itemize}
        \item Instale o \textbf{Guest Additions} para melhor integração (opcional)
        \item Ative o \textbf{modo de tela cheia} para melhor experiência
        \item Salve estados da máquina para retomá-la rapidamente
    \end{itemize}
\end{frame}

\begin{frame}{Resumo}
    \begin{itemize}
        \item VirtualBox instalado no Windows
        \item Ubuntu instalado como máquina virtual
        \item Ambiente pronto para aplicações de Robótica Móvel
        \item Existem máquinas virtuais já com ROS Noetic e outras versões previamente instalados
    \end{itemize}
\end{frame}


% ---------- Referências ----------
\begin{frame}{Referências e Recursos}
\begin{itemize}
  \item \textbf{Site}: \url{https://www.ros.org/}
  \item \textbf{Documentação ROS 2}: \url{https://docs.ros.org/}
  \item \textbf{Tutorials e Examples}: \url{https://docs.ros.org/en/ros2\_packages.html}
  \item \textbf{Awesome ROS 2 (curadoria de pacotes)}: buscar por ``awesome-ros2'' no GitHub.
\end{itemize}
\end{frame}

% ---------- Encerramento ----------
\begin{frame}
\centering
\Large Dúvidas? \\
\bigskip
\small \texttt{andre.marcato@ufjf.br} \quad
\end{frame}

