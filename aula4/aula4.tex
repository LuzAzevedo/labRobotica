% ---------- Metadata ----------
\title[Aula 4]{Aula 4 - ROS 2 (Python) — Publishers \& Subscribers}
\author[Marcato]{Professor: André L. Marcato}
\institute[UFJF]{Universidade Federal de Juiz de Fora \\ Engenharia Elétrica — Robótica e Automação Industrial}

% ---------- Title ----------
\begin{frame}
  \titlepage
\end{frame}

% ---------- Overview ----------
\begin{frame}{Roteiro}
\tableofcontents
\end{frame}

% ---------- Conteúdo ----------
\section{Conceitos Fundamentais}

% --- Slide 1: Publishers & Subscribers — conceitos ---
\begin{frame}[fragile]{\href{https://docs.ros.org/en/humble/Tutorials/Beginner-Client-Libraries/Writing-A-Simple-Py-Publisher-And-Subscriber.html}{\uline{ROS 2 (Python) — Publishers \& Subscribers}}}
    \begin{itemize}
      \item \textbf{Publisher}: node que \textbf{envia} mensagens para um \textbf{tópico}.
      \item \textbf{Subscriber}: node que \textbf{recebe} mensagens de um \textbf{tópico}.
      \item Desacoplamento por \textbf{tópicos} (pub/sub); múltiplos publishers/subscribers por tópico.
      \item Comunicação \textbf{assíncrona} baseada em callbacks; qualidade controlada por \textbf{QoS}.
    \end{itemize}
  \end{frame}
  
  % --- Slide 2: Nó mínimo (rclpy) ---
  \begin{frame}[fragile]{Nó mínimo em \texttt{rclpy}}
    \begin{itemize}
      \item Elementos: \texttt{rclpy.init()}, classe \texttt{Node}, \texttt{rclpy.spin(node)}, \texttt{rclpy.shutdown()}.
      \item Componentes comuns: publisher(es), subscriber(s), \textbf{timers}, \textbf{callbacks}, \textbf{logs}.
      \item Encapsulamos lógica em uma subclasse de \texttt{Node}.
    \end{itemize}
  
  \begin{center}
  \includegraphics[width=0.95\textwidth]{code1_minimal.png}
  \end{center}
  \end{frame}
  
  % --- Slide 3: Timers, callbacks e logs ---
  \begin{frame}[fragile]{Timers, callbacks e logs}
    \begin{itemize}
      \item \textbf{Timer}: executa uma callback em período fixo (\texttt{create\_timer}).
      \item \textbf{Callback}: função que roda em resposta a evento (timer, mensagem recebida, etc.).
      \item \textbf{Logs}: \texttt{self.get\_logger().info()/warn()/error()}.
    \end{itemize}
  
  \begin{center}
  \includegraphics[width=0.95\textwidth]{code2_timer.png}
  \end{center}
  \end{frame}

\section{QoS (Quality of Service)}
  
  % --- Slide 4: QoS — noções básicas ---
  \begin{frame}[fragile]{QoS — noções básicas}
    \begin{itemize}
      \item \textbf{depth} (History/KeepLast): tamanho do buffer de mensagens.
      \item \textbf{reliability}: \texttt{RELIABLE} (entrega garantida) vs \texttt{BEST\_EFFORT} (melhor esforço).
      \item \textbf{durability}: \texttt{VOLATILE} (apenas novas) vs \texttt{TRANSIENT\_LOCAL} (late-joiner recebe últimas).
      \item Perfis prontos: \texttt{QoSPresetProfiles.SENSOR\_DATA}, \texttt{DEFAULT}, etc.
    \end{itemize}
  
  \begin{center}
  \includegraphics[width=0.95\textwidth]{code3_qos_basic.png}
  \end{center}
  \end{frame}

\section{Implementação Prática}
  
  % --- Slide 5: Publisher (Python) ---
  \begin{frame}[fragile]{\href{https://docs.ros.org/en/humble/Tutorials/Beginner-Client-Libraries/Writing-A-Simple-Py-Publisher-And-Subscriber.html}{\uline{Publisher em Python (minimal\_publisher.py)}}}
  \begin{center}
  \includegraphics[width=0.95\textwidth]{code4_publisher.png}
  \end{center}
  \end{frame}
  
  % --- Slide 6: Subscriber (Python) ---
  \begin{frame}[fragile]{\href{https://docs.ros.org/en/humble/Tutorials/Beginner-Client-Libraries/Writing-A-Simple-Py-Publisher-And-Subscriber.html}{\uline{Subscriber em Python (minimal\_subscriber.py)}}}
  \begin{center}
  \includegraphics[width=0.95\textwidth]{code5_subscriber.png}
  \end{center}
  \end{frame}
  
  % --- Slide 7: Executando (do workspace) ---
  \begin{frame}[fragile]{Executar publisher \& subscriber}
    \begin{itemize}
      \item Em terminais separados, com o ambiente \textbf{sourced}.
    \end{itemize}
  
    \begin{columns}[T,onlytextwidth]
      \begin{column}{0.48\textwidth}
        \textbf{Terminal 1 — Publisher}
        \begin{lstlisting}[style=bashstyle]
  $ source ~/ros2_ws/install/setup.bash
  $ ros2 run my_py_pkg minimal_publisher
        \end{lstlisting}
      \end{column}
      \begin{column}{0.48\textwidth}
        \textbf{Terminal 2 — Subscriber}
        \begin{lstlisting}[style=bashstyle]
  $ source ~/ros2_ws/install/setup.bash
  $ ros2 run my_py_pkg minimal_subscriber
        \end{lstlisting}
      \end{column}
    \end{columns}
  \end{frame}

\section{QoS na Prática}
  
  % --- Slide 8: Ajustando QoS no código ---
  \begin{frame}[fragile]{QoS na prática (exemplo)}
    \begin{itemize}
      \item Combine \textbf{publisher} e \textbf{subscriber} com QoS compatível.
      \item Útil para sensores (\texttt{BEST\_EFFORT}) vs controle/estado (\texttt{RELIABLE}).
    \end{itemize}
  
  \begin{center}
  \includegraphics[width=0.95\textwidth]{code6_qos_practice.png}
  \end{center}
  \end{frame}

 \section{Atividade}
  \begin{frame}[fragile]{Atividade: criando um publisher e subscriber (Py)}
    \begin{itemize}
      \item No \texttt{ros2\_ws}, crie um pacote Python com \textbf{dois nós}:
        \textbf{publisher} envia dois inteiros; \textbf{subscriber} recebe e imprime a soma.
      \item Use \texttt{std\_msgs/msg/Int64MultiArray} (evita criar interfaces novas).
    \end{itemize}
  \end{frame}

  \begin{frame}[fragile]{Atividade: Passo 1}
    \textbf{1) Criar pacote e registrar executáveis}
    \begin{lstlisting}[style=bashstyle]
  $ cd ~/ros2_ws/src
  $ ros2 pkg create --build-type ament_python my_int_adder_py --dependencies rclpy std_msgs
  # Em setup.py, adicione em entry_points:
  # 'console_scripts': [
  #   'int_publisher = my_int_adder_py.int_publisher:main',
  #   'int_sum_subscriber = my_int_adder_py.int_sum_subscriber:main',
  # ],
    \end{lstlisting}
  \end{frame}

  \begin{frame}[fragile]{Atividade: Passo 2}
    \textbf{2) Estrutura de arquivos (mínima)}
    \begin{lstlisting}[style=bashstyle]
  my_int_adder_py/
    my_int_adder_py/
      __init__.py
      int_publisher.py
      int_sum_subscriber.py
    package.xml
    setup.py
    \end{lstlisting}
  \end{frame}

  \begin{frame}[fragile]{Atividade: Passos 3 e 4}
    \textbf{3) Implemente os nós}
    \begin{itemize}
      \item \texttt{int\_publisher.py}: cria \texttt{rclpy.Node}, \texttt{create\_publisher(Int64MultiArray, 'ints', 10)}, usa timer para publicar dois inteiros periodicamente.
      \item \texttt{int\_sum\_subscriber.py}: cria \texttt{create\_subscription(Int64MultiArray, 'ints', callback, 10)} e imprime a soma dos dois inteiros recebidos.
    \end{itemize}
  
    \vspace{0.5em}
    \textbf{4) Build e execução}
    \begin{lstlisting}[style=bashstyle]
  $ cd ~/ros2_ws && colcon build --symlink-install
  $ source install/setup.bash
  # Terminal 1:
  $ ros2 run my_int_adder_py int_publisher
  # Terminal 2:
  $ ros2 run my_int_adder_py int_sum_subscriber
    \end{lstlisting}
  \end{frame}
  
  
% Disable automatic section TOC for remaining frames
\AtBeginSection[]{}

\begin{frame}{Referências e Recursos}
\begin{itemize}
  \item \url{https://docs.ros.org/en/}
  \item \url{https://github.com/ros2/examples}
  \item \url{https://index.ros.org/}
\end{itemize}
\end{frame}

\begin{frame}
\centering
\Large Dúvidas? \\
\bigskip
\small \texttt{andre.marcato@ufjf.br}
\end{frame}

